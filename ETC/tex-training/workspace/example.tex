\documentclass[a4paper,11pt,final]{article}
\usepackage{makeidx}
\usepackage{graphicx}
\usepackage{wrapfig}
\usepackage{hyperref}
\title{The Fate of Fasilnor}
\author{Kencur - Fasilkom 2003}
\makeindex
\begin{document}
\maketitle
\tableofcontents
\clearpage
%\chapter{Fishiphenia}

\section{Awal}

\begin{wrapfigure}{r}{7cm}
  \centering
    \includegraphics[width=7cm]{images/pic1}
  \caption{The Toucan}
\end{wrapfigure}

Fisiphenia, negeri yang tidak jauh dari Fasilnor. 
Negeri hijau yang indah dan makmur dimana penduduknya ganteng dan 
cantik-cantik. 
Di negeri inilah Satron berkuasa dan memerintah dengan bijak.

\subsection{Satron dan Kerajaannya}
Pada suatu masa dimana Satron dikenal sebagai tiran, Fasiphenia sangat 
jauh dari keadaan saat ini. 
Zaman itu dikenal sebagai masa Peperangan Besar, peperangan antara 
Satron dengan Aria The Eldest. 
Masa tersebut sudah lewat, Satron menebus dosanya dengan membangun 
sebuah negara yang adil dan sejahtera. 
Dunia di mana persamaan hak antar ras dijaga dengan baik. 

\subsection{Mulainya Kekacauan}
Kedamaian terkadang hanya sebuah kehampaan, rumor-rumor tak pernah 
berhenti menjalar dan berkembang. 
Hari-hari ini pun demikian, dipenginapan dan pasar-pasar beredar rumor 
bahwa Satron Yang Agung baru saja kehilangan putri kesayangannya. 
Rumor-rumor itu mengatakan bahwa Yang Mulia Satron terus mengurung 
diri dalam aulanya dan ini sudah berlangsung selama berbulan-bulan.

Satron sangat bersedih atas kejadian yang menimpa Amelani, anak Satron 
dari hasil perselingkuhan dengan istri sang Saudagar dari Utara. 
Amelani merupakan anak kesayangan Satron yang Agung, seperti sayangnya 
Dewa Zeus pada Heracles putranya\ldots
Sejak saat itu langit di Fisiphenia lebih gelap dari biasanya. 
Anggur-anggur yang dibuat terasa lebih masam dari sebelumnya. 
Sapi dan domba tidak menghasilkan susu, dan panen gandum tahun ini 
terancam gagal. 
Semua orang resah.

Semua bertanya-tanya, ada apakah gerangan yang terjadi. 
Kejadian seperti ini pernah terjadi sekali, saat Satron kehilangan 
Yulion yang dikubur hidup-hidup oleh Aria The Eldest, seperti diceritakan 
turun-temurun dari generasi ke generasi.
Melihat keadaan seperti ini rakyat Fisiphenia pun berembuk. 
Mereka memutuskan mengirim perwakilan untuk menemui Satron Yang Agung 
yang sedang mengurung diri. 
Orang-orang yang ditunjuk sebagai perwakilan adalah\ldots

\section{Penghuni Fisiphenia}
Tinggalkan dulu keributan tersebut dan melihat ke sebuah sekolah di 
Fisiphenia. 
Dua orang tokoh yang bernama Randolf Hidler son of Randalf hasil 
perkawinannya dengan bangsa elf. 
Luthien Tinaviel putri bangsa Elf yang juga sedang bersekolah.

\subsection{Munculnya Sang Pahlawan}

Randolf adalah salah satu dari mereka yang garis keturunannya 
terjaga dengan baik. 
Kisah yang diceritakan dari kakeknya, Randuin Putra Aditarii 
adalah bahwa Hati Fisiphenia tertaut dengan Hati Satron. 
Kegelapan akan kembali menyelimuti negara ini. 
Randolf merasa terpanggil untuk menyelesaikan problema ini. 
Dia mengajukan dirinya sebagai perwakilan untuk menemui Satron. 
Namun sayang ia ditolak oleh Para Dewan Tetua Fisiphenia, 
disebabkan ia belum cukup umur serta blasteran.

Randolf, seperti bangsa Cheribouw memiliki penampilan seperti 
leluhurnya, hitam dan tetap kribo. 
Sifat blasterannya membuat ia memiliki kuping elf, tapi kulitnya 
hitam, aneh kan. 
Tampang juga pas-pasan, tetapi ia memiliki keberanian yang 
sangat besar.
"Buseng..gw lagi gw lagi..hina aja teruuusss..saitooon.." 
batin Randolf dalam hati karena disetiap cerita dia selalu ada 
dan selalu dihina-hina.

Memang begitulah kondisi di dunia Wimor ini, rasisme walaupun 
ditentang tapi dalam praktiknya masih ada. 
Tapi untuk bangsa Cheribouw lebih disebabkan oleh sifat mereka 
yang tengil dan selalu minta ditonjok.

Tapi salah seorang dari Dewan Tetua kembali mengingatkan pada 
senat, "Satron Yang Adil selalu menginginkan persamaan hak terhadap 
semua ras, tidak peduli berkulit hitam atau putih, kaya atau miskin. 
Tidakkah kita mengikuti ajarannya yang baik itu?".
Semua orang disana mengangguk... namun masih ada sebagian yang 
tampak ragu dan tidak setuju.

"Tapi kita semua tahu bahwa Satron tidak suka orang yang tengil 
dan selalu minta ditonjok kelakuannya" sergah salah satu dari 
dewan tetua yg bernama....

"Apalagi dia hitam dan blasteran kangguru dan doberman", tambah 
yang lain. 
"Betul, mirip Adunil hitamnya". 

\subsection{Satron Yang Agung}
Suasana council pun ricuh. 
Satu orang beradu mulut dengan yang lain. 
Tiba-tiba Satron pun muncul.

Rupanya beliau cuma numpang lewat mau pergi ke WC. 
Maklum sudah sebulan dia tidak keluar kamar.
Setelah itu dia kembali, berjalan ke tengah Aula Dewan. 
Dalam diam dia menatap setiap orang yang ada di situ. 
Seluruh ruangan terdiam, beku, begitu dingin. 
Tak ada yang berani bicara. 
Sunyi.

Tiba-tiba salah satu dari mereka mencoba memecah keheningan dengan 
menceletuk "Ke pantai yuk, Tron!".

Setelah itu ada seorang lagi yang berusaha mencairkan suasana, 
"Bego lu Tron".

"Udah udah..semuanya diam!!!" kata Satron.

Rupanya masih ada seseorang yang terlambat berkomentar tapi sangat 
ingin berkomentar, "Waah jangan gitu Troon.".
"Saat ini aku sedang dalam kesedihan yang luar biasa, bisakah 
kalian tenang?" tanya Satron. 
"Belum puaskah kalian para Dewan Tetua berdebat setiap hari, 
Hah? Hah? Hah?" Satron semakin emosi.

"...." Satron akhirnya speechless menatap tingkah polah anggota 
Dewan yang sama sekali tidak terhormat.

Tetap ada salah satu dari mereka yang masih berusaha 
beramah-tamah, "Jadi gimanna Tron?".

Tiba-tiba "Pyuuungg!!" telunjuk Satron mengeluarkan sinar yang 
langsung tepat menghunjam kening anggota Dewan Tetua yang kurang 
ajar itu.
Anggota Dewan Tetua yang terkena sinar Satron itu tiba-tiba mati 
seketika, lenyap tanpa bekas.

Satron pun kembali masuk ke Istananya. Tak lama kemudian para dewan 
kembali dalam suasana ricuh.

\subsection{Rencana Dewan Tetua}
"Udah gitu doank?? Lewat sini cuman pengen ke kamar mandi? Parah 
betul\ldots gw pikir ada apaan" kata seorang Anggota Dewan pada 
sebelahnya.
Ketua Dewan tetua yang malu untuk disebutkan namanya memukul-mukul 
palu ke mejanya. Selain untuk menenangkan sidang dewan juga 
sekaligus untuk membetulkan mejanya yg memang sedikit goyang.

"Hadirin sekalian...hari ini kita berkumpul di sini dalam rangka 
membahas sebuah masalah yang sangat membuat rakyat kita gelisah" 
kata Ketua Dewan Tetua.
"benar..benarr.. Jadi gimana nih coy...", timpal para anggota dewan 
lainnya.
"Apa yang terjadi pada Satron yang Agung? Apakah ada diantara kalian 
yang tau ada apa gerangan dengan Yang Mulia Satron?" tanya Ketua 
Dewan Tetua.

"Bukankah itu tujuan kita mencari perwakilan untuk dikirim menemui 
Satron. Agar kita tahu ada apa dengan Satron yang Agung?" 
celetuk salah seorang anggota Dewan Tetua. 

"Oh iya betul!! Kepala saya rasanya penat sekali, semalam saya baru 
mencoba meminum pil biru yang konon ke sohor itu" cerocos 
Ketua Dewan Tetua.
"Ah sial..kenapa gw ngomong.." batin Ketua Dewan Tetua karena tak mau 
di anggap 'tidak mampu' oleh Anggota Dewan yang lain. 
"Jadi bagaimana? Apa keputusan kalian?" kata Ketua pada anggota Dewan.

"Saya mengusulkan Kepala Intelijen negeri kita Anjariel of Euller 
untuk menemui Satron Yang Agung, dia sudah banyak pengalaman dalam hal 
mengorek informasi" jawab salah satu anggota dewan tetua.

Seorang anggota Dewan angkat bicara, "Sebelum itu, saya hanya ingin 
mengingatkan sebuah ramalan tua tentang kejatuhan kedua Satron. 
Nasib negara ini tertaut pada keberadaannya!".

Dalam ramalan tertua yang pernah ditulis oleh Fitriel, ada sekelompok 
orang yang dapat mengalahkan Ar-Mour. 
Namun kelompok orang tersebut sudah menghilang dan sulit ditemukan di 
muka Wimor ini.
Namun sayang, Fitriel sudah mati saat terjadinya perang besar dalam 
legenda\ldots dan hanya ada satu cara untuk mengetahuinya. 
"Kita harus menghidupkan kembali Fitriel!", celetuk salah seorang Dewan. 
"Tapi..Bagemana caranya? membaca saja aku sulit".

"Di di maen bola lagi yuk!", tambah anggota dewan lain. 

Tiba-tiba Wong Cen Lau, salah satu wise man dari negeri timur berkata, 
"Tenang saudara-saudara. Fitriel mempunyai banyak anak, dan setau saya 
salah satu anaknya masih hidup.".

"Oh iya siapa namanya? bisa di temukan dimana dia?" tanya Ketua Dewan Tetua. "Namanya adalah..Mithiel, jangan lupa pake h..." ujar Wong. Seluruh anggota Dewan serentak berkata "Ya iyalah pake h..kalo pake p kan jadi Pithiel".
Mithiel, daughter of Fitriel. Dialah satu-satunya yang mungkin mengingat ramalan tua mengenai kejatuhan kedua Satron, yang juga terkait dengan ramalan mengenai kekalahan Ar-Mour.
"Berarti sekarang kita perlu menemukan si Mithiel pake h ini" ujar Ketua Dewan Tetua. "Segera perintahkan pihak intelijen untuk menelusuri keberadaan Mithiel pake h ini" sambung Ketua Dewan Tetua lagi.
"Tidak perlu!" sela Wong. "Menurut catatan Fitriel yang saya ambil di tukang fotokopi, Mithiel which is the daughter of Fitriel ada di sekolahnya Randolf which is kribo item. ".
Tapi tanpa disangka-sangka sekolah Randolf sangatlah jauh. Terletak di ruang sidang di dalam Gunung Gridas, daerah ujung selatan di Fasilnor. Konon katanya daerah tersebut didiami oleh monster dan penghuni yang kejam.
"Alhamdulillah" syukur Anjariel of Euller namun agak terlambat, maklum dia dapat infonya terlambat. Dengan begini tugas untuk dirinya tidak jadi. Dia pun kembali menjadi cerpenis sekaligus narator cerita ini.
***

Beberapa utusan Dewan Tetua segera memulai perjalanan mereka ke sekolah Randolf. Mereka berjumlah 3 orang.
Setelah melewati 3 hari 3 malam penuh rapat yang sengit, akhirnya diputuskan lah nama-nama orang yang dikirim. Dia adalah Wong Cen Lau si bijak dari timur dan Randolf yang dikirim sebagai tumbal karena anggota dewan males melihatnya.
Satu lagi adalah Direzion son of Herzon Pemburu dari Selatan, dia datang menyusul, dikarenakan rumahnya lebih dekat dengan sekolah-nya Randolf.
Dia diajak sebagai penunjuk jalan oleh Wong Cen Lau. Awalnya dia tidak mau, tapi apa daya, Wong dan Randolf adalah temannya. Lagian tampang melas dari Randolf meyakinkannya untuk menempuh perjalanan menantang mara bahaya ini.
Dalam hati dia bertanya "Nih bocah Chreibouw bego banget sih, moso sekolahannya sendiri kagak tahu dimana tempatnya, kagak pernah sekolah apa ini bocah Chreibouw?".
Ternyata sekolah Randolf cukup unik. Walau tempatnya sudah diketahui umum, jalan masuknya berpindah-pindah setiap hari, sehingga hanya pemburu handal seperti Direzion yang mampu melacaknya tanpa kesalahan.
Sedangkan Randolf bagaikan orang linglung, dia seperti lupa klo dia sekolah di sana. Wajar saja ternyata ini adalah untuk pertama kalinya dia datang ke sekolahnya, selama ini ia selalu menitip absen.
"Di situ", ucap Direzion sambil menunjuk ke utara tempat mereka berada. Saat itu kabut cukup tebal meliputi daerah tersebut, sehingga hanya insting Direzon saja yang meng-guide mereka.
Namun, belum selesai Direzion berbicara, muncullah 3 makhluk yang siap menghadang mereka. Rintangan pertama bagi tiga sekawan.
Satu di antara ketiga makhluk tersebut serta merta menyerang dengan giginya yang tajam ke arah Randolf yang saat itu sedang melamun dengan bodohnya.
Di saat yang sama, Direzion dan Wong Cen Lau bekerja sama menyerang dua makhluk lainnya.
Direzion dan Wong Cen Lau menghabisi makhluk-makhluk tersebut dengan mudahnya. Sedangkan Randolf, sudah tidak suci lagi dibuat makhluk itu.
"jrooot" gigi tajam makhluk tersebut menghunjam ke rambut kribo Randolf, dalam sekali hingga terlihat kepala Randolf terbuka menganga. Namun tiba-tiba makhluk itu langsung terkapar, ternyata rambut bangsa Chreibouw memiliki racun ganas.
"Wahahaha, emang enak makan rambut gue? Belom keramas sejak lahir nih!", ujar Randolf.
Tapi tiba-tiba Randolf merasa lebih lemot dari biasanya. Ternyata makhluk tersebut adalah Prastudil, son of Prastudon. Seorang makhuk Geek yang gemar menghisap ilmu para korbannya.
Ternyata makhluk yg terkapar itu tiba-tiba berubah bentuk menjadi manusia. Rupanya selama ini ia dalam pengaruh kutukan. Berkat rambut bangsa Chreibouw yg ternyata lebih terkutuk, dia terbebas dari kutukan sebelumnya.
Tibalah mereka di depan pintu gerbang.
Ternyata pintu gerbangnya salah. Ini adalah pintu gerbang Pasar Malam Gunung Kidul. "Bang*tuuuuuttt*, salah jalan gw, kita muter lagi lah ke arah barat", tegas Direzion.
Tapi sudah terlambat, Randolf sudah keburu masuk ke dalam Pasar Malam, begitu mendegar alunan musik dangdut kegemarannya bertalu-talu dari arah dalam Pasar Malam.
"Aseeek, musik gue banget nih, Guys!", seru Randolf.
Dalam waktu 5 detik Randolf langsung hilang dalam keramaian Pasar Malam, bahkan Direzion san Pemburu dari Selatan tidak bisa mencium bau Randolf. Mungkin karena ia bangsa Chreibouw, yg memang susah dibedakan baunya dengan bau bangkai.
Namun, yang sebenarnya terjadi adalah..Pasar Malam tersebut adalah sebuah ilusi dari penunggu daerah tersebut...
Prastudil, teman baru mereka (dia sudah membuktikan dirinya melawan Randolf) berhasil menarik Randolf ke jalan yang benar.
Prastudil sangat menguasai medan tersebut, dan meminta untuk ikut menunjukkan jalan.
"oiii bangsaaad, terus gimana iniii??" maki Direzion pada Prastudil lupa disensor.
Rupanya mereka telah salah sekolah. Mereka telah masuk ke MIT (Margonaut Institute of Tenung), black magic school yang sangat berbahaya dan kesohor akan alumni2 buasnya.
Hanya Rektor-nyalah yang paling disegani oleh para almuni, yakni Margonout Ear One...
Tapi Rektor tersebut jarang masuk kampus.
Hal itu membuat murid terbandel dan tersampah sepanjang sejarah persekolahan pun hormat padanya. Dan sering mengajaknya bermain ping pong.
Margonaut The Ear One adalah salah satu dari The Five Sages of Fasilnor.
Selain itu, Margonaut adalah master pingpong. Terutama setelah kepergian Amelani, Margonaout tidak terkalahkan dalam urusan ping pong.
The fellowship pun mulai ragu akan perjalanan ini. Mereka cukup segan terhadap salah satu dari Five Sages of Fasilnor yang sangat kuat. Lagipula jika sudah masuk ke daerah kekuasaan mereka, sudah dipastikan sulit untuk keluar hidup-hidup.
Tuh kaan beneer..!! Randolf, Prastudil, Direzion \& Wong seketika itu dikelilingi oleh makhluk-makhluk cantik bertaring.
"Wanjeeer...apalagi ini..ada makhluk cantik bertaring segala??? Gag kebayang gw" celetuk Randolf tiba-tiba. "Eee buset,,," kata Prastudil, Direzion dan Wong kaget dengan celetukan-celetukan ajaib Randolf, yang biasanya selalu tidak penting.
Salah satu dari makhluk cantik itu dicubit pipinya oleh Randolf "iiih lucunya.." membuat mereka berang dan menggigit liar-liar asik.
Ternnytaa Randolf sama sekali tidak memiliki ketakutan terhadap makhluk-makhluk cantik bertaring tersebut. Justru sebaliknya merekalah yang menjadi takut dicubit oleh Randolf. Memang aneh kekuatan yang dimiliki oleh Randolf Hidler son of Randalf...
Setiap pipi manis yang dicubit oleh Randolf menjadi kasar, rusak dan basi.
Teriakan-teriakan liar mulai bersahutan, makhluk-makhluk itupun langsung kabur menjauhi Randalf. Konsekuensinya adalah, kawanan Randalf pun menjadi selamat dari serangan mereka. Mereka pun melanjutkan kembali perjalanan mereka.
Di tengah perjalanan, mereka melihat seorang gadis nyangkut diatas pohon. "Hey guys! bisa tolong turunin gw which is dari atas pohon ini?".
"..." mereka berempat saling berpandangan. "Brebet" dalam sekali loncat, Direzion sudah berada di atas pohon dan membawa gadis itu turun ke bawah. "Apa yang sedang Adinda lakukan di atas pohon?" tanya Direzion sok gombal.
"Well, tadinya gue lagi mau ngambil rempah-rempah di hutan ini untuk ramuan. Tetapi entah dari mana tiba-tiba ada sekelompok makhluk cantik bertaring yang berlari seperti mengejar gue! Gue langsung aja manjat ni pohon", ujar gadis which is.
"Oooh gitu.." kata mereka (berempat bukan berlima) serempak. "Btw..kamu siapa?" tanya Randolf mengulurkan tangan berusaha mengajak kenalan yang langsung di-'cie cie'-in sama yang lain, gag penting sih, tapi lucu aja.
"Mithiel Haren", ujar gadis tersebut seraya menyambut uluran tangan Randolf. Seketika mereka berempat terperanjat, "Mithiel Haren daughter of Fitriel?", mereka serempak. "Iya, Mithiel. Jangan lupa pake h ya.", balas gadis tersebut.
Sebelum mereka sempat beramah tamah lebih lanjut. Tiba-tiba "Brebet!! Slepet!!" Mithiel diculik oleh Ray Shaiton, murid kesayangan Margonaut the Ear One. Sangat cepat, bahkan Randolf menyadari retsletingnya terbuka, Ray Shaiton telah hilang dari situ.
Suara teriakan Mithiel Haren terdengar bergema di antara pepohonan yang tinggi, "Hey guys! Selamatkan aku ya, which is dari penculikan orang aneh ini. Thanks anyway.".
Dalam waktu se-brebetan-an Ray menghilang dari pandangan. "E buset..siapa itu tadi?" tanya Direzion pada rekannya. "Kayaknya gw tau.." kata Wong yang dikenal berpengetahuan luas. "Dia adalah Ray..Shaiton, anak murid Margonaut" lanjut Wong.
"Tidak mungkin!" balas Randolf terkejut. "Kakekku pernah bercerita tentang Klan Shaiton. Mereka hidup di dalam bayangan dan kegelapan, which is keberadaan mereka selama ini hanya sebatas legenda. Halah kok ketularan jadi which is juga sih gw".
"Nah, itu dia alasannya mengapa Ray begitu cepat bergerak.", ujar Prastudil secara analitis, "Untuk mengalahkannya, kita harus mempersiapkan jebakan baginya di pagi hari!".
"Klan Shaiton? in Great danger we are kawan.. Waspada kita harus...", kata Prastudil memperingatkan kawan-kawanya.
"Dia tidak akan masuk ke dalam perangkap," bantah Wong. "Justru kita harus mengejarnya dengan cepat!" Lalumereka berempat melesat menelusuri jejak Ray Shaiton. Setelah berapa lama mereka menemukan sebuah keranjang rempah bertuliskan which is.
Di keranjang itu terdapat sebuah selebaran. Selebarannya berisi tentang perkawinan Jalesun dan calon istrinya. Ternyata Mithiel temannya Jalesun! Fellowship cukup terkejut, karena dalam legenda, Jalesun sudah mati terhisap lobang yang dibuat Ar-Mour.
"Jalesun yang dalam legenda dia adalah seorang yang ahli menghilang dan bergerak secepat kilat?" kata Randolf heran. "Berarti dia masih hidup?" tanyanya kemudian.
"Semua orang mengira ksatria legendaris itu telah mati. Kalau begitu, kita lewat sini! Kita belum terlambat," seru Wong. Tak berapa lama mereka menemukan lukisan 'pre-wed' Jalesun dan calon istrinya, beserta kartu ucapan selamat dari Mithiel.
"Hehe, lama ga keliatan tau2 nyebar undangan" celetuk Prastudil pada Randolf tiba-tiba.
"Hooiiii! kejar gw lagi dong!", tiba-tiba Ray Shaiton muncul di depan mereka. Membuat mereka melongo semua saking kagetnya. Dan "Plop!!" Ray Shaiton hilang kembali.
"Guys, ayo dong jangan cuma ngebahas Jalesun," teriak Mithiel samar-samar. "Aku masih diculik nih. Ayo ikutin si penculik ini, which is ke arah gunung, menara yang paling tinggi di sini. You can see that.".
Keempat pendekar kita melongo lagi. Lalu tiba-tiba terdengar suara, "Si Mithiel ngomongin menara apa si?" tanya Randolf gag ngerti. "Pras, Rez..Wong..tau gag lo itu menara apa??" tanya Randolf kemudian.
"Kemungkinan yang dimaksud adalah Menara Suci Kaum Shaiton," gumam Wong sambil memejamkan mata. "Dan arahnya kesana." Dia menunjuk ke arah papan nama 'Menara Suci Kaum Shaiton, 10 KM lagi'.
"Umm... Menara Suci kaum Shaiton? Kok sepertinya kontradiksi ya?", ujar Prastudil.
"Ah, deket klo gitu, lurus aja Coy!" kata Randolf sok tahu. Baru 10 langkan berjalan, mereka terhenyak, ternyata di depan mereka terdapat jurang. Mereka dapat melihat Menara itu, namun mereka bingung bagaimana melewati jurang tersebut.
Ketika Randolf melihat ke bawah tampak seperti lautan manusia. Ternyata itu adalah jurang tempat pembuangan bencong. Tentu saja Randolf tak mau terjatuh ke jurang itu, karena...
Ia sedang puasa, ia takut tidak dapat menahan nafsunya yg mana dapat membatalkan puasanya hari ini.
Selain itu, kalau jatuh, dia bakal mati. Dia tidak ingin nasib serupa dengan ayahnya, yaitu mati dalam kegelapan.
"Hmm, kita harus mencari jalan lain," seru Direzion. Mengikuti tips and trick dari Wong, dia berusaha mencari papan nama 'Jalan Lain', tapi tidak ketemu.
"Bagaimana kalau kita mencari jalan alternatif saja?", pikir Wong. Dan ternyata, terdapat papan nama "Jalan Alternatif" dalam jarak pandang mereka.
"Coba kita bertanya ke Googlarium!" celetuk Randolf. Googlarium adalah benda sakti yang bisa mencari semua jawaban di dunia Wimor. "Ada yg tahu alamatnya Googlarium?" tanya Direzion.
"Goglarium.com" seru Randolf karena kesal setiap kali ingin berkontribusi dalam cerita ini selalu keduluan.
"Goblog elo Ndolf, mana ada alamat seperti itu, itu emang protocolnya apaan?" maki Direzion.
Singkat cerita, Googlarium menjawab "Telusuri Jalan Rahasia di balik dunia, di mana Matahari Yang Perkasa tak akan pernah menyentuh tanah kaki di mana kalian berjejak.".
"Lain kali jika ingin menghubungiku, mintalah bantuan Anjariel of Euller di nomor 007, dia tahu dimana aku berada" sambung Googlarium.
"Aku juga buka sms hotline lho. Sms aja ke 212. Semua sms yang kamu terima, langsung dari henponku!" seru Googlarium sebelum menghilang ditelan asap putih.
Tiba-tiba seekor tikus raksasa keluar dari tanah, kemudian kabur. "Aku tahu maksudnya," seru Direzion sambil tersenyum, "Liang-liang Tikus Edwing selalu bermuara di kaki menara-menara!" Lord Edwing adalah penguasa tikus-tikus raksasa di Fasilnor.
"Sial, sepertinya kita memang harus melewati Lorong Tikuzon di bawah tanah", kata Direzion, "Walaupun aku tahu, banyak mahluk berbahaya yang merangkak dan siap menerkamn disana". Akhirnya dengan gemetaran, The fellowship maju menuju Lorong Tikuzon.
***

Setapak demi setapak lorong gelap itu disusuri oleh empat sekawan itu demi menyelamatkan Mithiel Haren. Entah apa yang menanti mereka di depan.
Baru masuk semeter tiba-tiba "Plup!" mereka sudah keluar di ujung jurang yang lain. Rupanya ini lobang ajaib.
"Wahai jurang dan segenap isinya, lain kesempatan aku akan mengunjungimu! Sekarang ada urusan yang jauh lebih penting dari sekedar menuruti nafsu!" batin Randolf dalam hati.
"Wahai Shaiton! Dengan kekuatan Yang Mulia Satron, kami akan menghukummu!" seru Direzion lantang. Mereka melanjutkan perjalanan.
Sesuai jawaban dari Googlarium, menelusiuri jalan tikus mendekatkan mereka pada tujuan mereka. Di jalan perempatan pertama yang mereka temui, terdapat tanda jalan ke arah kanan: "Menara Suci Kaum Shaiton, 14km" .
Mereka belok ke kanan dengan percaya diri, berjalan ringan melawan angin sepoi-sepoi yang mengibaskan rambut mereka, diiringi lagu rock alternative yang melankolis. kayak para jagoan di tipi-tipi.
Ternyata belokan ke kanan malah membawa mereka menuju lorong-lorong yang lebih gelap dari sebelumnya. Memang daerah sini terkenal dengan lorong yang menyesatkan. Dan juga terkenal dengan minuman khasnya Jus Shaiton Arab.
"Randolf mana ya?" Wong menyadarkan Direzion dan Prastudil akan Randolf yang menghilang. "Sori men, gw salah belok arah ke kiri, soalnya gw masih bingung mana yang kanan mana yang kiri".
"Kiri tuh arah tangan lo yang biasa dipake cebok!" dengus Direzion pada Randolf yang semakin kebingungan akan pernyataan kawannya itu. FYI, Randolf selama ini belum pernah cebok menggunakan tangan.
Ya, biasanya Randolf menggunakan tisu. Kayak orang-orang Barat gitu lho...
Kalo ga ada tisu, rambutnya pun bisa jadi pengelap.
Prastudil yang pintar kemudian mengeluarkan sebuah kitab dari tasnya, "Kalau begitu baca ini!" Kitab itu bertuliskan 'Kanan Kiri for Dummies', halaman pertama bertuliskan 'Jangan kelamaan cerita yang ga penting cuy, lanjut petualangannya!'.
Tiba-tiba terdengar suara dari kejauhan.. "Hei guys.. tolong selametin gw dong which is ada diatas sini!!". Jelas sekali itu suaranya Mithiel pake h.
"Ah ini pasti perangkap!" bisik Wong yang sangat berpengalaman dalam pembebasan sandera. "Tidak mungkin semudah itu, kita pasti akan dijebak!".
"Ah udah lah langsung dah kita ke atas", kata Randolf sambil dengan sembrono menginjakkan kaki ke dalam menara Shaiton. Fellowship yang lain pun terpaksa mengikutinya, padahal mereka tau betapa menyeramkannya menara Shaiton.
"Guuys.. ini beneran gw, Mithiel, pake h seperti pada thompson" Teriakan itu pun semakin jelas terdengar.
"Iya, sepertinya memang itu Mithiel", seru Prastudil, "sebab kita telah menempuh jarak 14km sejak belok kanan tadi. Tapi hati-hati, siapa tahu masih ada jebakan.".
Benar saja kata Prastudil! Karena tiba-tiba Randolf berteriak "Ah sial! gw nginjek tae kuda".
"Tae kuda bukan ya?" tanya Randolf, "Coba dulu baunya" kata Randolf baunya sih tae kuda. "Kalo rasanya?" lanjutnya lalu mencicip dikit "Wanjeng beneran tae kuda..Setan..Siapa si yang naro tae kuda di tengah jalan menara???" Umpat Randolf kesal.
Tiba-tiba Wong ikutan mencicipi tae kuda tersebut. "Hmmh.. rasanya agak aneh. Tae kuda di kampung gw lebih manis soalnya" Celetuk Wong.
Dengan penasaran akhirnya si Direzion dan Prastudil ikut-ikutan. "Bener euy..kurang manis..." Kata mereka berdua serempak.
Prastudil pergi ke kantin menara Shaiton. Lalu kembali membawa gula merah dan mencampurnya dengan tae kuda. Dia mempersilakan teman-temannya untuk mencobai rasanya apakah sudah manis.
"Kalian itu ngapain sih?" Tiba-tiba Anjariel of Euller datang dan bertanya.
Dilihatnya tae kuda dicicipi mereka. Anjariel pun mengeluarkan butir-butir tae embek dari kotak makannya. "Dicampur enak nih pasti". Tiba-tiba terdengar suara teriakan Mithiel..
"Hooi.. selametin aku dulu dong! Gak pake lama!".
Mereka berempat pun bergegas lari ke arah sumber suara. Mereka sadar telah membuang-buang waktu. "Gawat...nyawa Mithiel harus kita selamatkan" kata Randolf diiyakan oleh yang laen. "Benar demi Paduka Yang Mulia Satron" timpal Prastudil kemudian.
"Eh gw ga diajak nih?" tanya Anjariel of Euller sambil muntah-muntah saat tersadar dia telah melakukan perbuatan hina.
"Udah gag usah banyak cebret..langsung brebat lah" kata Randolf pada Anjariel. Anjariel pun berlari menyusul mereka berempat.
"Kalo gitu naek kereta kuda saya saja" Anjariel pun bersiul memanggil kereta kudanya. Sebagai kepala intelijen ia mendapatkan kendaraan dinas, yang tentu saja gratis pula biaya perawatannya.
Memang enak ya jadi pejabat, pikir mereka berempat melihat kereta kuda Anjariel.
Ketika mereka naik kereta kuda Anjariel, the fellowship merasakan keanehan. Kuda-kuda ini tidak seperti biasanya.
"Kok kudanya item dan kribo sih?" tanya Randolf. "Iya dulu induk kuda2 ini pernah diperkosa oleh orang gila yg bernama Randuin son of Aditarii" jawab Anjariel sambil mengendali kuda supaya baik jalannya.
Tuk tik tak tik tuk tik tak tik tuk tik tik tak tuk.. suara sepatu kuda tuk tik tak tik clep clep clep! nginjek tae kuda. Mereka pun sampai lah di..
Pasar Rumput Fasilnor, untuk mengisi rumput kereta kuda yang semakin menipis, "Pertamax ya mas!!" pinta Anjariel ke petugas SPRU (Stasiun Pengisian Rumput Umum).
"Woi guuys! gw dah jenggotan nih nungguin kalian which is lama banget nolongnya" Suara teriakan itu terdengar lagi.
"Oooo, kita mo nyari Ray Shaiton ama Mithiel? Bilang dong dari tadi" Anjariel langsung mengeluarkan jurusnya dimana langsung terkoneksi ke Googlarium dan langsung menemukan lokasi mereka. Anjariel pun mengedipkan matanya, dan "Plop!" sampailah di ...
Gerbang menara Shaiton lagi. "Loh kok balik lagi sih" kata Randolf. "Hmm, ternyata kita sedang dipermainkan" Wong menimpali. Rupanya menara itu benar-benar Shaiton alasion, The fellowship pun dibuat bingung karenanya.
Akhirnya mereka pun memasuki menara Shaiton. Mereka terkejut karena melihat di dalamnya tidak ada apa-apa. Hanya sebuah tangga melingkar yang cukup terjal untuk dinaiki. Tanpa babibu mereka langsung menaiki tangga tersebut tanpa menyadari bahayanya.
Tiba-tiba Ray Shaiton terbang ke arah Wong dengan cepat mendaratkan pukulannya. Kapow!.
"wadooww... setann.. siapa tuh yang nampol gw", umpat Wong sambil mencabut giginya yang patah karena pukulan tersebut.
Pertarungan pun tak dapat dihindarkan walaupun mereka sedang menaiki tangga. Ray Shaiton tidak lupa membawa anak buah yang tak kalah ganasnya, Mirfan the Red Beard. Dwarf yang satu ini sangat lihai memakai Kampak Naga 212.
Tanpa babibu, Mirfan menyerang dari sektor kiri pertahanan fellowship. Dia berspekulasi dengan melancarkan Kampak Naga 212 dan... Crot! Anjriel kehilangan kepalanya.
Tapi Anjariel tanpa kepala membalas Mirfan dengan sihir putih yang melesat dari tangannya. Mirfan the Red Beard terlempar jauh ke dasar menara. "Aku sudah lama mati," seru Anjariel kepada keempat temannya, "jadi tidak bisa mati lagi. Tenang saja Bro".
Selain tidak bisa mati, Anjariel memang susah matinya.. daya tahan tubuhnya mirip bagai kecoa yang ga mati-mati meski uda di injek-injek ampe gepeng...
"Yah mukanya memang mirip sama kecoa sih", celetuk Wong sambil bersiap-siap menghadapi serangan berikutnya dari Mirfan dan Ray. Kali Ray membisikkan aba-aba menyerang pada Mirfan. Mereka pun langsung maju menyerang berduaan dengan mesra.
"Kok bisa, yang tadi?" tanya Prastudil kepada Anjariel. Anjariel menjawab, "Aku mengikatkan nyawaku mengabdi pada Satron. Sebenarnya aku sudah mati ratusan tahun lalu. Tapi aku akan terus hidup untuk melayaninya.".
Pembaca mungkin jadi bingung, tapi ceritanya memang begitu. Dalam dunia perdongengan, semuanya mungkin. If there is a will, there is a way.
"Karena hidup adalah perbuatan," lanjut Anjariel sambil menatap menyongsong masa depan.
"Setan, banyak omong kalian", seru Ray sambil berpegangan tangan mesra dengan Mirfan, "Kita lagi mo nyerang nih, siap siap dong ah". Kemudian akhirnya mereka melanjutkan serangan ke jantung pertahanan the fellowship dari sebelah kiri.
Namun sebelum itu dengan kekuatannya Anjariel menyambungkan lagi kepalanya. Prosesnya tidak terlalu jelas, karena selama proses penyambungan tubuh Anjariel bersinar sangat menyilaukan. Dan ketika sinar itu hilang Anjariel kembali in one piece.
Prastudil lalu mendapatkan ide cemerlang, "Randolf, kau selamatkan Mithiel. Kami berempat mengalihkan perhatian. Karena ... hanya kau yang bisa ... bersembunyi dalam kegelapan." Kemudian Prastudil membalikkan badan dan ketawa-ketawa gila sendirian.
Randolf lalu mengeluarkan jurusnya. Dia mencari jalur yang gelap menuju tempatnya Mithiel.
Saking hitam kulitnya, dia tampak seperti menghilang ditengah bayangan benteng. Bahkan Ray yang punya mata Shaiton Alas juga tidak dapat melihat kemana perginya Randolf, sehingga ia marah dan mulai menyerang kembali the fellowship.
"Syuuuung" "Blats" Tiba-tiba Anjariel menembakkan sinar laser dari telunjuknya ke arah Ray dan Mirfan. Namun sayang mereka bisa menghindar dan sinar laser menghancurkan pilar2 di menara Shaiton.
Plak! Ray memukul Direzion. Kapow! Wong membalas Ray dari belakang. Jeb!! hantaman dari Mirfan untuk Wong. Crot!! Prastudil meludahi Mirfan. Crash! Anjariel salah sasaran menghantam kepalanya Prastudil.
Sementara itu, Randolf sampai di tingkat paling atas. Di hadapannya terdapat tiga Mithiel yang terikat. Ini sihir! Yang mana yang asli? Pikirnya.
Mithiel pertama bilang:"Ini saya Mithiel asli, Mithiel pake m seperti pada thompson".
Mithiel di ujung berkata; "Ku tak perlu berkata apa-apa untuk membuktikan. Karena memang akulah yang asli".
Mithiel yang kedua (di tengah) berucap; "Saya yang asli, Mithiel pake h seperti pada thompson". Randolf jadi ling-lung.
Lalu Mithiel yang di ujung berkata lagi, "Aku Mithiel yang asli, who is pake i seperti pada thompsoni".
Mithiel yang pertama juga tak mau kalah, "Dia palsu, aku Mithiel asli where is paling smart".
"No smoking!" kata Mithiel kedua saat Randu akan mengeluarkan rokoknya, saat ia coba rehat sejenak, karena ia pusing kebingungan.
"Ah tolol!" tiba-tiba ada suara dari belakang, ternyata Prastudil.
Ternyata Prastudil berhasil lolos dari kepungan Ray dan Mirfan, kemudian ia berinisiatif untuk membantu Randolf yang kebingungan. Sementara Direzion, Anjariel dan Wong masih bertarung dengan Ray dan Mirfan.
Namun sayang sekali akibat kepalanya terhantam tidak sengaja oleh serangan Anjariel sebelumnya, Prastudil mengalami gegar otak yg membuat ia jadi tolol.
"Eh tolong mithiel yang asli angkat tangan dong", pinta Prastudil. Mithiel kedua dan ketiga angkat tangan. "Naah, berarti kalian berdua palsu. Kalian kan harusnya masih terikat, kok bisa angkat tangan?", kata Prastudil.
Randolf pun percaya terhadap penilaian Prastudil, sehingga ia akhirnya melepaskan ikatan tangan Mithiel pake h yang berada di tengah. "Nah sekarang kau bebas Mithiel. Ayo ikut kami", kata Randolf penuh wibawa.
"Huahahahahaha", tiba-tiba Mithiel yg dilepaskan oleh Randolf berubah menjadi Troll raksasa. Ternyata dia adalah Aditron the Bringaz, troll sekutu klan Shaiton.
"Tolol lu Randolf! Mithiel yang asli itu yang nomer 1, berarti yang di kiri! Ngapain lo lepasin yang tengah!!", umpat Prastudil dongkol. Sementara itu Mithiel yang asli cuman bisa terbengong2 melihat ketololan Randolf.
"Kalian itu kok datangnya which is lama sekali sih," kata Mithiel yang sekali.
"Duarrr", Aditron tiba-tiba mengayunkan gadanya ke tengah-tengah mereka dan membuyarkan perseteruan mereka sejenak.
Duarr Gada itu malah menghajar Aditron. Harap maklum, karena ini debut pertamanya Aditron. Duarr Gada itu sekali lagi menghajar Aditron, Aditron pun tewas seketika.
Karena Aditron baru muncul sudah langsung mati, maka Randolf pun memberanikan diri melepaskan ikatan Mithiel yang asli, yang pertama. "Nah kali ini bener kan gw ngelepasinnya?".
Namun tiba-tiba Aditron bangkit kembali, rupanya Margonaut the Ear One, membangkitkan kembali Aditron, mengingat stok pasukannya yg minim.
Aditron berkata: "Saya tidak akan jatuh lagi ke lubang yang sama!" tapi tiba-tiba Duarr Gada yang ia ayunkan kembali menghajar dirinya terpental jauh dan jatuh ke jurang.
Aditron sekali lagi setelah baru muncul langsung mati lagi.
Margonaut The Ear One pun sudah kapok. Sebodoh-bodohnya Randolf, lebih bodoh Aditron.
"Kapan-kapan saya tidak akan menerima lagi Troll sebagai anak buah saya lagi", geram Margonaut the Ear One dari kejauhan di tempat yg tak terlihat.
Singkat cerita, Randolf, Prastudil, dan Mithiel berhasil kabur dari menara. Sementara itu Wong, Direzion, dan Anjariel masih bertempur.
Srosot! Mirfan menampar Direzion. Dhuarr! Wong ngaget-ngagetin Mirfan sampe pingsan. Slep!! Ray meracuni Wong dengan nasi padang. Brebes! Anjriel tanya sama Ray: "Brebes itu dimana, Ray?". Ah tolol! kata Prastudil.
"Coba tanya Sapion!" jawab Ray, "katanya deket-deket kampungnya di Tegaluvell" sambung Ray lagi.
Mumpung Ray sedang sibuk melayani pernyataan Anjriel yang gak mutu, Anjriel buru2 membuka Googlarium sekaligus membawa Wong dan Direzion ke tempat Randolf dkk berada. "Hihihi kapan lagi bisa ngadalin si Ray", gumam Anjariel.
"Kita harus segera pergi!" seru Prastudil. "Tenang cuy, kita punya tumpangan," kata Direzion, "bangsaku beraliansi dengan Lord Edwing, dan aku tahu sedikit bahasa mereka." Kemudian ia mencicit, dan datanglah enam tikus raksasa menghampiri mereka.
"Segera menuju Fisiphenia, ke Gedung Dewan tetua" perintah Anjariel. Ia pun segera mengirimkan surat lewat burung onta mengabarkan mereka segera menuju Fisiphenia.
"cit ciitt cit cuitt cittt (tolong antarkan kami ke fisiphenia)", kata Direzion dalam bahasa tikus. "Ciiiitttt (oke boss!)", kata keenam tikus serempak.
***

Akhirnya sampailah tikus-tikus itu di Fisiphenia bersama Fellowship dan Mithiel.
Segera mereka melaju ke arah gedung Dewan Tetua di samping Colosseum di jantung kota. Untuk ke sana mereka harus melewati lalu lintas yg sedang padat-padatnya.
"Bang2, kok macet gini sih??", tanya Randolf ke tukang teh botol di salah satu ruas jalan. "Maklum dek, ada si komo lewat! Mending adek jangan ditengah jalan, kalo gak mau keinjek", kata si tukang teh botol tanpa menoleh.
Melihat kondisi jalanan yg semakin macet parah banget. Anjariel segera melaukan telepati kepada koleganya di Kepolisian Lalu Lintas Fisiphenia untuk mengawal rombongan mereka langsung ke gedung Dewan Tetua.
Tapi pak polisi juga bingung seperti lagu dulu Macet lagi.. macet lagi.. Gara-gara sikomo lewat.. Pak polisi jadi bingung.. urang-urang ikut bingung kalau penulis tidak salah ingat, begitu lagunya.
"Sial!" Anjar pun akhirnya terpaksa mencari 6 ojek untuk mengantar mereka ke gedung Dewan Tetua dengan cepat. "Ojek emang bikin bangga" celetuk Direzion.
Akhirnya sampailah mereka ke gedung Dewan Tetua, dengan tergesa-gesa, mereka berenam masuk untuk menghadiri sidang Dewan Ketua.
Sesampainya di ruang sidang, Dewan Tetua menyambut fellowship bak pahlawan.
Segeralah mereka memulai sidang untuk mendengarkan ramalan Fitriel yg akan ditututkan Mithiel. Mithiel sendiri yg sedang asyik ngupil di pojokan, masih bingung kenapa dia bisa ada di sini, maklum selama perjalanan fellowship tak sekalipun menjelaskan.
Akhirnya Direzion berbisik pada Mithiel was wes wos ramalan was wes wos Satron was wes wos akhirnya... Mithiel jadi geli-geli asik (kupingnya).
"Tok-tok!" Ketua Dewan Tetua membuka sidang, "Kali ini kita akan mendengarkan ramalan penting mengenai keberlangsungan negeri kita yg menyangkut hajat hidup orang banyak, untuk itu saudari Mithiel kami mohon untuk segera naik ke mimbar" lanjutnya.
Mithiel pun segera naik ke atas mimbar, bukan mimbar untuk khotib kutbah lho. Mimbar ini berada di tengah-tengah ruang sidang, tepat persis di depan meja Ketua Dewan Tetua.
"Saudari Mithiel, langsung saja ke intinya, karena kami semua bingung mau apa lagi, bisakah Anda ceritakan ramalan dari Ibunda Anda Fitriel the White?", tanya Ketua Dewan tanpa basa-basi.
"Hmm, maksud Anda apa ciiyy? Which is gak ngerti...", ujar Mithiel..
Namun tiba-tiba lampu di gedung dewan mati semua. Mithiel yg awalnya cengar-cengir tiba-tiba terbang ke atas tak tersadarkan diri dan diselimuti sinar yg terang, mulutnya pun bertutur...
"Pecinta Yang Terjatuh tidak akan pernah bisa dibunuh oleh manusia, Tapi mereka yang terbuang akan menuntut balas...
..Ketika Putri hilang, Dia Yang Menebus Dosa akan sampai pada senjanya. Di balik mawar yang indah, tertidur sebuah bangkai yang keji...
..Pada masanya, kegelapan akan menyelimuti hati, Kebaikan dan Persamaan selama tiga zaman akan terhapus dalam beberapa tahun...
..Menara-Menara Tinggi akan diruntuhkan, Lima pilar yang suci akan kembali bersatu...
..Lima sulur yang jahat akan muncul ke permukaan, Fasilnor yang indah akan diselimuti api....
..Namun setelahnya akan tumbuh kembali bunga-bunga.".
Setelah Mithiel sadar, ia terjatuh bebas ke atas mimbar Gubrakks!! menimpa Randolf.
Tiba-tiba lampu gedung menyala kembali dan Mithiel berhasil membuat dirinya dan Randolf tak sadarkan diri.
Semua orang yang berada di dalam ruangan terdiam..Sunyi...Tatapan bingung muncul di masing-masing muka mereka..
"Notulen!!!" teriak ketua Dewan, "Tadi itu kamu catat gak?" tanyanya lagi.
Notulen mengangkat jempolnya kepada ketua Dewan.
Tidak hanya satu jempol, tetapi tiga jempol, dua jempol tangan dan satu jempol kaki nya tanpa basa basi.
Ternyata sang notulen ngasal ngangkat jempolnya. "Ya ampun, gue lupa nyatet tadi dia ngomong apaan!!", ujar Notulen Dewan Tetua..Semua orang kembali terdiam...
"Dasar bego..udah capek capek dibawa kesini malah gag di catet.." Umpat para anggota Dewan yang lain.
"Iya, masih nubi nih kakak... gebein dunk...", jawab Notulen-Nubi..
"Tenang-tenang" kata Mithiel sambil bangkit memegangi pinggangnya yg keseleo, "Saya punya hardcopy-nya kok, Ibu saya menaruh masternya di tempat fotokopian" lanjutnya.
"Trus ngapain tadi repot-repot bawa si Mithiel ke Dewan Tetua" umpat the Fellowship bersamaan.
"Pake acara pala gw kepenggal lagi, trus disambungin lagi, susah tauk itu!!" geram Anjariel of Euller.
"Hehe, gak tau juga, aku cuman nurutin kata Ibunda saja, which is harus heboh kalo menyampaikan ramalan", ujar Mithiel dengan ceria.
Sementara mereka ribut-ribut, Prastudil sedang berpikir keras tentang arti ramalan itu. Biasanya, Prastudil bisa memecahkan teka-teki semacam ini.
karena dari semua orang di situ, Prastudil adalah orang yang paling jago, secara dia ikut olimpiade matematika gitu loh.
Melihatnya berpikir, Randolf pun pura-pura ikut berpikir dengan menggaruk kepala. Wong pura-pura berpikir dengan memegang dagunya.
Prastudil mulai berbicara, "Hmm, sepertinya aku mulai mengerti apa maksud dari ramalan tersebut...Walau masih agak ragu..".
"Yaa, aku juga," timpal Randolf agar kelihatan pintar. "Aku juga," sambung Wong.
"Eee..udah deee.." kata Prastudil. "Pecinta Yang Terjatuh adalah Ar-Mour" lanjutnya sambil mengingat dan mencocokkan dengan legenda Perang Balgebunen.
"Benar sekali, sama persis dengan pikiranku," sambung Randolf. "Persis," seru Wong.
"Sebentar, saya coba koneksi ke Googlarium, biar ada referensinya" timpal Anjariel menanggapi. Kemudian dia langsung melakukan telepati ke Googlarium.
Menurut sejarah perang Balgebunen, Ar-Mour jatuh cinta pada Amelani. Tetapi ia malah membunuh Amelani. Disitulah maksudnya Ar-Mour terjatuh.
"Eh lanjutannya apa lagi?" tanya Wong, "Bentar coy, gw fotokopi dulu dari masternya di tempat fotokpian" jawab Direzion.
"Ni dia..udah gw potokopi semua, lengkap sob" kata Direzion tak lama kemudian pada Wong.
"Mereka yang terbuang bisa saja monyetnya suhu Acus, bisa juga babi atau tikus. Yang pasti bukan manusia, karena Pecinta Yang Terjatuh tidak akan pernah bisa dibunuh oleh manusia" Lanjut Prastudil.
"Atau mungkin juga biji-biji Aria the Eldest" timpal Anjariel, "Aria the Eldest......" yg lain kompak terkejut mendengar nama itu.
"Kalau begitu..." Semua mata melirik foto Satron yang terpampang di ruangan sidang.
Semua orang dalam ruangan itu terdiam lalu menerawang, menyelami pikiran masing-masing, mengingat-ingat cerita perang antara Satron dan Aria the Eldest.
"Satron sudah ditakdirkan berperang dengan Ar-Mour", Ucap Prastudil menganalisa semua kata-kata ramalan itu.
Anjariel pun menjelajahi dunia maya lewat telepatinya dengan bantuan Googlarium dia mendapatkan rekaman pertarungan Balgebunen yg dipimpin wasit Jimmy, dimana Satron dibuat terluka hatinya oleh Ar-Mour yg membunuh Amelani, di situs yg sudah diblokir.
"Hmm..'Pada masanya, kegelapan akan menyelimuti hati, Kebaikan dan Persamaan selama tiga zaman akan terhapus dalam beberapa tahun' berarti sebentar lagi dunia akan mengalami masa kekacauan' kata Wong ketakutan.
"TIDAAAAAK!" teriak Randolf, semua orang melihatnya dengan simpati. Kemudian ia berbisik kepada Wong, "tolong jempol kaki gw jangan diinjek. Ramalan sih ramalan, tapi jangan heboh gitu dong. Lebay nih...".
Sementara ribut-ribut masalah jempol, tiba-tiba Satron datang ke ruang sidang (hening sebentar), lalu Satron naik ke atas mimbar lalu berkata: "Ada apa ya ribut-ribut?".
"Hehehehehe, biasalah Tron, Anak muda" kata Randolf sok akrab, sok ayik dan sok anak muda ke Satron.
"Eee..buset" kata Satron sambil nimpuk Randolf pake piano yang entah kenapa tiba-tiba ada di situ. "Kenapa ya ribut-ribut?" lanjut Satron repost. Semua masih tampak hening, menunduk dan tidak berani menatap mata Satron yang tiba-tiba berwarna merah.
"Hehehehehe, biasalah Tron, anak gaul," kata Direzion (hampir) repost. Satron kemudian menimpuk Direzion dengan Bajaj Pulsar warna biru metalik yang tiba-tiba muncul, maklum lah Satron kan sakti. "Ada apa ribut-ribut?" katanya, repost lagi.
Prastudil lalu mendekati Satron dan berbisik "was wes wos Ar-Mour was wes wos ramalan was wes wos".
"Ah gak ada urusan saya sama lembur-lemburan!" bentak Satron, rupanya dia bolot sehingga menyangka Prastudil berbicar tentang lembur. "Terserah kalian mau ngeributin apa, saya gak ada urusan!" Satron pun ngeloyor pergi meninggalkan mereka.
Lega, itulah yang terpancar dari muka para hadirin, hal itu karena mereka menghadirkan Mithiel di sidang tanpa sepengetahuan Satron.
Mereka pun kembali mendiskusikan mengenai arti dari ramalan itu, Wong bergumam "Ketika Putri hilang, Dia Yang Menebus Dosa akan sampai pada senjanya. Di balik mawar yang indah, tertidur sebuah bangkai yang keji...", "Kira-kira apa ya maksudnya?".
"Putri Amelani! Tidak salah lagi!" Seru Prastudil mantabs.
"Dia yg Menebus Dosa, apakah itu Yang Mulia Satron?" tanya Anjariel "Jika ya berarti negeri kita dalam bahaya, karena dikatakan ia akan sampai pada senjanya, alias akan meninggal" lanjutnya.
Tiba-tiba seekor burung kakak tua hinggap di jendela, giginya tinggal dua. Dengan nafas terengah-engah ia berkata: "Ga ada angin, ga ada hujan, ga ada ojek, GAWAT JEK!! Satron pingsan".
Ternyata Satron cuma kepeleset waktu mau masuk ke istananya, dikarenakan genteng di istana bocor gara-gara disambitin sama anak2 tetangga. Setelah dibopong ke ranjangnya Satron pun siuman. "Fiuhhhh, bikin kaget aja" serempak semua orang lega.
Akhirnya setelah menyelimuti Satron yang masih letoy karena ketiban genteng, mereka kembali ke ruang sidang untuk membahas kembali arti ramalan yang diberikan oleh Mithiel.
"Jadi gimana, Pras?", tanya Randolf kepada Prastudil. Prastudil kembali melanjutkan analisisnya, "Sebentar-sebentar, sepertinya kita melupakan sesuatu..Ada satu faktor yang terlupakan..".
Tapi mereka yang terbuang akan menuntut balas... Prastudil semakin keras berpikir apa artinya itu.
''Siapakah yang terbuang? Menuntut balas atas apa? Apakah benar Satron yang menuntut balas akan kematian Amelani? Atau sebuah ras yang dulunya bersekutu dengan Ar-Mour, dan dikhianati olehnya? Atau... ayahku, Prastudon? '.
"Gag ada hubungannya sama Prastudon kaleee" kata Randolf pada Prastudil yang tiba-tiba jadi ngaco.
Randolf menjelaskan sejarah nenek moyangnya. Ayahnya, Randalf "The blackest" son of Randuin, pernah memimpin bangsa Chreibouw masuk pertempuran untuk membalas kematian Kakek Randuin kepada Ar-Mour.
Tiba-tiba atap kaca Ruang Sidang Gedung Dewan Tetua pecah, dan dari atas muncullah Ray Shaiton dan Mirfan! mereka langsung menyerang ke arah Prastudil. "Mati kau, pengkhianat!" teriak mereka serempak.
Brebet! Gerakan Ray Shaiton sangat cepat dalam sekali kedipan langsung mengunci Prastudil. "Tak ada tempat untuk pengkhianat!", seru Mirfan sambil mengayunkan kapak 212 ke arah bahu Prastudil.
Crot! Termuncrat darah merah dari terputusnya tangan kanan Prastudil.
"Ahhh", erang Prastudil kesakitan. "Habisi dia, Mirfan!", ujar Ray Shaiton sembil mempertahankan kunciannya. Seketika Mirfan mengayunkan kembali kapak 212-nya, tiba-tiba ada seberkas cahaya membutakan mereka berdua.
Anjariel melepaskan sinar laser dari telunjuknya, beruntung Mirfan dengan sigap menghadang dengan kapa 212-nya. Klontang kapak 212 Mirfan terpental ke arah tengah-tengah para Dewan Tetua, yg membuat mereka panik berhamburan.
*Cleeb!* kapak itu nancep di kepala Randolf. Prastudil langsung inisiatif menolong Randolf untuk mencabutnya *Crot!!* tapi Randolf teriak "Setan alas!!" kesakitan membuat Prastudil merasa bersalah dan *Clep!!* menancapkannya lagi.
Satron merasa terusik dengan keributan yang terjadi, lalu bergegas ke ruang sidang. "Siapa kalian, berani mengacau di sini?Bukan kah kalian pendekar dari Fasilnor?Tidak puas setelah membunuh Amelani putriku" tanya Satron sesampainya di ruang sidang.
Ray Shaiton pun langsung mengeluarkan pedang lasernya *Bleng!!!*, kemudian mengayunkan ke arah Satron dengan kecepatan tinggi berkat jurus langkah Shaiton-nya.
Satron tak mau kalah, dia juga mengeluarkan pedang lasernya,"swwwiing" suara pedang itu terdengar ke segala penjuru. Mereka kini bersiap bertarung layaknya pertarungan Star Wars menggunakan lightsaber.
"Sialan lo, bapak gak punya anak!" teriak Ray menghina. Lalu Satron hanya terdiam membatu, membisu, tiada berucap lagi. Ruangan menjadi hening hingga desak tangis Satron mulai terdengar jelas, air matanya perlahan membasahi wajahnya yang kayak setan.
Dan langit pun menjadi semakin gelap, karena Fisiphenia terkait dengan hati Satron. Para hadirin pun sadar, apa yang terjadi di Fisiphenia karena Satron sedang sedih sepeninggal Amelani.
Semua pun merasa simpati melihat keadaan Satron yg terpuruk itu. *Duerrrr!!!* tiba-tiba pintu gedung Dewan meledak dan sepasukan klan Shaiton dari batalyon Tjetjunguk masuk ke dalam gedung dewan. The battle begins.
Sungguh, sebenarnya Satron terlalu kuat bagi mereka. Satron menampar Ray tepak! hingga terpental jauh. Begitu juga kawanan shaiton yang lain dia buat tepok! pak! pok! satu per satu hilang sekejab.
Kini yang tersisa dari gerombolan Satron adalah The Margonaut, salah satu dari The Five Sages of Fasilnor, Guru Besar Ilmu Hitam Mikromagic.
***

Battle I:Satron berhadapan dengan Margonaut The Ear One.
Dengan kekuatan Signal Processor-nya Margonaut membangkitkan pasukannya untuk menyerang fellowship dan prajurit isatana Fisiphenia. Di tengah pergulatan sengit antara dua kubu Satron dan Margonaut saling berpandangan dengan nafsu membunuh di keduanya.
"Anak-anak, kalian main di luar dulu ya.." ucap Satron pada fellowship. "Iyaa.. kalian main berantem-beranteman ama fellowship di luar aja ya" ucap Margonaut pada para klan shaiton. Anak buah pun berantem di luar, bos-bos berantem di ruang sidang.
Satron memulai langkah awal, diayunkanlah pedangnya membabat perut Margonaut. Zwiing..nyaris mengenai perut Margonaut kalau dia tidak menghindar. Margonaut pun membalas.
Dia pun mengeluarkan senjata andalannya Golok Ping-Pong golok sakti yg bisa membalas semua serangan lawan. *Duar!!!* goloknya menghantam pilar gedung tempat dimana Satron berada sebelum ia menghindar bersalto ke udara.
Tapi sayang, Satron kehilangan ujung jempol kirinya. Dia terjerembab. *suasana menjadi hening* Lalu Margonaut memanfaatkan kesempatan ini, ia berlari ke arah Satron melancarkan serangannya.
Satron pun lompat dan beratraksi di udara, mengerahkan segenap daya tempurnya dan mengumpulkan kekuatannya, lalu menghujam ke arah Margonaut, brebet, tapi jduk Margonaut terkena bongkahan pilar lebih dulu. "Sial" kata Margonaut.
Sebelum dia sempat menyiapkan kuda-kudanya, pedang Satron suad terlebih dahulu melukai wajahnya *Croott* darah segar mengalir dari wajah Margonaut.
Tak terima dengan luka di wajahnya, Margonaut membalas serangan Satron membabi buta, ganas, liar, tak terarah yang membuat Satron terpaksa mengeluarkan jurus pertahanan miliknya.
Jurus itu adalah jurus andalannya: Biji-biji intan! Lalu seribu biji pun melesat cepat menembaki Margonaut hingga kalang kabut.
Margonaut pun membalas dengan mengeluarkan jurus Signal Disturbance-nya yg membuat seribu biji itu jadi tak terkendali dan malah balik menyerang Satron. Luar biasa memang Margonaut. Kekuatan Five Sages memang bukan isapan jempol.
Tapi dari antara biji yang melesat balik, ternyata bijinya Margonaut juga ikutan. Akhirnya Satron mengeluarkan pedangnya dan membelah biji itu menjadi 2. "Wadaaaaw!!" Margonaut teriak kesakitan.
Ternyata biji itu adalah salah satu jimat dimana 25% kekuatannya tersimpan. Alhasil sekarang Life Power Margonaut berkurang 25%.
Biji-biji itu mengenai Satron, satron pun sekarat karena kelemahannya ada pada bijinya sendiri. Margonaut pun kabur keluar dan dikejar dengan susah payah oleh Satron. Satron kembali mengeluarkan jurus pamungkas, jeger, ke arah luar ruang sidang..
Jurusnya itu diarahkan kepada Margonaut Jeger! Margonaut pun sempat terjatuh dari pelariannya. Ternyata di pojokan masih ada Mirfan dan Ray, langsung saja Jeger! Jeger!' keduanya tewas seketika, tak akan hidup lagi.
Satron pun makin kalap dia mengejar Margonaut di seluruh penjuru kota. Dan dalam pengejarannya dia melepaskan sinar laser ke segala arah, membuat kota hancur dan porak-poranda di mana-mana. Matanya nanar seperti ada setan yg merasukinya.
Margonaut yang sekarat menghilang, kembali ke Fasilnor tempat Ar-Mour bertahta, di sana ada 4 Sages lainnya sedang berpesta. Mereka adalah Bobats, Chanium, Ibamiel dan Jhomon, para guru besar yang dikenal dengan The Five Sages of Fasilnor.
swii..iing Gdebug!! Margonaut tiada berdaya lagi untuk pendaratan sempurna. Ia menjatuhkan diri dari langit, di tengah pesta yang meriah. Margonaut adalah guru besar, sehingga semuanya terkejut. "Aww! Terkejut my heart!" seru pak Ibamiel.
Sementara Satron yg mulai kehabisan tenaga tertatih-tatih berjalan di gurun di batas luar kota. Dia pun jatuh terjermbab kehabisan tenaga. Namun amarahnya semakin menggelora, "Margonauuuuuut!!" teriaknya di tengah gurun.
Teriakannya menggetarkan seluruh Fasilnor dan Fisiphenia, termasuk mengusik The Five Devils yang menjadi seteru abadi The Five Sages, moment penyerangan Margonaut ini dimanfaatkan untuk bersekutu dengan Satron melawan The Five Sages dan Fasilnor.
The Five Devils adalah 5 jiwa jahat yang diam di dalam tanah kegelapan. Dia akan muncul ke permukaan melalui tubuh makhluk lainnya, yang menyerahkan dirinya kepada kekuatan mereka.
Machdion Iblis Laut Barat, tertawa gembira melihat kejadian ini. "Kita akan punya sekutu baru, kawan-kawan" serunya kepada seluruh penghuni Laut Barat.
"Hanya saja, kita butuh tubuh untuk ditempati. Kira-kira, Satron dan 4 fellowships mau gak ya?" Ucap 4 Iblis (devils) lainnya.
Salah satu dari The Five Devils itu aslinya adalah seorang Peri. Peri Sulistianto nama aslinya. Namun sejak bergabung dengan The Five Devils, nama panggungnya adalah Mozzlion Pickon.
Satu lagi adalah Helmer Genderuwo dari Gurun Timur, wujudnya berbentuk manusia berkepala domba, lebih tepatnya domba garut asli.
Ada juga Hipasdon, Iblis Pantai Selatan. Konon katanya sempat ada affair dengan ratu pantai selatan.
Dan satu lagi adalah the chief of Five Devils, yg terkuat dari semua Five Devils. Berdiam di dalam jurang di tengah-tengah dunia. Yg Terkuat dari semua Five Devils,kesaktiannya bahkan bersaing ketat dengan Aria the Eldest saat dulu ia belum introvert.
Beliau adalah yang terhormat Tuan muda Fuadon Rosmon Hidayaton McFaddenon, atau biasa disebut Iblis Fu.
Iblis Fu ini adalah penguasa Underworld. Dialah yg tiba-tiba membuat lubang misterius dahulu kala di tengah2 perang balgebunen yg menyedot masuk banyak ksatria yg bertarung.
***

The Return of the Five Devils.
Iblis Fu berencana mengadakan pertemuan dengan semua devils. Oleh karena itu, dia mengetik perintah: "NET SEND * KUMPUL COY!". Dan terdengarlah rentetan bunyi "teet" itu, disertai gerutuan, "Maba bego, pake net send bintang lagi nih!".
*Duar!!!* terdengar ledakan dari arah setiap orang yg menggerutu. "Apa Aku tanya pendapatmu?" tanya Iblis Fu dengan mata nanar ke setiap orang yg tadi menggerutu.
Mereka berembug, untuk negosiasi kontrak dengan Satron dan para Fellowships. Setelah rencana matang, akhirnya Helmer merasuki seekor kelinci dan berangkat ke Fisiphenia, dimana Satron dan Fellowship berada.
Kelinci yang sudah dirasuki Helmer menghadapi banyak godaan dari ayam kampus yang berseliweran di Fisiphenia. Namun Helmer tetap teguh melanjutkan perjalanan mencari Satron dan Fellowship.
Tentu saja itu karena Helmer adalah penyuka sesama jenis, makanya ia lebih dikenal dengan sebutan Iblis Bencong dari Gurun Timur.
Setibanya di gedung dewan tetua, Helmer dikejutkan dengan segerombolan Tikus2 raksasa yang sedang kelaparan di parkiran gedung. "Ciiitt citt ciiiiittt (aseekk! MAKAN MAKAN!!)", kata tikus2 tsb serempak ketika melihat Helmer.
Tepat sebelum tikus-tikus raksasa itu menyergap kelinci, Helmer pindah dan merasuki biji salak yang tergeletak di sana. Dan menggelindinglah dia sampai ke depan kamar Satron.
Tiba-tiba biji salak itu terinjak oleh pelayan Satron yang baru keluar dari kamar. "Aduuuh kepreet" kata biji salak yang tidak lain tidak bukan adalah Helmer.
Pelayan Satron itu heran, Perasaan tadi udah bersih deh kamarnya Tuan Satron, tapi kok ada biji salak ya? Tidak mau ambil pusing, biji salaknya dibuang ke tong sampah. Sampah-sampah akan segera dibuang ke dekat gurun, sekitar batas luar kota.
Helmer ingin kabur ke tubuh pelayan ini. Tapi sayangnya, Iblis tidak akan mampu memasuki tubuh manusia tanpa mereka menginginkannya sendiri.
Tidak lama kemudian, petugas pengangkut sampah memasukkan isi tong sampah berisi biji salak tersebut ke truk sampah. Truk kemudian dibawa melintasi gurun. Ke arah tempat dimana Satron sedang merenung, setelah kegagalannya menangkap Margonaut...
Supir Truk sampah ini masih newbie, tapi sok-sok ngebut secepat kentut. Tiba-tiba Truk tersandung batu dan "gubrak!" tuuh kaan jatuh, Truknya jadi terbalik-balik di Gurun.
Dan biji salak yg diisi Helmer pun menggelinding ke arah Satron. Satron yang sedang di tengah gurun sendirian dan kesepian pun kaget melihat ada sebuah biji salak di sebelahnya. Karena perutnya keroncongan, akhirnya dia pun memakannya.
Saat mengunyah biji tersebut Satron bergumam, "Hmm, biji yg aneh, rasanya tidak seperti biji". "Teksturnya lembut, sedikit sepet-sepet asoy, tapi merekah di dalam mulut". "Pokoknya maknyus pemirsa", lanjut Satron. Tiba-tiba Satron merasa pusing..
.. hingga Satron pun tertidur. Di dalam tidurnya, Helmer menemui Satron dan berbincang-bincang menawarkan sebuah kontrak persetubuhan. Emm.. Maksud penulis adalah kontrak persekutuan dan perizinan untuk memasuki tubuh mereka sebagai media persekutuan.
"Aku bisa membantu engkau mengalahkan Margonaut dan The Five Sages, namun aku perlu mengendalikan tubuhmu dan kekuatan-kekuatanmu", tawar Helmer kepada Satron.
"Selain itu aku akan memberikanmu kekuatan untuk membalaskan dendam mu pada Fasilnor, yang telah berjasa membunuh Putri tercintamu. Bagaimana, Satron?", Tambah Helmer. Satron pun terdiam sejenak, namun bara api dendam mulai memercik dalam hatinya.
Beruntung Satron masih bisa mengendalikan emosinya yang hampir saja mengambil alih akal sehatnya. "Bagaimana aku bisa yakin kau tidak akan mengkhianatiku setelah kita berhasil mengalahkan Margonaut?", tanya Satron penuh selidik.
"Aku akan memberikan kekuatan dan senjata rahasia yang dapat membuatmu menghancurkan Fasilnor, dan kau bisa menjadi Raja di muka Wimor ini", bujuk Helmer. Satron pun kembali memikirkan tawaran menggiurkan itu.
"Dan yang lebih penting kau akan lebih berkuasa dari siapapun, termasuk Aria the Eldest" lanjut Helmer *Jgerrr!!* mendengar nama itu hati Satron kembali berguncang, pikirannya terlambung ke masa lalu, masa2 dimana ia adalah sebulir biji hina.
Apalagi Satron adalah pembawa sifat Raja dari Aria The Eldest, sehingga keinginannya untuk menjadi Raja Sejagat Wimor sudah sejak dulu diimpikannya. Kini hati dan niatnya tampak sudah teguh, dan siap menjawab ajakan dari Helmer.
"Baiklah aku siap bekerja sama dengan kalian" jawab Satron yg telah dibutakan kekuasaan. "Kapan kita bisa mulai?" tanyanya "Sabar Satron, aku hanyalah pembawa pesan, untuk proses selanjutnya kau harus menghadap Iblis Fu" jawab Helmer.
Lalu Satron mengajak Fellowship untuk menjadi media Five Devils beraksi. Tadinya, mereka gak mau. Tapi dengan iming-iming permen, akhirnya mereka luluh juga. Satron dan Fellowship berangkat menuju..
Sebenarnya para Fellowship tidak mengetahui apa tujuan kepergian mereka dengan Satron. Mereka hanya luluh karna permen yang dikasih dan karena mereka adalah abdi dari Satron. Akhirnya mereka pun mengikuti Satron menuju tempat Iblis Fu.
Karena singgasana iblis Fu ada di dalam tanah. Mereka pun naik subway ke sana, berhubung praktis dan harga tiketnya terjangkau. Randolf yg untuk pertama kalinya akan pergi jauh sebelumnya berpamitan ke tetangganya sekaligus mengadakan selametan.
Di tengah perjalanan saat ingin membeli tiket busway ke singgasana Iblis Fu, tiba-tiba Fellowship tersadar bahwa..
.. mereka sudah berjalan ke arah yang salah. Mereka mengetahui bahwa Iblis Fu adalah iblis yg sangat liar. Dialah penyebab bencana alam akhir-akhir ini di Wimor. Tetapi apa daya, mereka harus mengikuti perintah dari tuan mereka, Satron.
Lalu tibalah mereka semua di pintu gerbang Singgasana Iblis. Entah kenapa, denyut nadi terasa deras dan jantung semakin berdegap-degup.
Singgasana Iblis sungguh menyeramkan. Di samping pintu gerbang terdapat tulang belulang tergantung tak berdaya, seperti habis disiksa. Di atas pintu gerbang terdapat berbagai macam tengkorak kepala binatang. "Hiss sungguh menyeramkan", batin mereka.
Penghuni Singgasan Iblis pun tidak kalah seramnya. Terlihat di sebelah kiri sepasukan Goblin dan Orc berkulit merah sedang berbaris rapi. Di atas mereka berterbangan sejumlah Gargoyle yang cukup besar.
Yang lebih menyeramkan lagi, Satron dan para Fellowship disambut oleh bidadari bencong tanpa busana menuju jantung Singgasana.
Randolf hanya bisa menelan ludah melihat para bidadari penyambut mereka. "Wah gede-gede yah!" seru Randolf ketika matanya tertuju kepada...
.... deretan bangunan di kanan kirinya. Bentuknya tidak karuan, dengan sisi tajam seperti duri. Aura kegelapan sangat terasa di sini. Tapi anehnya Randolf merasa nyaman, mungkin karena berbau gelap-gelap itu.
Rasa nyaman itu tiba-tiba hilang ketika melihat sosok-sosok yang mendekat. Tapi tidak, tidak mungkin! Mereka telah mati dalam perang Balgebunen! Prastudon, dan...Randalf?
Mengetahui sosok yang dia dekati adalah Randolf anaknya, Randalf pun berusaha melepaskan diri dari dekapan Prastudon. "Malu ah, Bang!" lirih Randalf pelan kepada Prastudon. Sementara Randalf masih seperti tidak percaya dengan apa yang dilihatnya.
"Ayah...? Aku pikir ayah dibunuh dalam Perang Balgebunen? Dulu ibu yang cerita lho..." , tanya Randolf kepada randalf. Di saat yang sama, Prastudil (luka pada tangan kanannya sudah disembuhkan) menanyakan hal yang sama kepada Prastudon.
"Ayah belum mati nak, Ayah memang ditelan lubang hitam sewaktu Perang Balgebunen. Sejak saat itu, Ayah hidup di Underworld, tanpa memakai underwear, tapi kamu jangan underestimate, understand?" Randalf berusaha menjelaskan keadaannya yang nista itu.
"Anakku", ujar Randalf." "Ayah diperbudak iblis Fu karena kalah maen pingpong"."Tapi apa daya, terpaksa kulakukan demi sesuap biji".
"Sedangkan aku kalah dalam bermain WE.", jelas Prastudon kepada Prastudil, "Dan selain itu, kebetulan Iblis Fu adalah ketua EAL, jadi aku harus mengikuti segala perintahnya. Ngoding, ikut Futorial, bikin runes. Selama bertahun-tahun terakhir ini".
"Sungguh kejam iblis itu.. beraninya dia melakukan hal terkutuk itu kepada ayahku yang sudah cukup terkutuk.." sungut randolf.
"Tenang anakku, selalu ada hikmah di balik sengsara. Ayah jadi bisa kenal lebih dekat dengan Bang Prastudon, ternyata dia lelaki yang baik", Randalf coba menenangkan anaknya.
"..dia bisa jadi ibumu yang baik nak...", sambung Randalf.
Ah, tolol! (tm) pikir Prastudil, mengapa ayahku tidak mengadu bermain catur saja melawan iblis Fu? Pasti terbebas deh!.
"Heh, sudah, sudah! Reuni keluarganya nanti aja!" teriak Iblis Fu kesal.
"Itu yang kribo iket aja, ngobrol mulu, bikin lama", lanjut Iblis Fu. Rupanya di Underworld Iblis Fu bisa mengetahui semua kejadian dari jauh. "Ayo jalan Tuan Iblis Fu sudah menunggu", kata salah satu bidadari bencong.
Satron, serta Fellowship \& keluarga berjalan menuju ruangan pribadi milik Iblis Fu, yang dinamakannya My Incineration Center, atau singkatnya MIC. Pusat untuk membakar sampah yang terproduksi banyak sekali akhir-akhir ini...
"Gak puasa lu ya?" celetuk Randolf pada seseorang yang dilihatnya sedang minum dari sebuah cawan hitam. Sosok itu kemudian berbalik, dengan tatapan matanya yang tajam. Dialah Iblis Fu.
"APA AKU TANYA PENDAPATMU!??" Bentaknya menggelegar dan membuat singgasana bergoncang, menandakan kekuasaan dan kedahsyatan Iblis Fu ditakuti di singgasana.
"Ampun Oom... Ampun Oom...", Randolf berujar seraya ketakutan. Meskipun kata-katanya malah terdengar seperti bocah yang hendak dicabuli seorang paedofil.
Satron, tamu yang paling dihormati saat itu, berusaha menenangkan Iblis Fu dengan berucap: "Alangkah baiknya kita langsung berbincang membahas persekutuan kita yang hebat ini".
"Baiklah..", nada bicara Iblis Fu mulai menurun. "Ada apa gerangan kiranya Ki Sanak jauh-jauh datang kemari?" lanjut Iblis Fu.
"Aku dan keempat orang ini," Satron menunjuk Wong Cen Lau, Randolf, Prastudil, dan Direzon, "bersedia untuk meminjamkan tubuh kami sebagai medium bagi the Five Devils, dengan syarat kami nantinya akan menguasai Wimor, sekaligus membalas dendamku!".
"Dan kami ingin ayah kami dibebaskan", ujar Randolf dan Prastudon. "Heh berani sekali kalian mengajukan permintaan seperti ini padaku, si Raja Iblis", Teriak Iblis Fu. "Untuk menguji loyalitas dan niat kalian, kalian harus dites terlebih dahulu".
"Gag bisa kurang lagi tuh Gan?" ceplos Mozzlion Pickon menyela Iblis Fu.
"Diam kau, Peri!" bentak Iblis Fu. "Baiklah, kalau begitu, kalian berlima akan diuji oleh dua suhu kami, A Lin dan A Num. Bila kalian lulus semua, tawaran sebelumnya saya terima. Tapi bila tidak... kalian akan merasakan akibatnya", Iblis Fu mengancam.
Mendengar nama suhu A Lin dan suhu A Num yang legendaris itu, Prastudil menjadi sumringah. Semua mata menjadi terkesima melihat wajahnya menjadi secerah embun pagi hari.
Singkat cerita supaya petualangan kita tidak panjang-panjang amat tapi bosan, Prastudil seorang diri berhasil melewati tes A Lin dan A Num dengan gemilang. Masing-masing lulus dengan nilai A+ dan predikat cum laude.
Satron, Wong Cen Lau, Randolf, dan Direzion pun merasakan akibatnya, yaitu...
Mendapat nilai I dikarenaka mereka tidak boleh mengikuti ujian akhir tes tersebut akibat absensi mereka kurang dari 75%. Mereka pun harus mengikuti tes yang lain sedangkan Prastudil dipersilahkan untuk bersenang-senang dahulu sambil menunggu yg lain.
Mereka harus mengikuti tes A Dbis dengan instrukturnya adalah tokoh yang sangat terkenal di dunia bisnis dangdut, Bang Toyib. Tapi karena sudah 3 kali puasa dan 3 kali lebaran Bang Toyib tidak pulang, maka tesnya pun diganti dengan...
Sebuah tes yang lebih berat lagi bersama Suhu A Num. Namun, kali ini mereka meminta nasihat-nasihat dari Prastudil sebelum tes dimulai, dan akhirnya mereka semua lulus! Iblis Fu merasa puas. "Baiklah, tawaran kalian saya terima", ujarnya.
"Tapi Tuan, masa kita memberikan dunia Wimor untuk mereka kuasai nanti?" sergah Hipasdon. Iblis Fu hanya mengedipkan sebelah matanya kepada Hipasdon untuk memberi tanda. Tapi sayang bidadari bencong di belakang Hipasdon melihatnya, dan menjadi Ge-eR.
Lalu Iblis Fu berkata pada Satron: "Satron, saya akan bergabung dengan kamu. Dan kita akan menjadi Iblis Futron". Iblis Hipasdon join dengan Prastudil menjadi Iblis Hipasdil, Iblis Machdon join dengan Randalf menjadi Iblis Machdalf.
"Tunggu dulu!", teriak Helmer, "Satron harusnya bergabung denganku! Kami sudah sepakat!" Iblis Fu menyeringai, "Boss dapet pilihan pertama dong. Udah, gabung sama Direzion sana, jadi Iblis Helmion".
Iblis Mozzlion Pickon jadi kebingungan: "Saya sama siapa, dong?" tanyanya. Karena dilihatnya fellowship yang tersisa adalah Wong, maka Mozzlion Pickon join dengan Wong menjadi Iblis Wong Pickon.
"Hahah, lengkap sudah pasukan kita, skarang hanya butuh Mithiel dan ramalannya untuk mendapatkan senjata rahasia kita", Kata Iblis Futron. Para iblis pun berpisah Machdalf dan Helmion menjemput Mithiel, sisanya menuju Fasilnor untuk berperang.
***

Final Battle: The Five Devils vs. The Five Sages.
Sementara itu di daratan Fasilnor, The Five Sages sedang duduk ngopi.
"There's something wrong!" kata Chanium kepada Jhomon. Entah apa yang dimaksud Chanium saat itu.
Chanium merasakan 3 kekuatan besar sedang bergerak menuju Fasilnor.
"Apa yg kau rasakan Chanium?" tanyanya sambil ngopi, "Kelihatannya kau gusar sekali?" kali ini sambil menenggak habis kopinya.
"3 kekuatan jahat yang sangat besar!" ucap Chanium sembari terlompat dari tempat duduknya, membuat isi gelas kopinya tertumpah. Para Sages terkaget dibuatnya, kemudian berusaha merasakan datangnya kekuatan tersebut.
"Tapi tunggu dulu!" tiba-tiba Bobats menyela sambil kembali menuang kopi ke gelasnya. "Ada apa Bats?" ketus Jhomon yg tersedak akibat dikagetkan Bobats. "Ada 2 kekuatan lagi yg sedang mengarah ke sekolah Azzura Rosso!" jawab Bobats. "Hah?" ....
"Itu pasti Satron, tapi tak mungkin ia memiliki hawa iblis sedahsyat ini, apakah mungkin ia..", gumam Margonaut. "Lebih baik kita segera mensummon pasukan kita dan berangkat ke 2 tempat tersebut. Aku merasa ada hal yang tak beres di balik semua ini.".
"Margonaut, bagaimana kondisi Mithiel? Apa kau berhasil mencegah Satron mendengar ramalan itu" tanya Chanium. "Sial! aku lupa" jawab Margonaut. Sementara itu Anjariel sedang pusing bagaimana caranya mengembalikan Mithiel kembali ke sekolahnya.
Mithiel ngambek which is dia males banget kembali ke sekolah.
Tanpa basabasi, Anjariel langsung memasukkan Mithiel ke dalam karung lalu menaikkannya ke punggung Tikus2 raksasa kemudian menyuruh mereka berangkat ke sekolah. Sementara itu, Machdalf dan Helmion menunggu di sekolah sampai jamuran.
"Akhirnya datang juga!", ujar Machdalf. Anjariel terperanjat melihat kedua sosok mengerikan tersebut dan tiba-tiba ::Brebet!!:: Dalam sekejap, Anjariel telah terjatuh dari tikusnya dan Mithiel telah dibawa kabur oleh kedua iblis.
"Piwwiit" Anjariel pun bersiul memanggil kereta kudanya yg mampu berlari melampaui kecepatan kentut. Segera ia pun melaju mengejar kedua iblis itu. Sementara Mithiel yg di dalam karung tidak menyadari kalau ia sudah berpindahtangan.
Namun memang secepat2nya kuda Anjariel berlari, tetap masih kalah cepat dibandingkan dengan kecepatan para iblis. Tidak butuh waktu lama agar mithiel dan kedua iblis lenyap dari pandangan Anjariel.
Mithiel gusar dalam gendongan Helmion yang tampan, perpaduan wajah belanda dan kelembutan orang garut. Mithiel semakin kecut, tapi matanya berbinar-binar.
Helmion dan Machdalf pun pergi secepat kilat menuju Gua Babe, dimana senjata rahasia pemusnah massal yang konon katanya terucap dalam ramalan Fitriel disimpan. Mithiel adalah kunci untuk menggunakan senjata legendaris itu.
Mereka pun meraba-raba seluruh dinding pintu gua mencari dimana lubang kuncinya. Tapi sampai dinding gua itu halus karena diraba-raba melulu, mereka tidak juga menemukan lubang kuncinya. Helmion pun marah dan membentak Mithiel "Lubangnya dimana sih?".
"Mas nanya lubang yang mana nih?", balas Mithiel sambil tersipu malu. "Ya lubang kunci lahh, lubang yg lain mah kita bicarakan di lain waktu saja". Tapi Mithiel teringat pesan ibunya agar tidak pernah membuka kunci ke Gua Babe.
Sementara itu Five Sages memutuskan untuk berpencar, Jhomon dan Ibamiel ditugaskan untuk menyelamatkan Mithiel sedangkan yg lain bersiap menghadapi iblis Fu. Jhomon dan Ibamiel segera menuju ke sekolah Mithiel, namun mereka tak menemui apapun kecuali.
Anjariel yang terkulai lemas di tanah. "Sial, Mithiel dibawa pergi oleh setan-setan alas itu", kata Jhomon. "Oh tidak, jangan2 mereka ingin membuka senjata legendaris itu?", Seru Ibamiel. Mereka berdua pun bergegas menyusul ke Gua Babe.
"Eh, [Njar] boleh minjem kereta kuda elo gak?" tanya Jhomon, "Biar lebih cepet, lagian kita dah ccapek nih lari dari tadi" lanjutnya. "Oke deh, tapi balikin ya!" kata Anjariel sambil menyerahkan tali kekang kereta kudanya.
Sementara itu di Pintu depan Gua Babe, Helmion dan Machdalf berhasil memaksa Mithiel untuk membuka gerbangnya. "Cepet buka oy, dengan senjata yg di ada di dalam, dunia akan jadi milik kami para iblis. hahahhaha grok grok", tawa Machdalf.
Mithiel terpaksa membuka pintu Gua itu, karena dia sudah tidak tahan melihat tampang menjijikkan kedua setan itu. Dia pun mengetuk pintu, dan mengucapkan "Assalamu'alaykum", dan pintu Gua pun tiba-tiba terbuka. "Wakss, gitu doang caranya?".
Begitu pintu goa terbuka,goa itu bergemuruh, sebuah lorong gelap dan berbau anyir memanjang lurus ke dalam goa.
Mereka pun langsung menyusuri goa tersebut, sesekali mereka tergelincir, sebab lantai goa itu penuh dgn lumut. Namun kesigapan mereka agak kurang karena tiba-tiba Machdalf kehilangan keseimbangan dan menabrak Helmion yg kemudian menabrak pula Mithiel.
Tanpa sengaja Mithiel berteriak "E buset" dan seketika itu, ratusan obor menyala sepanjang goa dan di ujung terlihat sinar yang menyilaukan mata. "Astaga" kata Machdalf dan Hilmion bersamaan, dan seketika itu juga, obor padam. Obor yang aneh.
Nun kejauhan di sana mereka melihat seberkas cahaya yg menyilaukan. "Nah pasti itu dia!" seru Machdalf dan Helmion kompakan. Segera mereka berlari ke arah cahaya itu.
Lama sudah mereka berlari namun tak kunjung sampai. Akhirnya mereka berhenti, "Jangan-jangan pake password lagi?" kata Mahdalf pada Hilmion. "Password?" tanya Mithiel dan tiba-tiba cahaya itu mendekati mereka. Ternyata passwordnya adalah 'Password'.
Dan dalam sekejap mata, mereka bisa melihat bahwa ternyata kilatan cahaya berasal dari bilah sebuah pedang. Inilah Debian Sword yg kesohor itu.
Debian Sword adalah sebuah mahakarya dari salah seorang Five Sages of Fasilnor. Menurut legenda, Debian Sword di buat dari bahan-bahan langka, seperti ekor phoenix, tanduk unicorn, sayap pegasus yang di campur dengan logam Adamantium.
Debian Sword menurut legenda adalah pedang dimana setiap orang yang memilikinya akan mampu mempunyai kekuatan yg dapat menguasai dunia. Kekuatan yg sebanding dengan kekuatan Aria the Eldest saat dia belum menjadi introvert dulu.
Debian Sword adalah buatan Ibamiel, dan menurut legenda pula, hanya orang-orang beriman lah yang dapat mencabut pedang tersebut dari batu yang seolah-olah mencengkeram dengan kuat, ah sungguh dahsyad kekuatan pedang itu.
Helmion pun segera kalap melihat Debian Sword, dia langsung mencoba untuk mencabut pedang itu. Dia pun menarik kuat2 pedang itu sambil mengerang. "Errrrghhhh!!!" erangannya membahana di seluruh gua dan di luar gua. Membuat orang2 berpikir aneh2.
"Udah udah..abis abis..lo tu udah gagal..sini gantian gw" kata Machdalf sotoy yang dengan sigap menggenggam erat pedang itu. "uuuwwwoooh..auuuwoooo" teriaknya tapi pedang tak bergeming sedikitpun dari tempatnya.
"Tunggu dulu Kisanak!" tiba-tiba mereka dikagetkan oleh teriakan dari luar goa, kemudian mereka melihat sebuah kereta kuda dan dua penumpangnya masuk ke dalam gua, yg kemudian tergelincir yg membuat kereta kuda dan penumpangnya jatuh terguling2.
"Tunggu sebentar kisanak" kata salah seorang di antaranya sambil merapikan baju dan menolong temannya kemudian langsung berdiri tegak, sok sok jaim. "Jangan kau teruskan usahamu yang sia-sia itu sob" katanya meyakinkan Machdalf.
"Siapa kah kalian?" tanya Machdalf tanpa melepaskan pelukannya dari Debian Sword membuat orang2 agak mual melihatnya."Berani-beraninya kalian menyela kami? Hah? Hah? Hah?" lanjutnya lagi.
"Kami adalah..." kata dua orang itu serempak membentuk formasi aneh. "Ibamiel" kata salah seorang dari mereka, "dan aku Jhomon" kata lainnya. "Dengan kekuatan ilmu..akan menghukummu" lanjutnya serempak, tidak lupa dengan gaya ala Sailormoon.
"Banyak omong kalian!" sahut Helmion, Duar!!! dia pun mengayunkan kapak raksasanya, yg entah dari mana tiba2 bisa muncul ke arah Ibamiel dan Jhomon. Beruntung kedua orang itu mampu menghindar dengan bersalto ke belakang.
Jhomon pun membalas mereka dengan serangan paket datagram segede gaban yang membuat Machdalf terpental ke belakang dan langsung bersalto juga tak mau kalah namun akhirnya kejedot stalakmit gua. "Mampuy" kata Jhomon penuh dengan senyum kemenangan.
Kemudian Jhomon bersiap-siap menhajar Helmion, dia pun melakukan lompatan ke arah Helmion sambil berteriak "Yahoooook!". Duar!!! Helmion mampu menghindar sayang dia mendarat di tempat yg licin jadi dia pun terpelanting juga akhirnya.
Tak terima dirinya dipermalukan di depan Mithiel yang lagi asik nonton pertarungan sambil makan popcorn yang entah dapat darimana, Machdalf pun mengeluarkan jurus Strike Freedom "BLAAAR!!" yang mengenai kaki Jhomon karena terlambat menghindar.
"Eh gantian dong, gw blom nterang2 nih dari tadi" tiba2 Ibamiel menceletuk. Dan dengan jurus kunyuk-nya dia pun menyerang sekaligus Helmion dan Machdalf.
Jurus Kunyuk Ibamiel sangat hebat, sehebat Jurus Mabuknya Jet Lee. Helmion pun terkena cakaran di mukanya. Helmion marah, karena asetnya yang berharga di rusak, dan langsung mengayunkan kapaknya mengenai dahi Ibamiel sehingga dahinya menjadi hitam.
Ibamiel pun terdesak, hingga ke pojokan tempat Debian Sword berada. Dia pun segera mencabut pedang itu, merasa posisinya sudah semakin sulit. Dengan mudah dia mencabut pedang itu, "Kreekkkk!!!" begitulah bunyi saat Debian Sword dicabut.
Seketika semua terdiam,goa tersebut bergemuruh, sang pedang pusaka telah bangkit dari tidurnya. Ibamiel pun menebaskan pedangnya ke arah Helmion dan ditangkis dengan kapaknya, namun kapak itu patah karena Debian Sword terbuat dari adamantium.
"Gileee!!!, masih sakti aja nih pedang" kata Ibamiel sambil memandangi pedang buatannya ribuan tahun lalu itu. Saat Ibamiel tengah asyik mengagumi pedang buatannya itu yg sudah ribuan tahun tak ditemuinya, Machdalf mengambil kesempatan untuk mencuri.
"Eits gag kena donk ah" kata Ibamiel mengelak dan mengejek Machdalf.Namun kesempatan tak di sia siakan Helmion, saat Ibamiel tengah asik mengejek Machdalf, secepat kilat diambilnya pedang itu dari tangan Ibamiel dan menghilang di ikuti Machdalf.
Dan "BREBET!!!" tiba2 Machdalf mengeluarkan jurus yg mengeluarkan jaring yg segera mengikat Ibamiel, Jhomon, dan Mithiel. Tanpa ba-bi-bu lagi Helmion dan Machdalf pun kabur menggunakan kereta kuda Anjariel yg dipinjem Jhomon tadi.
Machdalf dan Helmion pun menghela kereta kudanya secepat kilat, menuju ke markas mereka untuk menyerahkan Debian Sword ke Futron. Saking cepatnya kereta mereka bergerak, tak sadar ternyata kudanya ketinggalan di belakang kereta. Kereta yang aneh.
Sementara itu Anjariel datang pergi menyusul ke Gua Babe. Terkejutlah dia mendapati Ibamiel dan Jhomon terikat lemas. Kemudian dia pun melepaskan ikatan kedua Sages tersebut. "Kita harus cepat ke Fasilnor. Fasilnor dalam bahaya!", seru Ibamiel.
"Naik apa kita kesana?kereta kuda gw ilang" tanya Anjariel gusar. "Tenang-tenang, kita naek ini" kata Ibamiel sambil menunjukkan sebuah kunci. "Apaan tu" kata yang lain serempak. "Kunci mobil kijang gw, tu gw parkir di bawah" kata Ibamiel santai.
***

Takorilien merupakan sebuah lembah perbatasan antara Fisiphenia dan Fasilnor bagian utara. Pasukan Futron yang terdiri dari mahluk-mahluk paling mengerikan sejagat Wimor terlihat dari kejauhan dan memenuhi horizon.
Hangatnya hawa jahat mereka berhembus sampai ke lembah Takorilien yang dingin, membuat para binatang penghuni lembah keluar dari sarangnya, karena udara yang biasanya sejuk menjadi panas.
Futron berdiri di garis paling depan memimpin pasukan monster itu. Dua Iblis lainnya yaitu Wong Pickon dan Hipasdil berada di sektor kanan dan kiri Futron memimpin pasukan masing-masing yang tak kalah mengerikan dengan pasukan Futron, iblis gitu loh.
Pemandangan di bukit itu sangat mengerikan bagi penjaga benteng perbatasan. Mereka pun segeara bersiap-siap mengerahkan pasukan. Kurir tercepat pun dikeluarkan untuk meminta bantuan kepada headquarters. Penduduk benteng pun mulai diliputi rasa cemas.
Suasana sangat mencekam kelam itu ditambah langit yang menjadi gelap dan awan mendung kejar-mengejar, halilintar sambar-menyambar, serigala melonglong, anjing menggonggong, kampred berseliweran.
Tiba-tiba asap mengepul dari arah selatan benteng perbatasan. Terlihat panji-panji lambang 5 Sages bermunculan. "Itu para five sagess, kita selamat", seru seorang penjaga benteng. Dari kejauhan terlihat Chanium, Bobats, dan Margonaut memimpin pasukan.
Kemehek-mehek!! Sebelum sempat penjaga benteng itu bergirang ria, Para Iblis langsung memasang kuda-kuda untuk menembakan bola api ke arah Margonaut, Duar! Duar! Duar!! .
Bola api membuat langit bagaikan diterangi seribu matahari. Langit pun berubah warnanya menjadi silver. Inilah SilverLight.
Serangan tersebut membuat kaget kubu Sages, hampir 1/3 bagian sayap kanan pasukan mereka lenyap oleh serangan mendadak tersebut. "Jangan panik, rapatkan saf rapatkan barisan. Seranggg!!", pimpin Chanium.
Namun terlambat Hipasdil memimpin pasukannya mengepung dari arah kanan luar pertahanan Fasilnor, yg membuat pasukan itu kini terkepung. Chanium pun kini terdesak.
Merasa terdesak, Chanium pun mengeluarkan jurus Eliminasi Gauss Jordan, "Blaarrr!!!" suara dentuman terdengar menghantam pasukan Hipasdil. Pasukan Hipasdil tercerai berai.
Melihat kondisi yg menguntungkan, Margonaut bersama klan Shaitonnya menyerbu langsung ke jantung pertahanan pasukan monster. Dimana Futron yang sedang mengupil tidak menyadari bahaya yg mengarah ke dia.
Brebet!! klan Shaiton berhasil mendekap mesra Futron yang tak bisa bergeming dan masih sempet-sempetnya mengupil.
Slepet!! dengan sekali tebasan Debian Sword, seluruh pasukan klan Shaiton yang mendekap Futron berubah menjadi abu. "Anjrid, curang abis!!" teriak Margonaut tak percaya dengan apa yang dia lihat.
Ternyata tanpa disadari oleh Sages, Machdalf dan Helmion telah kembali membawa Debian Sword. Melihat kejadian ini pasukan Sages menjadi gentar, seperti orang yg mau menyeberang jalan sudah di tengah jalan bingung mau maju atau mundur.
"Woi! yang balik badan ntar gw injek2 ampe tinggal 4 Byte!", teriak Jhomon yang makin memposisikan para pasukan Sages dalam posisi terjepit. Maju kena, mundur kena *kaya judul filmnya warkop*.
Dengan tambahan pasukan Machdalf dan Helmion, pasukan iblis pun bertambah liar dan brutal. Dalam sekali serangan, benteng perbatasan berhasil dikuasai. Kubu Sages tampak gentar, mereka terpaksa mundur 20 kilo ke arah benteng Gedung C.
Futron tertawa senang atas kemenangannya. The five sages mengatur strategi dan melihat buku kurikulum, buku yang menyimpan jurus-jurus rahasia mereka. Sementara dari kejauhan, Elf wanita mengamati setiap kejadian dengan seksama.
Buku kurikulum itu sudah bapuk, terlalu lama tidak diperbaharui lagi. Ketika dibaca kembali, rasanya ada yang ganjil dalam buku kurikulum tersebut. Komposisinya jurusnya amat membingungkan dan tidak dilengkapi dengan jurus-jurus terbaru.
Buku tersebut bahkan kalah up-to-date dibanding fotokopian di Pako.
Namun, ada satu jurus klasik yang belum pernah dibahas dan dipelajari. Jurus itu tidak diajarkan di Universitas A maupun Institut Teknologi B manapun. Jurus itu adalah..
Jurus Pedati!
Ke esokan harinya perang pun dilanjutkan. Pasukan Futron berteriak-berteriak dengan lantang, sementara pasukan Fasilnor membuat posisi siaga. Jauh dibelakang mereka The Five Sage sedang menyiapkan Jurus Pedati yang sudah terlupakan...
Gelombang serangan pertama dimulai dengan serangan anjing-anjing liar pasukan Hipasdil, meudian diikuti serangan udara dari korps Helmion yang terdiri dari gargoyles mengerikan. Dengan sekejap serangan pertama sudah membuat barisan depan kalang kabut.
Barisan kedua pasukan Fasilnor telah bersiap, kondisi bertahan. Sesuai instruksi, mereka tidak boleh menyerang. Persiapan selesai, Jurus Pedati membuat sebuah lobang diudara dan keluarlah Odin.
Namun sebelum Odin beraksi, Iblis Machdalf sudah mengirimkan serigala betina liarnya, Asena untuk beraksi duluan. Serigala betina itu mengobrak-abrik pertahanan barisan kedua.
Odin son of Acephun adalah summon terkuat yg dimiliki Fasilnor. Dengan sigap Odin pun menangkal gelombang serangan serigala dan gargoyles Iblis. Kini pasukan Fasilnor dapat bernafas untuk beberapa detik.
" Tunggu! " teriak Iblis Futron kepada para Iblis. Para Iblis pun menahan serangannya. "Itu adalah Odin, itu adalah jurus Pedati yang hilang dari kaum Iblis".
The Five Sage bangkit dari posisi summoning. Dengan gerakan mata Chanium separuh pasukan Fasilnor beserta ke empat Five Sage maju. Odin mengamuk menghancurkan barisan depan pasukan Iblis. Futron terkejut... perang semakin kisruh...
Futron pun tak mau kalah, dia dan keempat Iblis merapal jurus sakti mandraguna yang terlarang. Jurus Biji Iblis. Dengan bantuan Debian Sword, jurus ini sangatlah mengerikan. Srebet, dengan sekali rapalan, hilanglah seluruh korps Jhomon menjadi debu.
Iblis Futron (Iblis Fu dan Satron) ditambah Debian Sword, keadaan ini memang membuat pertarungan semakin tidak berimbang.
Tetapi pasukan Numerikal Chanium tidak dapat dipandang remeh. Dalam keadaan kacau balau ini, mereka dengan tepat waktu datang dan menyerbu sayap kanan pasukan Iblis. Chanium pun terus mendesak maju ke pertahanan Futron.
Sementara itu Futron menganti siasat. Dia memerintahkan para jendralnya melawan Odin. Namun pasukannya terus digempur pasukan Fasilnor. The Five Sage tidak tinggal diam. Futron dalam keadaan terdesak.
Kini pasukan gabungan Fasilnor sudah sampai ke barisan pasukan cadangan Futron. Sementara itu 4 Iblis bahu membahu menerjang Odin, Odin pun tertahan lajunya untuk sementara. Futron pun kesal, kemudian dia pun mengeluarkan jurus simpanannya.
Futron meringis kesal... tidak pernah terpikir jurus ini akan dia keluarkan. Kembali langit menjadi terang. Ya ini Silverlight, namun terlihat dari terangnya langit jurus ini jauh lebih kuat, jauh lebih hebat dari sebelumnya.
Holy Silverlight Sonata Arctica, jurus termaut dan terlarang yg pernah diciptakan oleh kaum iblis. Langit pun mendadak berwarna silver, dan tiba-tiba petir2 menggelegar ke arah pasukan Fasilnor. Diikuti munculnya sosok besar berjubah hitam.
Dari langit sosok ini melemparkan petirnya ke arah pasukan belakang Fasilnor. Tepat pada ledakan petir tersebut, langit kembali normal. Sosok besar itu kini terlihat jelas, dia dikenal sebagai Isrovil son of Zeusese. Pembawa petaka didunia.
Petir-petir tadi hampir menyisakan hanya 1/4 pasukan Fasilnor yang bertahan. Isrovil merupakan ahli listrik bekas kerja di PLN, sehingga ia bisa membuat serangan listrik yang sangat dahsyat. Kini pertempuran kembali seimbang.
Namun kehadiran Isrovil dapat menjadi kunci kekalahan pasukan Fasilnor. Chanium mengkaji situasi tersebut. Keberuntungan sepertinya berpihak. Para jendral iblis yang sudah letih melarikan diri dari pertarungan dengan Odin.
***

Pertarungan sangat sengit dan hampir mencakup sebagian besar pusat kerajaan Fasilnor. Hutan-hutan terbakar, bangunan hancur, warga sipil yang tak tahu menahu pun kehilangan nyawa mereka. Sungguh kejam aksi para Iblis di tanah Fasilnor.
Chainum memutuskan sudah saatnya menghabisi Futron. Odin mengganti targetnya, Isrovil. Mengendari kudanya yang mampu berjalan diudara, dia berpacu, bersiap menyerang Isrovil.
Isrovil dan Odin adalah musuh bebuyutan di dunia summon-summonan, darah Isrovil pun panas melihatnya, ia pun menerjang Odin di udara. BUOOMM!! Terjadilah ledakan keras menyusul benturan kedua mahluk summon tersebut.
Mahluk-mahluk summon tidak dapat bertempur di dunia kita. Mereka harus kembali ke dunia mereka. Tak ada yang tahu kisah mereka selanjutnya. Namun semua tahu bahwa The Five Sages langsung mengepung Futron dan melancarkan serangan-serangan mematikan.
4 Iblis lain mencoba mendobrak kepungan yang dilakukan oleh pasukan Fasilnor dari segala arah. Namun usaha mereka tak kunjung berbuah. Futron pun kini terkepung, tetapi semakin meliar dan membunuh semua yg ada di dekatnya.
Futron semakin kalap, dia menyerang membabi buta. Setiap gerakan akan mengirim satu orang ke akhir hidupnya. Tangannya tak lagi dapat dikontrol pikirannya. Gerakannya bagaikan tarian yang dibimbing Debian Sword. Bagaikan sebuah manual, tanpa cacat.
Korban sudah terlalu banyak berjatuhan di kedua belah pihak. Korps Jhomon dan korps Margonaut hampir sudah tak ada daya tempur, begitu sebaliknya dengan korps 4 Iblis. Futron pun kian mengganas menari dengan Debian Sword.
Tarian Futron nyaris tak kasat mata dan semakin cepat. Hanya The Five Sage yang dapat mengimbanginya. Para Jendral Iblis merasa ada yang salah dengan Futron dan ada yang salah dengan pertempuran ini.
Pertempuran yang sudah mulai tidak kelihatan ujungnya ini makin menggusarkan seluruh pasukan dari kedua belah pihak. Cukup banyak kerugian yg ditimbulkan. The fellowship pun makin gundah gulana hatinya.
Penggabungan The Fellowship dengan para iblis tidak dilakukan sepenuh hati. Tidak seperti Satron dan Iblis Fu. Kadang kala terjadi peperangan dalam diri Jendral-jendral Iblis ini, salah satunya adalah saat ini.
Kesadaran the Fellowship sepertinya muncul, pikiran-pikiran mereka tidak lagi dikuasai oleh para iblis, entah mengapa, semua terasa begitu janggal.
Namun pengaruh para Iblis semakin kuat, menyadari tubuh inangnya menolak, mereka pun semakin mencengkeram jiwa-jiwa para fellowship.
Fellowship makin merasa yakin kalau mereka di bodohi oleh para iblis. The Fellowship pun berusaha sekuat tenaga melawan cengekraman para Iblis, yang akhirnya mereka berhasil memperoleh kesadaran masing-masing. Para Iblis pun lepas dari tubuh mereka.
Kilatan cahaya yang silau meliputi proses terlepasnya para Iblis dari tubuh inangnya. Keempat fellowship pun jatuh pingsan selepas para iblis keluar dari tubuh mereka. Keempat iblis yg telah keluar itu.
Melayang-layang di arena pertempuran tanpa arah. Tanpa tubuh inangnya, para iblis hanyalah sekumpulan makhluk yang tidak berdaya. Mereka pun melayang-layang mendekati Futron yang sedang bertarung dengan Five Sages.
Fu yang merasakan bahwa Jendral Iblis telah kehilangan pengaruh mereka pada The Fellowship menarik mereka masuk ke dalam tubuh Satron. Futron pun semakin bertambah kuat.
Bersatunya kelima iblis ini dalam satu tubuh membuat tubuh inangnya mengeluarkan listrik, seperti Son Goku dengan super saiya 3 nya tapi bedanya Futron tidak menjadi gondrong. Ctar ctar, suara kilatan listrik di sekeliling tubuhnya sangat keras.
BLAR!!! seketika tubuh Futron bercahaya terang. Hawa Iblis tercium sangat pekat, membuat pasukan yg tidak kuat imannya tertunduk lemas. Aura Ultimate Devil memang sungguh dahsyat.
Kini Futron semakin liar lagi, satu sabetan pedang Debian Swordnya dapat menebas hingga 500 meter dan ratusan prajurit pun rubuh. Hentakan kakinya membuat tanah disekitarnya retak. Teriakkannya membuat orang dalam radius 5 kilo menjadi gila.
Ultimate Devil plus Satron dan Debian Sword, sungguh sebuah kombinasi yang sangat mengerikan. Futron berubah menjadi salah satu makhluk terkuat di alam semesta, selain Aria The Eldest, Ar-Mour dan Yulion.
Five Sages pun bahu membahu melawan Futron, mereka terus menerus menggempurnya dari semua arah. Tapi Futron begitu kuat hingga serangan Sages tidak berasa, hanya seperti sentilan saja.
Tidak ada prajurit ataupun jendral yg berani mendekati Futron. Namun tiba-tiba Anjariel muncul dari kerumunan pasukan, hanya dialah seorang yg berani, karena dia tidak dapat mati lagi. Namun saat sudah berhadapan dgn Futron, dia berhenti dan diam...
Dia pun mengingat kontrak nyawanya dengan Satron. "haha Anjariel, bergabunglah denganku. Kau adalah milikku", Futron berkata. Sementara itu Fellowship sudah sadarkan diri dan menyiapkan diri untuk berhadapan dengan Futron.
Fellowship bersiap-siap, masing-masing mengeluarkan senjata pamungkasnya, memasang kuda-kuda untuk bertempur dalam pertempuran hidup dan mati ini. Mereka terperanjat, melihat tanah di sekeliling mereka hancur, retak dan porak poranda.
Terlebih lagi mereka melihat Anjariel sahabat perjalanan mereka di kubu Futron. Mau tak mau mereka kini harus menghadapinya. Sungguh pedih memang perang ini, tidak ada lagi kawan maupun lawan. Sungguh keji.
Seseorang yang memang terus mengamati dari jauh sekarang telah berdiri. Dia tahu sesuatu yang penting akan segera terjadi. Sesuatu yang menentukan nasib Fasilnor, bahkan nasib Wilmor.
Dialah Aria the Eldest akhirnya dia memutuskan unutk meninggalkan dunia autis-nya dan berkontribusi untuk menyelamatkan nasib semesta.
"Aku adalah Aria the Eldest" kata Aria ketika muncul di tengah-tengah arena pertempuran. Turunnya Aria the Eldest dari singgasananya membuat semua orang terkejut. Baru pertama kali ini mereka melihat sosok sang Legenda, kecuali Satron bekas bijinya.
Namun ketika dia menyadari banyak sekali orang di padang Takorilien. Aria agak ragu, hasrat autisnya bergejolak, apalagi saat dia menjadi pusat perhatian seluruh petarung di situ.
Akhirnya Aria melarikan diri dan kembali ke khayangan. Futron semakin bergejolak setelah melihat Aria. Energi penghancur yang sangat kuat terasa. Dengan yakin dia tancapkan Debian Sword ke tanah. Tanah meretak dan gunung gundah ingin meledak.
Bledar..Duaaaar..Door..Kratak Kratak, gunung gunung di Wilmor bergejolak, memuntahkan isi perutnya, lahar dimana-mana, langit di selimuti kabut hitam kelam.
Semua tertegun... inikah kekuatan Futron atau ini adalah kekuatan Debian Sword. Hanya The Five Sage dan Anjariel yang mengerti bahwa ini kekuatan Debian Sword. Pedang keramat yang seharusnya sudah dilupakan.
Kini kehancuran Wimor tinggal menunggu waktu... Akankah itu terjadi... Di manakah kini pahlawan pujaan hati yg dapat menyelamatkan Wimor dari kehancuran...
Di tengah kekacauan yang terjadi, para prajurit setan Futron kembali bangkit dari kubur dan makin menimbulkan kekacauan di seluruh penjuru Wimor. Orang2 tak berdosa menjadi korban. Fellowship pun merasa bersalah karena telah membangunkan Para Iblis.
Sebagai makhluk yang masih mencintai kehidupan di Wimor pun berteriak dalam hati "Selamatkan Wimor..!" The Five Sage, The Fellowship dan Aria The Eldest, satu-satunya pilihan mereka adalah bersatu padu melawan Futron.
Keadaan telah berubah, pertempuran antara Iblis dengan Fasilnor telah menjadi peperangan untuk mengentikan Futron dan menyelamatkan Wilmor. Debian Sword pun ditarik, napas-napas terhenti sesaat. Semua memerah karena warna lahar.
Tiba-tiba langit terbakar api, ketika Debian Sword diangkat tinggi.
"Ini akan berakhir sekarang" dan guntur bergelegar disekitar Futron. Tanah-tanah terangkat. The Five Sage tak mampu bergerak. Hanya Ibamiel yang tahu ini memang yang terakhir.
Namun Ibamiel memilih diam, sambil memperhatikan daratan Fasilnor yang indah telah diselimuti oleh api.
DOENNNNG tiba-tiba terdengar bunyi sangkakala dari arah laut. Semua orang menengok. Tiba-tiba muncul satu armada kapal penuh dengan pasukan. Di kapal paling depan di atas haluan berdiri dengan gagah pahlawan pujaan hati Ar-Mour.
Ar-Mour sang pahlawan dalam legenda telah datang, bersama pasukan pilihan dari Fasilnor. Ar-Mour datang untuk menumpas Futron, yang telah membuat WIlmor jadi porak poranda. Sungguh mulia memang ksatria kita satu ini.
Pasukan Ar-Mour yang berpakaian putih mengkilat dengan cepat menerjang pasukan Futron. Terjangan itu membawa angin segar bagi pasukan gabungan Wimor dan semangat pun kembali membara di hati mereka untuk menghancurkan Futron dan menyelematkan Wimor.
Terjangan yg sangat dahsyat. Pukulan telak pun dirasakan kubu Futron... Keadaan semakin tidak menguntungkan mereka, ketika kapal2 perang pasukan Ar Mour menembakkan meriam2 mereka ke kubu2 pertahanan pasukan Futron.
Namun, Helmion mengetahui kelemahan Ar Mour. Ar Mour tidak tahan geli, apalagi kalo digelitikin di daerah pinggangnya!
Karena tidak tahan geli, Ar-Mour pun mengeluarkan jurus maut Jurus Ganteng Bersertifikat. Dalam sekejap jurus itu melibas kawan maupun lawan. Dunia pun semakin kelam, kehancuran dunia hanya tinggal menunggu waktu saja.
Kehancuran dunia diprediksi akan terjadi sebelum matahari terbit keesokan harinya. Jurus Ganteng Bersertifikat tidak dapat dikalahkan pasukan Futron. Namun, pada tengah malam, terjadi keajaiban: masa berlaku sertifikat ganteng Ar-Mour habis!
Merasa mendapat momen untuk melakukan serangan balik. Pasukan Futron langsung mengumpulkan sisa2 pasukan dan tenaga mereka. Futron pun mulai menyerang sambil melayang di udara dia pun melontarkan jurus bola api gattling gun ke arah pasukan Ar-Mour.
Sementara Ar-Mour kebingungan. Memperpanjang sertifikat gantengnya butuh tanda tangan ketua RT, dan membutuhkan waktu proses satu minggu tambahan. Birokrasi oh birokrasi...
Pasukan Ar-Mour dan para Sages kocar-kacir menghindari serangan gattling gun Futron. Suasana sangat mencekam karena hujan bola api terus menerus menghujam padang pertempuran. Bahkan saking kalapnya Futron sampai menyerang pasukannya sendiri.
"Bos Futron, kok nyerang kita sih?" salah satu pasukannya ngomel. "Oh.." Bos Futron mikir dulu cari alasan apa. "Sori.. sori.. salah pencet".
Semakin lama... kondisi semakin kacau... pasukan yg bertarung sudah tidak kenal mana lawan mana kawan. Cerita-cerita dan legenda yg dituturkan pun simpang siur. Tidak ada yg tahu pasti kelanjutan pertarungan dahsyat ini.
Ar-Mour bilang: "Gimana sih nih. penulisnya pada pikun semua yak!!?".
Hal ini disebabkan dulu setiap seseorangbercerita ttg kisah Yuli Saga di tengah perbincangan yang hangat, tiba2 orang2 akan langsung kehilangan gairah dan meninggalkan perbincangan. Seolah2 ada kutukan bila memperbincangkan Yuli Saga. Entah kenapa???
Lupa saya itu kenapa...
Seperti peperangannya yang carut marut, kisah peperangan ini pun carut marut. Hanya dataran penuh darah dan teriakan bagai kegilaan dari para prajurit yang terus terekam dalam kenangan.
tiba-tiba Randolf pun berteriak dengan lirih: "Bring my blood to hell, mosquitos!!". Halah, ternyata dia belum diolesin Domestos Nomos.
Karena bosen ni cerita udah setaun kok gak tamat2, akhirnya Aria the eldest memutuskan untuk mereset seluruh carut marut yang terjadi. Aria mengeluarkan skill ultimate-nya sehingg seluruh bala tentara futron (termasuk futron) lenyap tak berbekas.
"Wah... gak bisa begitu dong ar! Elo musti kontribusi juga dong. Jangan asal reset-reset aja". Tiba-tiba Futron berteriak. Ternyata dia gak keikut ter-reset.
Oleh: anwarchandra
Jadi sekarang pokoknya tinggal Aria The Eldest \& Futron. Pertarungan ini sampai tetes darah yang terakhir.Fisiphenia, negeri yang tidak jauh dari Fasilnor. Negeri hijau yang indah dan makmur dimana penduduknya ganteng dan cantik-cantik. Di negeri inilah Satron berkuasa dan memerintah dengan bijak.
Pada suatu masa Satron pernah dikenal sebagai tiran, pada zaman Peperangan Besar dengan Aria The Eldest. Namun kini ia menebus dosanya, membangun sebuah negara yang adil dan sejahtera. Dunia di mana persamaan hak antarras dijaga dengan baik.
Namun beberapa hari ini beredar rumor di penginapan dan pasar-pasar, bahwa Satron Yang Agung baru saja kehilangan putri yang sangat disayanginya. Orang-orang membicarakan bahwa Yang Mulia Satron mengurung diri dalam aulanya selama berbulan-bulan.
Satron sangat bersedih atas kejadian yang menimpa Amelani, anak Satron dari hasil perselingkuhan dengan istri sang Saudagar dari Utara. Amelani merupakan anak kesayangan Satron yang Agung, seperti sayangnya Dewa Zeus pada Heracles putranya..
Sejak saat itu langit di Fisiphenia lebih gelap dari biasanya. Anggur-anggur yang dibuat terasa lebih masam dari sebelumnya. Sapi dan domba tidak menghasilkan susu, dan panen gandum tahun ini terancam gagal. Semua orang resah.
Semua bertanya-tanya, ada apakah gerangan yang terjadi. Kejadian seperti ini pernah terjadi sekali, saat Satron kehilangan Yulion yang dikubur hidup-hidup oleh Aria The Eldest, seperti diceritakan turun-temurun dari generasi ke generasi.
Melihat keadaan seperti ini rakyat Fisiphenia pun berembuk. Mereka memutuskan mengirim perwakilan untuk menemui Satron Yang Agung yang sedang mengurung diri. Orang-orang yang ditunjuk sebagai perwakilan adalah...
***

Tinggalkan dulu keributan tersebut dan melihat ke sebuah sekolah di Fisiphenia. Dua orang tokoh yang bernama Randolf Hidler son of Randalf hasil perkawinannya dengan bangsa elf. Luthien Tinaviel putri bangsa Elf yang juga sedang bersekolah.
Randolf adalah salah satu dari mereka yang garis keturunannya terjaga dengan baik. Kisah yang diceritakan dari kakeknya, Randuin Putra Aditarii adalah bahwa Hati Fisiphenia tertaut dengan Hati Satron. Kegelapan akan kembali menyelimuti negara ini.
Oleh karena itu Randolf merasa terpanggil untuk menyelesaikan problema ini. Dia mengajukan dirinya sebagai perwakilan untuk menemui Satron. Namun sayang ia ditolak oleh Para Dewan Tetua Fisiphenia, disebabkan ia belum cukup umur serta blasteran.
Randolf, seperti bangsa Cheribouw memiliki penampilan seperti leluhurnya, hitam dan tetap kribo. Sifat blasterannya membuat ia memiliki kuping elf, tapi kulitnya hitam, aneh kan.Tampang juga pas-pasan, tetapi ia memiliki keberanian yang sangat besar.
"Buseng..gw lagi gw lagi..hina aja teruuusss..saitooon.." batin Randolf dalam hati karena disetiap cerita dia selalu ada dan selalu dihina-hina.
Memang begitulah kondisi di dunia Wimor ini, rasisme walaupun ditentang tapi dalam praktiknya masih ada. Tapi untuk bangsa Cheribouw lebih disebabkan oleh sifat mereka yang tengil dan selalu minta ditonjok.
Tapi salah seorang dari Dewan Tetua kembali mengingatkan pada senat, "Satron Yang Adil selalu menginginkan persamaan hak terhadap semua ras, tidak peduli berkulit hitam atau putih, kaya atau miskin. Tidakkah kita mengikuti ajarannya yang baik itu?".
Semua orang disana mengangguk... namun masih ada sebagian yang tampak ragu dan tidak setuju.
"Tapi kita semua tahu bahwa Satron tidak suka orang yang tengil dan selalu minta ditonjok kelakuannya" sergah salah satu dari dewan tetua yg bernama....
"Apalagi dia hitam dan blasteran kangguru dan doberman", tambah yang lain, "Betul, mirip Adunil hitamnya". Suasana council pun ricuh. Satu orang beradu mulut dengan yang lain. Tiba-tiba Satron pun muncul.
Rupanya beliau cuma numpang lewat mau pergi ke WC. Maklum sudah sebulan dia tidak keluar kamar.
Setelah itu dia kembali, berjalan ke tengah Aula Dewan. Dalam diam dia menatap setiap orang yang ada di situ. Seluruh ruangan terdiam, beku, begitu dingin. Tak ada yang berani bicara. Sunyi.
Tiba-tiba salah satu dari mereka mencoba memecah keheningan dengan menceletuk "Ke pantai yuk, Tron!".
Setelah itu ada seorang lagi yang berusaha mencairkan suasana, "Bego lu Tron".
"Udah udah..semuanya diam!!!" kata Satron.
Rupanya masih ada seseorang yang terlambat berkomentar tapi sangat ingin berkomentar, "Waah jangan gitu Troon.".
"Saat ini aku sedang dalam kesedihan yang luar biasa, bisakah kalian tenang?" tanya Satron, "Belum puaskah kalian para Dewan Tetua berdebat setiap hari, Hah? Hah? Hah?" Satron semakin emosi.
"...." Satron akhirnya speechless menatap tingkah polah anggota Dewan yang sama sekali tidak terhormat.
Tetap ada salah satu dari mereka yang masih berusaha berramah-tamah, "Jadi gimanna Tron?".
Tiba-tiba "Pyuuungg!!" telunjuk Satron mengeluarkan sinar yang langsung tepat menghunjam kening anggota Dewan Tetua yang kurang ajar itu.
Anggota Dewan Tetua yang terkena sinar Satron itu tiba-tiba mati seketika, lenyap tanpa bekas.
Satron pun kembali masuk ke Istananya. Tak lama kemudian para dewan kembali dalam suasana ricuh.
"Udah gitu doank?? Lewat sini cuman pengen ke kamar mandi? Parah betul..gw pikir ada apaan" kata seorang Anggota Dewan pada sebelahnya.
Ketua Dewan tetua yang malu untuk disebutkan namanya memukul2 palu ke mejanya selain untuk menenangkan sidang dewan juga sekaligus untuk membetulkan mejanya yg memang sedikit goyang.
"Hadirin sekalian...hari ini kita berkumpul di sini dalam rangka membahas sebuah masalah yang sangat membuat rakyat kita gelisah" kata Ketua Dewan Tetua.
"benar..benarr.. Jadi gimana nih coy...", timpal para anggota dewan lainnya.
"Apa yang terjadi pada Satron yang Agung? Apakah ada diantara kalian yang tau ada apa gerangan dengan Yang Mulia Satron?" tanya Ketua Dewan Tetua.
Bukankah itu tujuan kita mencari perwakilan untuk dikirim ke menemui Satron. Agar kita tahu ada apa dengan Satron yang Agung.
"Oh iya betul!! Kepala saya rasanya penat sekali, semalam saya baru mencoba meminum pil biru yang konon ke sohor itu" cerocos Ketua Dewan Tetua.
"Ah sial..kenapa gw ngomong.." batin Ketua Dewan Tetua karena tak mau di anggap 'tidak mampu' oleh Anggota Dewan yang lain. "Jadi bagaimana? Apa keputusan kalian?" kata Ketua pada anggota Dewan.
"Saya mengusulkan Kepala Intelijen negeri kita Anjariel of Euller untuk menemui Satron Yang Agung, dia sudah banyak pengalaman dalam hal mengorek informasi" jawab salah satu anggota dewan tetua.
Seorang anggota Dewan angkat bicara, "Sebelum itu, saya hanya ingin mengingatkan sebuah ramalan tua tentang kejatuhan kedua Satron. Nasib negara ini tertaut pada keberadaannya!".
Dalam ramalan tertua yang pernah ditulis oleh Fitriel, ada sekelompok orang yang dapat mengalahkan Ar-Mour. Namun kelompok orang tersebut sudah menghilang dan sulit ditemukan di muka Wimor ini.
Namun sayang, Fitriel sudah mati saat terjadinya perang besar dalam legenda.. dan hanya ada satu cara untuk mengetahuinya. "Kita harus menghidupkan kembali Fitriel!", celetuk salah seorang Dewan. "Tapi..Bagemana caranya? membaca saja aku sulit".
"Di di maen bola lagi yuk". tambah anggota dewan lain. Tiba-tiba Wong Cen Lau, salah satu wise man dari negeri timur berkata, "Tenang saudara-saudara. Fitriel mempunyai banyak anak, dan setau saya salah satu anaknya masih hidup.".
"Oh iya siapa namanya? bisa di temukan dimana dia?" tanya Ketua Dewan Tetua. "Namanya adalah..Mithiel, jangan lupa pake h..." ujar Wong. Seluruh anggota Dewan serentak berkata "Ya iyalah pake h..kalo pake p kan jadi Pithiel".
Mithiel, daughter of Fitriel. Dialah satu-satunya yang mungkin mengingat ramalan tua mengenai kejatuhan kedua Satron, yang juga terkait dengan ramalan mengenai kekalahan Ar-Mour.
"Berarti sekarang kita perlu menemukan si Mithiel pake h ini" ujar Ketua Dewan Tetua. "Segera perintahkan pihak intelijen untuk menelusuri keberadaan Mithiel pake h ini" sambung Ketua Dewan Tetua lagi.
"Tidak perlu!" sela Wong. "Menurut catatan Fitriel yang saya ambil di tukang fotokopi, Mithiel which is the daughter of Fitriel ada di sekolahnya Randolf which is kribo item. ".
Tapi tanpa disangka-sangka sekolah Randolf sangatlah jauh. Terletak di ruang sidang di dalam Gunung Gridas, daerah ujung selatan di Fasilnor. Konon katanya daerah tersebut didiami oleh monster dan penghuni yang kejam.
"Alhamdulillah" syukur Anjariel of Euller namun agak terlambat, maklum dia dapat infonya terlambat. Dengan begini tugas untuk dirinya tidak jadi. Dia pun kembali menjadi cerpenis sekaligus narator cerita ini.
***

Beberapa utusan Dewan Tetua segera memulai perjalanan mereka ke sekolah Randolf. Mereka berjumlah 3 orang.
Setelah melewati 3 hari 3 malam penuh rapat yang sengit, akhirnya diputuskan lah nama-nama orang yang dikirim. Dia adalah Wong Cen Lau si bijak dari timur dan Randolf yang dikirim sebagai tumbal karena anggota dewan males melihatnya.
Satu lagi adalah Direzion son of Herzon Pemburu dari Selatan, dia datang menyusul, dikarenakan rumahnya lebih dekat dengan sekolah-nya Randolf.
Dia diajak sebagai penunjuk jalan oleh Wong Cen Lau. Awalnya dia tidak mau, tapi apa daya, Wong dan Randolf adalah temannya. Lagian tampang melas dari Randolf meyakinkannya untuk menempuh perjalanan menantang mara bahaya ini.
Dalam hati dia bertanya "Nih bocah Chreibouw bego banget sih, moso sekolahannya sendiri kagak tahu dimana tempatnya, kagak pernah sekolah apa ini bocah Chreibouw?".
Ternyata sekolah Randolf cukup unik. Walau tempatnya sudah diketahui umum, jalan masuknya berpindah-pindah setiap hari, sehingga hanya pemburu handal seperti Direzion yang mampu melacaknya tanpa kesalahan.
Sedangkan Randolf bagaikan orang linglung, dia seperti lupa klo dia sekolah di sana. Wajar saja ternyata ini adalah untuk pertama kalinya dia datang ke sekolahnya, selama ini ia selalu menitip absen.
"Di situ", ucap Direzion sambil menunjuk ke utara tempat mereka berada. Saat itu kabut cukup tebal meliputi daerah tersebut, sehingga hanya insting Direzon saja yang meng-guide mereka.
Namun, belum selesai Direzion berbicara, muncullah 3 makhluk yang siap menghadang mereka. Rintangan pertama bagi tiga sekawan.
Satu di antara ketiga makhluk tersebut serta merta menyerang dengan giginya yang tajam ke arah Randolf yang saat itu sedang melamun dengan bodohnya.
Di saat yang sama, Direzion dan Wong Cen Lau bekerja sama menyerang dua makhluk lainnya.
Direzion dan Wong Cen Lau menghabisi makhluk-makhluk tersebut dengan mudahnya. Sedangkan Randolf, sudah tidak suci lagi dibuat makhluk itu.
"jrooot" gigi tajam makhluk tersebut menghunjam ke rambut kribo Randolf, dalam sekali hingga terlihat kepala Randolf terbuka menganga. Namun tiba-tiba makhluk itu langsung terkapar, ternyata rambut bangsa Chreibouw memiliki racun ganas.
"Wahahaha, emang enak makan rambut gue? Belom keramas sejak lahir nih!", ujar Randolf.
Tapi tiba-tiba Randolf merasa lebih lemot dari biasanya. Ternyata makhluk tersebut adalah Prastudil, son of Prastudon. Seorang makhuk Geek yang gemar menghisap ilmu para korbannya.
Ternyata makhluk yg terkapar itu tiba-tiba berubah bentuk menjadi manusia. Rupanya selama ini ia dalam pengaruh kutukan. Berkat rambut bangsa Chreibouw yg ternyata lebih terkutuk, dia terbebas dari kutukan sebelumnya.
Tibalah mereka di depan pintu gerbang.
Ternyata pintu gerbangnya salah. Ini adalah pintu gerbang Pasar Malam Gunung Kidul. "Bang*tuuuuuttt*, salah jalan gw, kita muter lagi lah ke arah barat", tegas Direzion.
Tapi sudah terlambat, Randolf sudah keburu masuk ke dalam Pasar Malam, begitu mendegar alunan musik dangdut kegemarannya bertalu-talu dari arah dalam Pasar Malam.
"Aseeek, musik gue banget nih, Guys!", seru Randolf.
Dalam waktu 5 detik Randolf langsung hilang dalam keramaian Pasar Malam, bahkan Direzion san Pemburu dari Selatan tidak bisa mencium bau Randolf. Mungkin karena ia bangsa Chreibouw, yg memang susah dibedakan baunya dengan bau bangkai.
Namun, yang sebenarnya terjadi adalah..Pasar Malam tersebut adalah sebuah ilusi dari penunggu daerah tersebut...
Prastudil, teman baru mereka (dia sudah membuktikan dirinya melawan Randolf) berhasil menarik Randolf ke jalan yang benar.
Prastudil sangat menguasai medan tersebut, dan meminta untuk ikut menunjukkan jalan.
"oiii bangsaaad, terus gimana iniii??" maki Direzion pada Prastudil lupa disensor.
Rupanya mereka telah salah sekolah. Mereka telah masuk ke MIT (Margonaut Institute of Tenung), black magic school yang sangat berbahaya dan kesohor akan alumni2 buasnya.
Hanya Rektor-nyalah yang paling disegani oleh para almuni, yakni Margonout Ear One...
Tapi Rektor tersebut jarang masuk kampus.
Hal itu membuat murid terbandel dan tersampah sepanjang sejarah persekolahan pun hormat padanya. Dan sering mengajaknya bermain ping pong.
Margonaut The Ear One adalah salah satu dari The Five Sages of Fasilnor.
Selain itu, Margonaut adalah master pingpong. Terutama setelah kepergian Amelani, Margonaout tidak terkalahkan dalam urusan ping pong.
The fellowship pun mulai ragu akan perjalanan ini. Mereka cukup segan terhadap salah satu dari Five Sages of Fasilnor yang sangat kuat. Lagipula jika sudah masuk ke daerah kekuasaan mereka, sudah dipastikan sulit untuk keluar hidup-hidup.
Tuh kaan beneer..!! Randolf, Prastudil, Direzion \& Wong seketika itu dikelilingi oleh makhluk-makhluk cantik bertaring.
"Wanjeeer...apalagi ini..ada makhluk cantik bertaring segala??? Gag kebayang gw" celetuk Randolf tiba-tiba. "Eee buset,,," kata Prastudil, Direzion dan Wong kaget dengan celetukan-celetukan ajaib Randolf, yang biasanya selalu tidak penting.
Salah satu dari makhluk cantik itu dicubit pipinya oleh Randolf "iiih lucunya.." membuat mereka berang dan menggigit liar-liar asik.
Ternnytaa Randolf sama sekali tidak memiliki ketakutan terhadap makhluk-makhluk cantik bertaring tersebut. Justru sebaliknya merekalah yang menjadi takut dicubit oleh Randolf. Memang aneh kekuatan yang dimiliki oleh Randolf Hidler son of Randalf...
Setiap pipi manis yang dicubit oleh Randolf menjadi kasar, rusak dan basi.
Teriakan-teriakan liar mulai bersahutan, makhluk-makhluk itupun langsung kabur menjauhi Randalf. Konsekuensinya adalah, kawanan Randalf pun menjadi selamat dari serangan mereka. Mereka pun melanjutkan kembali perjalanan mereka.
Di tengah perjalanan, mereka melihat seorang gadis nyangkut diatas pohon. "Hey guys! bisa tolong turunin gw which is dari atas pohon ini?".
"..." mereka berempat saling berpandangan. "Brebet" dalam sekali loncat, Direzion sudah berada di atas pohon dan membawa gadis itu turun ke bawah. "Apa yang sedang Adinda lakukan di atas pohon?" tanya Direzion sok gombal.
"Well, tadinya gue lagi mau ngambil rempah-rempah di hutan ini untuk ramuan. Tetapi entah dari mana tiba-tiba ada sekelompok makhluk cantik bertaring yang berlari seperti mengejar gue! Gue langsung aja manjat ni pohon", ujar gadis which is.
"Oooh gitu.." kata mereka (berempat bukan berlima) serempak. "Btw..kamu siapa?" tanya Randolf mengulurkan tangan berusaha mengajak kenalan yang langsung di-'cie cie'-in sama yang lain, gag penting sih, tapi lucu aja.
"Mithiel Haren", ujar gadis tersebut seraya menyambut uluran tangan Randolf. Seketika mereka berempat terperanjat, "Mithiel Haren daughter of Fitriel?", mereka serempak. "Iya, Mithiel. Jangan lupa pake h ya.", balas gadis tersebut.
Sebelum mereka sempat beramah tamah lebih lanjut. Tiba-tiba "Brebet!! Slepet!!" Mithiel diculik oleh Ray Shaiton, murid kesayangan Margonaut the Ear One. Sangat cepat, bahkan Randolf menyadari retsletingnya terbuka, Ray Shaiton telah hilang dari situ.
Suara teriakan Mithiel Haren terdengar bergema di antara pepohonan yang tinggi, "Hey guys! Selamatkan aku ya, which is dari penculikan orang aneh ini. Thanks anyway.".
Dalam waktu se-brebetan-an Ray menghilang dari pandangan. "E buset..siapa itu tadi?" tanya Direzion pada rekannya. "Kayaknya gw tau.." kata Wong yang dikenal berpengetahuan luas. "Dia adalah Ray..Shaiton, anak murid Margonaut" lanjut Wong.
"Tidak mungkin!" balas Randolf terkejut. "Kakekku pernah bercerita tentang Klan Shaiton. Mereka hidup di dalam bayangan dan kegelapan, which is keberadaan mereka selama ini hanya sebatas legenda. Halah kok ketularan jadi which is juga sih gw".
"Nah, itu dia alasannya mengapa Ray begitu cepat bergerak.", ujar Prastudil secara analitis, "Untuk mengalahkannya, kita harus mempersiapkan jebakan baginya di pagi hari!".
"Klan Shaiton? in Great danger we are kawan.. Waspada kita harus...", kata Prastudil memperingatkan kawan-kawanya.
"Dia tidak akan masuk ke dalam perangkap," bantah Wong. "Justru kita harus mengejarnya dengan cepat!" Lalumereka berempat melesat menelusuri jejak Ray Shaiton. Setelah berapa lama mereka menemukan sebuah keranjang rempah bertuliskan which is.
Di keranjang itu terdapat sebuah selebaran. Selebarannya berisi tentang perkawinan Jalesun dan calon istrinya. Ternyata Mithiel temannya Jalesun! Fellowship cukup terkejut, karena dalam legenda, Jalesun sudah mati terhisap lobang yang dibuat Ar-Mour.
"Jalesun yang dalam legenda dia adalah seorang yang ahli menghilang dan bergerak secepat kilat?" kata Randolf heran. "Berarti dia masih hidup?" tanyanya kemudian.
"Semua orang mengira ksatria legendaris itu telah mati. Kalau begitu, kita lewat sini! Kita belum terlambat," seru Wong. Tak berapa lama mereka menemukan lukisan 'pre-wed' Jalesun dan calon istrinya, beserta kartu ucapan selamat dari Mithiel.
"Hehe, lama ga keliatan tau2 nyebar undangan" celetuk Prastudil pada Randolf tiba-tiba.
"Hooiiii! kejar gw lagi dong!", tiba-tiba Ray Shaiton muncul di depan mereka. Membuat mereka melongo semua saking kagetnya. Dan "Plop!!" Ray Shaiton hilang kembali.
"Guys, ayo dong jangan cuma ngebahas Jalesun," teriak Mithiel samar-samar. "Aku masih diculik nih. Ayo ikutin si penculik ini, which is ke arah gunung, menara yang paling tinggi di sini. You can see that.".
Keempat pendekar kita melongo lagi. Lalu tiba-tiba terdengar suara, "Si Mithiel ngomongin menara apa si?" tanya Randolf gag ngerti. "Pras, Rez..Wong..tau gag lo itu menara apa??" tanya Randolf kemudian.
"Kemungkinan yang dimaksud adalah Menara Suci Kaum Shaiton," gumam Wong sambil memejamkan mata. "Dan arahnya kesana." Dia menunjuk ke arah papan nama 'Menara Suci Kaum Shaiton, 10 KM lagi'.
"Umm... Menara Suci kaum Shaiton? Kok sepertinya kontradiksi ya?", ujar Prastudil.
"Ah, deket klo gitu, lurus aja Coy!" kata Randolf sok tahu. Baru 10 langkan berjalan, mereka terhenyak, ternyata di depan mereka terdapat jurang. Mereka dapat melihat Menara itu, namun mereka bingung bagaimana melewati jurang tersebut.
Ketika Randolf melihat ke bawah tampak seperti lautan manusia. Ternyata itu adalah jurang tempat pembuangan bencong. Tentu saja Randolf tak mau terjatuh ke jurang itu, karena...
Ia sedang puasa, ia takut tidak dapat menahan nafsunya yg mana dapat membatalkan puasanya hari ini.
Selain itu, kalau jatuh, dia bakal mati. Dia tidak ingin nasib serupa dengan ayahnya, yaitu mati dalam kegelapan.
"Hmm, kita harus mencari jalan lain," seru Direzion. Mengikuti tips and trick dari Wong, dia berusaha mencari papan nama 'Jalan Lain', tapi tidak ketemu.
"Bagaimana kalau kita mencari jalan alternatif saja?", pikir Wong. Dan ternyata, terdapat papan nama "Jalan Alternatif" dalam jarak pandang mereka.
"Coba kita bertanya ke Googlarium!" celetuk Randolf. Googlarium adalah benda sakti yang bisa mencari semua jawaban di dunia Wimor. "Ada yg tahu alamatnya Googlarium?" tanya Direzion.
"Goglarium.com" seru Randolf karena kesal setiap kali ingin berkontribusi dalam cerita ini selalu keduluan.
"Goblog elo Ndolf, mana ada alamat seperti itu, itu emang protocolnya apaan?" maki Direzion.
Singkat cerita, Googlarium menjawab "Telusuri Jalan Rahasia di balik dunia, di mana Matahari Yang Perkasa tak akan pernah menyentuh tanah kaki di mana kalian berjejak.".
"Lain kali jika ingin menghubungiku, mintalah bantuan Anjariel of Euller di nomor 007, dia tahu dimana aku berada" sambung Googlarium.
"Aku juga buka sms hotline lho. Sms aja ke 212. Semua sms yang kamu terima, langsung dari henponku!" seru Googlarium sebelum menghilang ditelan asap putih.
Tiba-tiba seekor tikus raksasa keluar dari tanah, kemudian kabur. "Aku tahu maksudnya," seru Direzion sambil tersenyum, "Liang-liang Tikus Edwing selalu bermuara di kaki menara-menara!" Lord Edwing adalah penguasa tikus-tikus raksasa di Fasilnor.
"Sial, sepertinya kita memang harus melewati Lorong Tikuzon di bawah tanah", kata Direzion, "Walaupun aku tahu, banyak mahluk berbahaya yang merangkak dan siap menerkamn disana". Akhirnya dengan gemetaran, The fellowship maju menuju Lorong Tikuzon.
***

Setapak demi setapak lorong gelap itu disusuri oleh empat sekawan itu demi menyelamatkan Mithiel Haren. Entah apa yang menanti mereka di depan.
Baru masuk semeter tiba-tiba "Plup!" mereka sudah keluar di ujung jurang yang lain. Rupanya ini lobang ajaib.
"Wahai jurang dan segenap isinya, lain kesempatan aku akan mengunjungimu! Sekarang ada urusan yang jauh lebih penting dari sekedar menuruti nafsu!" batin Randolf dalam hati.
"Wahai Shaiton! Dengan kekuatan Yang Mulia Satron, kami akan menghukummu!" seru Direzion lantang. Mereka melanjutkan perjalanan.
Sesuai jawaban dari Googlarium, menelusiuri jalan tikus mendekatkan mereka pada tujuan mereka. Di jalan perempatan pertama yang mereka temui, terdapat tanda jalan ke arah kanan: "Menara Suci Kaum Shaiton, 14km" .
Mereka belok ke kanan dengan percaya diri, berjalan ringan melawan angin sepoi-sepoi yang mengibaskan rambut mereka, diiringi lagu rock alternative yang melankolis. kayak para jagoan di tipi-tipi.
Ternyata belokan ke kanan malah membawa mereka menuju lorong-lorong yang lebih gelap dari sebelumnya. Memang daerah sini terkenal dengan lorong yang menyesatkan. Dan juga terkenal dengan minuman khasnya Jus Shaiton Arab.
"Randolf mana ya?" Wong menyadarkan Direzion dan Prastudil akan Randolf yang menghilang. "Sori men, gw salah belok arah ke kiri, soalnya gw masih bingung mana yang kanan mana yang kiri".
"Kiri tuh arah tangan lo yang biasa dipake cebok!" dengus Direzion pada Randolf yang semakin kebingungan akan pernyataan kawannya itu. FYI, Randolf selama ini belum pernah cebok menggunakan tangan.
Ya, biasanya Randolf menggunakan tisu. Kayak orang-orang Barat gitu lho...
Kalo ga ada tisu, rambutnya pun bisa jadi pengelap.
Prastudil yang pintar kemudian mengeluarkan sebuah kitab dari tasnya, "Kalau begitu baca ini!" Kitab itu bertuliskan 'Kanan Kiri for Dummies', halaman pertama bertuliskan 'Jangan kelamaan cerita yang ga penting cuy, lanjut petualangannya!'.
Tiba-tiba terdengar suara dari kejauhan.. "Hei guys.. tolong selametin gw dong which is ada diatas sini!!". Jelas sekali itu suaranya Mithiel pake h.
"Ah ini pasti perangkap!" bisik Wong yang sangat berpengalaman dalam pembebasan sandera. "Tidak mungkin semudah itu, kita pasti akan dijebak!".
"Ah udah lah langsung dah kita ke atas", kata Randolf sambil dengan sembrono menginjakkan kaki ke dalam menara Shaiton. Fellowship yang lain pun terpaksa mengikutinya, padahal mereka tau betapa menyeramkannya menara Shaiton.
"Guuys.. ini beneran gw, Mithiel, pake h seperti pada thompson" Teriakan itu pun semakin jelas terdengar.
"Iya, sepertinya memang itu Mithiel", seru Prastudil, "sebab kita telah menempuh jarak 14km sejak belok kanan tadi. Tapi hati-hati, siapa tahu masih ada jebakan.".
Benar saja kata Prastudil! Karena tiba-tiba Randolf berteriak "Ah sial! gw nginjek tae kuda".
"Tae kuda bukan ya?" tanya Randolf, "Coba dulu baunya" kata Randolf baunya sih tae kuda. "Kalo rasanya?" lanjutnya lalu mencicip dikit "Wanjeng beneran tae kuda..Setan..Siapa si yang naro tae kuda di tengah jalan menara???" Umpat Randolf kesal.
Tiba-tiba Wong ikutan mencicipi tae kuda tersebut. "Hmmh.. rasanya agak aneh. Tae kuda di kampung gw lebih manis soalnya" Celetuk Wong.
Dengan penasaran akhirnya si Direzion dan Prastudil ikut-ikutan. "Bener euy..kurang manis..." Kata mereka berdua serempak.
Prastudil pergi ke kantin menara Shaiton. Lalu kembali membawa gula merah dan mencampurnya dengan tae kuda. Dia mempersilakan teman-temannya untuk mencobai rasanya apakah sudah manis.
"Kalian itu ngapain sih?" Tiba-tiba Anjariel of Euller datang dan bertanya.
Dilihatnya tae kuda dicicipi mereka. Anjariel pun mengeluarkan butir-butir tae embek dari kotak makannya. "Dicampur enak nih pasti". Tiba-tiba terdengar suara teriakan Mithiel..
"Hooi.. selametin aku dulu dong! Gak pake lama!".
Mereka berempat pun bergegas lari ke arah sumber suara. Mereka sadar telah membuang-buang waktu. "Gawat...nyawa Mithiel harus kita selamatkan" kata Randolf diiyakan oleh yang laen. "Benar demi Paduka Yang Mulia Satron" timpal Prastudil kemudian.
"Eh gw ga diajak nih?" tanya Anjariel of Euller sambil muntah-muntah saat tersadar dia telah melakukan perbuatan hina.
"Udah gag usah banyak cebret..langsung brebat lah" kata Randolf pada Anjariel. Anjariel pun berlari menyusul mereka berempat.
"Kalo gitu naek kereta kuda saya saja" Anjariel pun bersiul memanggil kereta kudanya. Sebagai kepala intelijen ia mendapatkan kendaraan dinas, yang tentu saja gratis pula biaya perawatannya.
Memang enak ya jadi pejabat, pikir mereka berempat melihat kereta kuda Anjariel.
Ketika mereka naik kereta kuda Anjariel, the fellowship merasakan keanehan. Kuda-kuda ini tidak seperti biasanya.
"Kok kudanya item dan kribo sih?" tanya Randolf. "Iya dulu induk kuda2 ini pernah diperkosa oleh orang gila yg bernama Randuin son of Aditarii" jawab Anjariel sambil mengendali kuda supaya baik jalannya.
Tuk tik tak tik tuk tik tak tik tuk tik tik tak tuk.. suara sepatu kuda tuk tik tak tik clep clep clep! nginjek tae kuda. Mereka pun sampai lah di..
Pasar Rumput Fasilnor, untuk mengisi rumput kereta kuda yang semakin menipis, "Pertamax ya mas!!" pinta Anjariel ke petugas SPRU (Stasiun Pengisian Rumput Umum).
"Woi guuys! gw dah jenggotan nih nungguin kalian which is lama banget nolongnya" Suara teriakan itu terdengar lagi.
"Oooo, kita mo nyari Ray Shaiton ama Mithiel? Bilang dong dari tadi" Anjariel langsung mengeluarkan jurusnya dimana langsung terkoneksi ke Googlarium dan langsung menemukan lokasi mereka. Anjariel pun mengedipkan matanya, dan "Plop!" sampailah di ...
Gerbang menara Shaiton lagi. "Loh kok balik lagi sih" kata Randolf. "Hmm, ternyata kita sedang dipermainkan" Wong menimpali. Rupanya menara itu benar-benar Shaiton alasion, The fellowship pun dibuat bingung karenanya.
Akhirnya mereka pun memasuki menara Shaiton. Mereka terkejut karena melihat di dalamnya tidak ada apa-apa. Hanya sebuah tangga melingkar yang cukup terjal untuk dinaiki. Tanpa babibu mereka langsung menaiki tangga tersebut tanpa menyadari bahayanya.
Tiba-tiba Ray Shaiton terbang ke arah Wong dengan cepat mendaratkan pukulannya. Kapow!.
"wadooww... setann.. siapa tuh yang nampol gw", umpat Wong sambil mencabut giginya yang patah karena pukulan tersebut.
Pertarungan pun tak dapat dihindarkan walaupun mereka sedang menaiki tangga. Ray Shaiton tidak lupa membawa anak buah yang tak kalah ganasnya, Mirfan the Red Beard. Dwarf yang satu ini sangat lihai memakai Kampak Naga 212.
Tanpa babibu, Mirfan menyerang dari sektor kiri pertahanan fellowship. Dia berspekulasi dengan melancarkan Kampak Naga 212 dan... Crot! Anjriel kehilangan kepalanya.
Tapi Anjariel tanpa kepala membalas Mirfan dengan sihir putih yang melesat dari tangannya. Mirfan the Red Beard terlempar jauh ke dasar menara. "Aku sudah lama mati," seru Anjariel kepada keempat temannya, "jadi tidak bisa mati lagi. Tenang saja Bro".
Selain tidak bisa mati, Anjariel memang susah matinya.. daya tahan tubuhnya mirip bagai kecoa yang ga mati-mati meski uda di injek-injek ampe gepeng...
"Yah mukanya memang mirip sama kecoa sih", celetuk Wong sambil bersiap-siap menghadapi serangan berikutnya dari Mirfan dan Ray. Kali Ray membisikkan aba-aba menyerang pada Mirfan. Mereka pun langsung maju menyerang berduaan dengan mesra.
"Kok bisa, yang tadi?" tanya Prastudil kepada Anjariel. Anjariel menjawab, "Aku mengikatkan nyawaku mengabdi pada Satron. Sebenarnya aku sudah mati ratusan tahun lalu. Tapi aku akan terus hidup untuk melayaninya.".
Pembaca mungkin jadi bingung, tapi ceritanya memang begitu. Dalam dunia perdongengan, semuanya mungkin. If there is a will, there is a way.
"Karena hidup adalah perbuatan," lanjut Anjariel sambil menatap menyongsong masa depan.
"Setan, banyak omong kalian", seru Ray sambil berpegangan tangan mesra dengan Mirfan, "Kita lagi mo nyerang nih, siap siap dong ah". Kemudian akhirnya mereka melanjutkan serangan ke jantung pertahanan the fellowship dari sebelah kiri.
Namun sebelum itu dengan kekuatannya Anjariel menyambungkan lagi kepalanya. Prosesnya tidak terlalu jelas, karena selama proses penyambungan tubuh Anjariel bersinar sangat menyilaukan. Dan ketika sinar itu hilang Anjariel kembali in one piece.
Prastudil lalu mendapatkan ide cemerlang, "Randolf, kau selamatkan Mithiel. Kami berempat mengalihkan perhatian. Karena ... hanya kau yang bisa ... bersembunyi dalam kegelapan." Kemudian Prastudil membalikkan badan dan ketawa-ketawa gila sendirian.
Randolf lalu mengeluarkan jurusnya. Dia mencari jalur yang gelap menuju tempatnya Mithiel.
Saking hitam kulitnya, dia tampak seperti menghilang ditengah bayangan benteng. Bahkan Ray yang punya mata Shaiton Alas juga tidak dapat melihat kemana perginya Randolf, sehingga ia marah dan mulai menyerang kembali the fellowship.
"Syuuuung" "Blats" Tiba-tiba Anjariel menembakkan sinar laser dari telunjuknya ke arah Ray dan Mirfan. Namun sayang mereka bisa menghindar dan sinar laser menghancurkan pilar2 di menara Shaiton.
Plak! Ray memukul Direzion. Kapow! Wong membalas Ray dari belakang. Jeb!! hantaman dari Mirfan untuk Wong. Crot!! Prastudil meludahi Mirfan. Crash! Anjariel salah sasaran menghantam kepalanya Prastudil.
Sementara itu, Randolf sampai di tingkat paling atas. Di hadapannya terdapat tiga Mithiel yang terikat. Ini sihir! Yang mana yang asli? Pikirnya.
Mithiel pertama bilang:"Ini saya Mithiel asli, Mithiel pake m seperti pada thompson".
Mithiel di ujung berkata; "Ku tak perlu berkata apa-apa untuk membuktikan. Karena memang akulah yang asli".
Mithiel yang kedua (di tengah) berucap; "Saya yang asli, Mithiel pake h seperti pada thompson". Randolf jadi ling-lung.
Lalu Mithiel yang di ujung berkata lagi, "Aku Mithiel yang asli, who is pake i seperti pada thompsoni".
Mithiel yang pertama juga tak mau kalah, "Dia palsu, aku Mithiel asli where is paling smart".
"No smoking!" kata Mithiel kedua saat Randu akan mengeluarkan rokoknya, saat ia coba rehat sejenak, karena ia pusing kebingungan.
"Ah tolol!" tiba-tiba ada suara dari belakang, ternyata Prastudil.
Ternyata Prastudil berhasil lolos dari kepungan Ray dan Mirfan, kemudian ia berinisiatif untuk membantu Randolf yang kebingungan. Sementara Direzion, Anjariel dan Wong masih bertarung dengan Ray dan Mirfan.
Namun sayang sekali akibat kepalanya terhantam tidak sengaja oleh serangan Anjariel sebelumnya, Prastudil mengalami gegar otak yg membuat ia jadi tolol.
"Eh tolong mithiel yang asli angkat tangan dong", pinta Prastudil. Mithiel kedua dan ketiga angkat tangan. "Naah, berarti kalian berdua palsu. Kalian kan harusnya masih terikat, kok bisa angkat tangan?", kata Prastudil.
Randolf pun percaya terhadap penilaian Prastudil, sehingga ia akhirnya melepaskan ikatan tangan Mithiel pake h yang berada di tengah. "Nah sekarang kau bebas Mithiel. Ayo ikut kami", kata Randolf penuh wibawa.
"Huahahahahaha", tiba-tiba Mithiel yg dilepaskan oleh Randolf berubah menjadi Troll raksasa. Ternyata dia adalah Aditron the Bringaz, troll sekutu klan Shaiton.
"Tolol lu Randolf! Mithiel yang asli itu yang nomer 1, berarti yang di kiri! Ngapain lo lepasin yang tengah!!", umpat Prastudil dongkol. Sementara itu Mithiel yang asli cuman bisa terbengong2 melihat ketololan Randolf.
"Kalian itu kok datangnya which is lama sekali sih," kata Mithiel yang sekali.
"Duarrr", Aditron tiba-tiba mengayunkan gadanya ke tengah-tengah mereka dan membuyarkan perseteruan mereka sejenak.
Duarr Gada itu malah menghajar Aditron. Harap maklum, karena ini debut pertamanya Aditron. Duarr Gada itu sekali lagi menghajar Aditron, Aditron pun tewas seketika.
Karena Aditron baru muncul sudah langsung mati, maka Randolf pun memberanikan diri melepaskan ikatan Mithiel yang asli, yang pertama. "Nah kali ini bener kan gw ngelepasinnya?".
Namun tiba-tiba Aditron bangkit kembali, rupanya Margonaut the Ear One, membangkitkan kembali Aditron, mengingat stok pasukannya yg minim.
Aditron berkata: "Saya tidak akan jatuh lagi ke lubang yang sama!" tapi tiba-tiba Duarr Gada yang ia ayunkan kembali menghajar dirinya terpental jauh dan jatuh ke jurang.
Aditron sekali lagi setelah baru muncul langsung mati lagi.
Margonaut The Ear One pun sudah kapok. Sebodoh-bodohnya Randolf, lebih bodoh Aditron.
"Kapan-kapan saya tidak akan menerima lagi Troll sebagai anak buah saya lagi", geram Margonaut the Ear One dari kejauhan di tempat yg tak terlihat.
Singkat cerita, Randolf, Prastudil, dan Mithiel berhasil kabur dari menara. Sementara itu Wong, Direzion, dan Anjariel masih bertempur.
Srosot! Mirfan menampar Direzion. Dhuarr! Wong ngaget-ngagetin Mirfan sampe pingsan. Slep!! Ray meracuni Wong dengan nasi padang. Brebes! Anjriel tanya sama Ray: "Brebes itu dimana, Ray?". Ah tolol! kata Prastudil.
"Coba tanya Sapion!" jawab Ray, "katanya deket-deket kampungnya di Tegaluvell" sambung Ray lagi.
Mumpung Ray sedang sibuk melayani pernyataan Anjriel yang gak mutu, Anjriel buru2 membuka Googlarium sekaligus membawa Wong dan Direzion ke tempat Randolf dkk berada. "Hihihi kapan lagi bisa ngadalin si Ray", gumam Anjariel.
"Kita harus segera pergi!" seru Prastudil. "Tenang cuy, kita punya tumpangan," kata Direzion, "bangsaku beraliansi dengan Lord Edwing, dan aku tahu sedikit bahasa mereka." Kemudian ia mencicit, dan datanglah enam tikus raksasa menghampiri mereka.
"Segera menuju Fisiphenia, ke Gedung Dewan tetua" perintah Anjariel. Ia pun segera mengirimkan surat lewat burung onta mengabarkan mereka segera menuju Fisiphenia.
"cit ciitt cit cuitt cittt (tolong antarkan kami ke fisiphenia)", kata Direzion dalam bahasa tikus. "Ciiiitttt (oke boss!)", kata keenam tikus serempak.
***

Akhirnya sampailah tikus-tikus itu di Fisiphenia bersama Fellowship dan Mithiel.
Segera mereka melaju ke arah gedung Dewan Tetua di samping Colosseum di jantung kota. Untuk ke sana mereka harus melewati lalu lintas yg sedang padat-padatnya.
"Bang2, kok macet gini sih??", tanya Randolf ke tukang teh botol di salah satu ruas jalan. "Maklum dek, ada si komo lewat! Mending adek jangan ditengah jalan, kalo gak mau keinjek", kata si tukang teh botol tanpa menoleh.
Melihat kondisi jalanan yg semakin macet parah banget. Anjariel segera melaukan telepati kepada koleganya di Kepolisian Lalu Lintas Fisiphenia untuk mengawal rombongan mereka langsung ke gedung Dewan Tetua.
Tapi pak polisi juga bingung seperti lagu dulu Macet lagi.. macet lagi.. Gara-gara sikomo lewat.. Pak polisi jadi bingung.. urang-urang ikut bingung kalau penulis tidak salah ingat, begitu lagunya.
"Sial!" Anjar pun akhirnya terpaksa mencari 6 ojek untuk mengantar mereka ke gedung Dewan Tetua dengan cepat. "Ojek emang bikin bangga" celetuk Direzion.
Akhirnya sampailah mereka ke gedung Dewan Tetua, dengan tergesa-gesa, mereka berenam masuk untuk menghadiri sidang Dewan Ketua.
Sesampainya di ruang sidang, Dewan Tetua menyambut fellowship bak pahlawan.
Segeralah mereka memulai sidang untuk mendengarkan ramalan Fitriel yg akan ditututkan Mithiel. Mithiel sendiri yg sedang asyik ngupil di pojokan, masih bingung kenapa dia bisa ada di sini, maklum selama perjalanan fellowship tak sekalipun menjelaskan.
Akhirnya Direzion berbisik pada Mithiel was wes wos ramalan was wes wos Satron was wes wos akhirnya... Mithiel jadi geli-geli asik (kupingnya).
"Tok-tok!" Ketua Dewan Tetua membuka sidang, "Kali ini kita akan mendengarkan ramalan penting mengenai keberlangsungan negeri kita yg menyangkut hajat hidup orang banyak, untuk itu saudari Mithiel kami mohon untuk segera naik ke mimbar" lanjutnya.
Mithiel pun segera naik ke atas mimbar, bukan mimbar untuk khotib kutbah lho. Mimbar ini berada di tengah-tengah ruang sidang, tepat persis di depan meja Ketua Dewan Tetua.
"Saudari Mithiel, langsung saja ke intinya, karena kami semua bingung mau apa lagi, bisakah Anda ceritakan ramalan dari Ibunda Anda Fitriel the White?", tanya Ketua Dewan tanpa basa-basi.
"Hmm, maksud Anda apa ciiyy? Which is gak ngerti...", ujar Mithiel..
Namun tiba-tiba lampu di gedung dewan mati semua. Mithiel yg awalnya cengar-cengir tiba-tiba terbang ke atas tak tersadarkan diri dan diselimuti sinar yg terang, mulutnya pun bertutur...
"Pecinta Yang Terjatuh tidak akan pernah bisa dibunuh oleh manusia, Tapi mereka yang terbuang akan menuntut balas...
..Ketika Putri hilang, Dia Yang Menebus Dosa akan sampai pada senjanya. Di balik mawar yang indah, tertidur sebuah bangkai yang keji...
..Pada masanya, kegelapan akan menyelimuti hati, Kebaikan dan Persamaan selama tiga zaman akan terhapus dalam beberapa tahun...
..Menara-Menara Tinggi akan diruntuhkan, Lima pilar yang suci akan kembali bersatu...
..Lima sulur yang jahat akan muncul ke permukaan, Fasilnor yang indah akan diselimuti api....
..Namun setelahnya akan tumbuh kembali bunga-bunga.".
Setelah Mithiel sadar, ia terjatuh bebas ke atas mimbar Gubrakks!! menimpa Randolf.
Tiba-tiba lampu gedung menyala kembali dan Mithiel berhasil membuat dirinya dan Randolf tak sadarkan diri.
Semua orang yang berada di dalam ruangan terdiam..Sunyi...Tatapan bingung muncul di masing-masing muka mereka..
"Notulen!!!" teriak ketua Dewan, "Tadi itu kamu catat gak?" tanyanya lagi.
Notulen mengangkat jempolnya kepada ketua Dewan.
Tidak hanya satu jempol, tetapi tiga jempol, dua jempol tangan dan satu jempol kaki nya tanpa basa basi.
Ternyata sang notulen ngasal ngangkat jempolnya. "Ya ampun, gue lupa nyatet tadi dia ngomong apaan!!", ujar Notulen Dewan Tetua..Semua orang kembali terdiam...
"Dasar bego..udah capek capek dibawa kesini malah gag di catet.." Umpat para anggota Dewan yang lain.
"Iya, masih nubi nih kakak... gebein dunk...", jawab Notulen-Nubi..
"Tenang-tenang" kata Mithiel sambil bangkit memegangi pinggangnya yg keseleo, "Saya punya hardcopy-nya kok, Ibu saya menaruh masternya di tempat fotokopian" lanjutnya.
"Trus ngapain tadi repot-repot bawa si Mithiel ke Dewan Tetua" umpat the Fellowship bersamaan.
"Pake acara pala gw kepenggal lagi, trus disambungin lagi, susah tauk itu!!" geram Anjariel of Euller.
"Hehe, gak tau juga, aku cuman nurutin kata Ibunda saja, which is harus heboh kalo menyampaikan ramalan", ujar Mithiel dengan ceria.
Sementara mereka ribut-ribut, Prastudil sedang berpikir keras tentang arti ramalan itu. Biasanya, Prastudil bisa memecahkan teka-teki semacam ini.
karena dari semua orang di situ, Prastudil adalah orang yang paling jago, secara dia ikut olimpiade matematika gitu loh.
Melihatnya berpikir, Randolf pun pura-pura ikut berpikir dengan menggaruk kepala. Wong pura-pura berpikir dengan memegang dagunya.
Prastudil mulai berbicara, "Hmm, sepertinya aku mulai mengerti apa maksud dari ramalan tersebut...Walau masih agak ragu..".
"Yaa, aku juga," timpal Randolf agar kelihatan pintar. "Aku juga," sambung Wong.
"Eee..udah deee.." kata Prastudil. "Pecinta Yang Terjatuh adalah Ar-Mour" lanjutnya sambil mengingat dan mencocokkan dengan legenda Perang Balgebunen.
"Benar sekali, sama persis dengan pikiranku," sambung Randolf. "Persis," seru Wong.
"Sebentar, saya coba koneksi ke Googlarium, biar ada referensinya" timpal Anjariel menanggapi. Kemudian dia langsung melakukan telepati ke Googlarium.
Menurut sejarah perang Balgebunen, Ar-Mour jatuh cinta pada Amelani. Tetapi ia malah membunuh Amelani. Disitulah maksudnya Ar-Mour terjatuh.
"Eh lanjutannya apa lagi?" tanya Wong, "Bentar coy, gw fotokopi dulu dari masternya di tempat fotokpian" jawab Direzion.
"Ni dia..udah gw potokopi semua, lengkap sob" kata Direzion tak lama kemudian pada Wong.
"Mereka yang terbuang bisa saja monyetnya suhu Acus, bisa juga babi atau tikus. Yang pasti bukan manusia, karena Pecinta Yang Terjatuh tidak akan pernah bisa dibunuh oleh manusia" Lanjut Prastudil.
"Atau mungkin juga biji-biji Aria the Eldest" timpal Anjariel, "Aria the Eldest......" yg lain kompak terkejut mendengar nama itu.
"Kalau begitu..." Semua mata melirik foto Satron yang terpampang di ruangan sidang.
Semua orang dalam ruangan itu terdiam lalu menerawang, menyelami pikiran masing-masing, mengingat-ingat cerita perang antara Satron dan Aria the Eldest.
"Satron sudah ditakdirkan berperang dengan Ar-Mour", Ucap Prastudil menganalisa semua kata-kata ramalan itu.
Anjariel pun menjelajahi dunia maya lewat telepatinya dengan bantuan Googlarium dia mendapatkan rekaman pertarungan Balgebunen yg dipimpin wasit Jimmy, dimana Satron dibuat terluka hatinya oleh Ar-Mour yg membunuh Amelani, di situs yg sudah diblokir.
"Hmm..'Pada masanya, kegelapan akan menyelimuti hati, Kebaikan dan Persamaan selama tiga zaman akan terhapus dalam beberapa tahun' berarti sebentar lagi dunia akan mengalami masa kekacauan' kata Wong ketakutan.
"TIDAAAAAK!" teriak Randolf, semua orang melihatnya dengan simpati. Kemudian ia berbisik kepada Wong, "tolong jempol kaki gw jangan diinjek. Ramalan sih ramalan, tapi jangan heboh gitu dong. Lebay nih...".
Sementara ribut-ribut masalah jempol, tiba-tiba Satron datang ke ruang sidang (hening sebentar), lalu Satron naik ke atas mimbar lalu berkata: "Ada apa ya ribut-ribut?".
"Hehehehehe, biasalah Tron, Anak muda" kata Randolf sok akrab, sok ayik dan sok anak muda ke Satron.
"Eee..buset" kata Satron sambil nimpuk Randolf pake piano yang entah kenapa tiba-tiba ada di situ. "Kenapa ya ribut-ribut?" lanjut Satron repost. Semua masih tampak hening, menunduk dan tidak berani menatap mata Satron yang tiba-tiba berwarna merah.
"Hehehehehe, biasalah Tron, anak gaul," kata Direzion (hampir) repost. Satron kemudian menimpuk Direzion dengan Bajaj Pulsar warna biru metalik yang tiba-tiba muncul, maklum lah Satron kan sakti. "Ada apa ribut-ribut?" katanya, repost lagi.
Prastudil lalu mendekati Satron dan berbisik "was wes wos Ar-Mour was wes wos ramalan was wes wos".
"Ah gak ada urusan saya sama lembur-lemburan!" bentak Satron, rupanya dia bolot sehingga menyangka Prastudil berbicar tentang lembur. "Terserah kalian mau ngeributin apa, saya gak ada urusan!" Satron pun ngeloyor pergi meninggalkan mereka.
Lega, itulah yang terpancar dari muka para hadirin, hal itu karena mereka menghadirkan Mithiel di sidang tanpa sepengetahuan Satron.
Mereka pun kembali mendiskusikan mengenai arti dari ramalan itu, Wong bergumam "Ketika Putri hilang, Dia Yang Menebus Dosa akan sampai pada senjanya. Di balik mawar yang indah, tertidur sebuah bangkai yang keji...", "Kira-kira apa ya maksudnya?".
"Putri Amelani! Tidak salah lagi!" Seru Prastudil mantabs.
"Dia yg Menebus Dosa, apakah itu Yang Mulia Satron?" tanya Anjariel "Jika ya berarti negeri kita dalam bahaya, karena dikatakan ia akan sampai pada senjanya, alias akan meninggal" lanjutnya.
Tiba-tiba seekor burung kakak tua hinggap di jendela, giginya tinggal dua. Dengan nafas terengah-engah ia berkata: "Ga ada angin, ga ada hujan, ga ada ojek, GAWAT JEK!! Satron pingsan".
Ternyata Satron cuma kepeleset waktu mau masuk ke istananya, dikarenakan genteng di istana bocor gara-gara disambitin sama anak2 tetangga. Setelah dibopong ke ranjangnya Satron pun siuman. "Fiuhhhh, bikin kaget aja" serempak semua orang lega.
Akhirnya setelah menyelimuti Satron yang masih letoy karena ketiban genteng, mereka kembali ke ruang sidang untuk membahas kembali arti ramalan yang diberikan oleh Mithiel.
"Jadi gimana, Pras?", tanya Randolf kepada Prastudil. Prastudil kembali melanjutkan analisisnya, "Sebentar-sebentar, sepertinya kita melupakan sesuatu..Ada satu faktor yang terlupakan..".
Tapi mereka yang terbuang akan menuntut balas... Prastudil semakin keras berpikir apa artinya itu.
''Siapakah yang terbuang? Menuntut balas atas apa? Apakah benar Satron yang menuntut balas akan kematian Amelani? Atau sebuah ras yang dulunya bersekutu dengan Ar-Mour, dan dikhianati olehnya? Atau... ayahku, Prastudon? '.
"Gag ada hubungannya sama Prastudon kaleee" kata Randolf pada Prastudil yang tiba-tiba jadi ngaco.
Randolf menjelaskan sejarah nenek moyangnya. Ayahnya, Randalf "The blackest" son of Randuin, pernah memimpin bangsa Chreibouw masuk pertempuran untuk membalas kematian Kakek Randuin kepada Ar-Mour.
Tiba-tiba atap kaca Ruang Sidang Gedung Dewan Tetua pecah, dan dari atas muncullah Ray Shaiton dan Mirfan! mereka langsung menyerang ke arah Prastudil. "Mati kau, pengkhianat!" teriak mereka serempak.
Brebet! Gerakan Ray Shaiton sangat cepat dalam sekali kedipan langsung mengunci Prastudil. "Tak ada tempat untuk pengkhianat!", seru Mirfan sambil mengayunkan kapak 212 ke arah bahu Prastudil.
Crot! Termuncrat darah merah dari terputusnya tangan kanan Prastudil.
"Ahhh", erang Prastudil kesakitan. "Habisi dia, Mirfan!", ujar Ray Shaiton sembil mempertahankan kunciannya. Seketika Mirfan mengayunkan kembali kapak 212-nya, tiba-tiba ada seberkas cahaya membutakan mereka berdua.
Anjariel melepaskan sinar laser dari telunjuknya, beruntung Mirfan dengan sigap menghadang dengan kapa 212-nya. Klontang kapak 212 Mirfan terpental ke arah tengah-tengah para Dewan Tetua, yg membuat mereka panik berhamburan.
*Cleeb!* kapak itu nancep di kepala Randolf. Prastudil langsung inisiatif menolong Randolf untuk mencabutnya *Crot!!* tapi Randolf teriak "Setan alas!!" kesakitan membuat Prastudil merasa bersalah dan *Clep!!* menancapkannya lagi.
Satron merasa terusik dengan keributan yang terjadi, lalu bergegas ke ruang sidang. "Siapa kalian, berani mengacau di sini?Bukan kah kalian pendekar dari Fasilnor?Tidak puas setelah membunuh Amelani putriku" tanya Satron sesampainya di ruang sidang.
Ray Shaiton pun langsung mengeluarkan pedang lasernya *Bleng!!!*, kemudian mengayunkan ke arah Satron dengan kecepatan tinggi berkat jurus langkah Shaiton-nya.
Satron tak mau kalah, dia juga mengeluarkan pedang lasernya,"swwwiing" suara pedang itu terdengar ke segala penjuru. Mereka kini bersiap bertarung layaknya pertarungan Star Wars menggunakan lightsaber.
"Sialan lo, bapak gak punya anak!" teriak Ray menghina. Lalu Satron hanya terdiam membatu, membisu, tiada berucap lagi. Ruangan menjadi hening hingga desak tangis Satron mulai terdengar jelas, air matanya perlahan membasahi wajahnya yang kayak setan.
Dan langit pun menjadi semakin gelap, karena Fisiphenia terkait dengan hati Satron. Para hadirin pun sadar, apa yang terjadi di Fisiphenia karena Satron sedang sedih sepeninggal Amelani.
Semua pun merasa simpati melihat keadaan Satron yg terpuruk itu. *Duerrrr!!!* tiba-tiba pintu gedung Dewan meledak dan sepasukan klan Shaiton dari batalyon Tjetjunguk masuk ke dalam gedung dewan. The battle begins.
Sungguh, sebenarnya Satron terlalu kuat bagi mereka. Satron menampar Ray tepak! hingga terpental jauh. Begitu juga kawanan shaiton yang lain dia buat tepok! pak! pok! satu per satu hilang sekejab.
Kini yang tersisa dari gerombolan Satron adalah The Margonaut, salah satu dari The Five Sages of Fasilnor, Guru Besar Ilmu Hitam Mikromagic.
***

Battle I:Satron berhadapan dengan Margonaut The Ear One.
Dengan kekuatan Signal Processor-nya Margonaut membangkitkan pasukannya untuk menyerang fellowship dan prajurit isatana Fisiphenia. Di tengah pergulatan sengit antara dua kubu Satron dan Margonaut saling berpandangan dengan nafsu membunuh di keduanya.
"Anak-anak, kalian main di luar dulu ya.." ucap Satron pada fellowship. "Iyaa.. kalian main berantem-beranteman ama fellowship di luar aja ya" ucap Margonaut pada para klan shaiton. Anak buah pun berantem di luar, bos-bos berantem di ruang sidang.
Satron memulai langkah awal, diayunkanlah pedangnya membabat perut Margonaut. Zwiing..nyaris mengenai perut Margonaut kalau dia tidak menghindar. Margonaut pun membalas.
Dia pun mengeluarkan senjata andalannya Golok Ping-Pong golok sakti yg bisa membalas semua serangan lawan. *Duar!!!* goloknya menghantam pilar gedung tempat dimana Satron berada sebelum ia menghindar bersalto ke udara.
Tapi sayang, Satron kehilangan ujung jempol kirinya. Dia terjerembab. *suasana menjadi hening* Lalu Margonaut memanfaatkan kesempatan ini, ia berlari ke arah Satron melancarkan serangannya.
Satron pun lompat dan beratraksi di udara, mengerahkan segenap daya tempurnya dan mengumpulkan kekuatannya, lalu menghujam ke arah Margonaut, brebet, tapi jduk Margonaut terkena bongkahan pilar lebih dulu. "Sial" kata Margonaut.
Sebelum dia sempat menyiapkan kuda-kudanya, pedang Satron suad terlebih dahulu melukai wajahnya *Croott* darah segar mengalir dari wajah Margonaut.
Tak terima dengan luka di wajahnya, Margonaut membalas serangan Satron membabi buta, ganas, liar, tak terarah yang membuat Satron terpaksa mengeluarkan jurus pertahanan miliknya.
Jurus itu adalah jurus andalannya: Biji-biji intan! Lalu seribu biji pun melesat cepat menembaki Margonaut hingga kalang kabut.
Margonaut pun membalas dengan mengeluarkan jurus Signal Disturbance-nya yg membuat seribu biji itu jadi tak terkendali dan malah balik menyerang Satron. Luar biasa memang Margonaut. Kekuatan Five Sages memang bukan isapan jempol.
Tapi dari antara biji yang melesat balik, ternyata bijinya Margonaut juga ikutan. Akhirnya Satron mengeluarkan pedangnya dan membelah biji itu menjadi 2. "Wadaaaaw!!" Margonaut teriak kesakitan.
Ternyata biji itu adalah salah satu jimat dimana 25% kekuatannya tersimpan. Alhasil sekarang Life Power Margonaut berkurang 25%.
Biji-biji itu mengenai Satron, satron pun sekarat karena kelemahannya ada pada bijinya sendiri. Margonaut pun kabur keluar dan dikejar dengan susah payah oleh Satron. Satron kembali mengeluarkan jurus pamungkas, jeger, ke arah luar ruang sidang..
Jurusnya itu diarahkan kepada Margonaut Jeger! Margonaut pun sempat terjatuh dari pelariannya. Ternyata di pojokan masih ada Mirfan dan Ray, langsung saja Jeger! Jeger!' keduanya tewas seketika, tak akan hidup lagi.
Satron pun makin kalap dia mengejar Margonaut di seluruh penjuru kota. Dan dalam pengejarannya dia melepaskan sinar laser ke segala arah, membuat kota hancur dan porak-poranda di mana-mana. Matanya nanar seperti ada setan yg merasukinya.
Margonaut yang sekarat menghilang, kembali ke Fasilnor tempat Ar-Mour bertahta, di sana ada 4 Sages lainnya sedang berpesta. Mereka adalah Bobats, Chanium, Ibamiel dan Jhomon, para guru besar yang dikenal dengan The Five Sages of Fasilnor.
swii..iing Gdebug!! Margonaut tiada berdaya lagi untuk pendaratan sempurna. Ia menjatuhkan diri dari langit, di tengah pesta yang meriah. Margonaut adalah guru besar, sehingga semuanya terkejut. "Aww! Terkejut my heart!" seru pak Ibamiel.
Sementara Satron yg mulai kehabisan tenaga tertatih-tatih berjalan di gurun di batas luar kota. Dia pun jatuh terjermbab kehabisan tenaga. Namun amarahnya semakin menggelora, "Margonauuuuuut!!" teriaknya di tengah gurun.
Teriakannya menggetarkan seluruh Fasilnor dan Fisiphenia, termasuk mengusik The Five Devils yang menjadi seteru abadi The Five Sages, moment penyerangan Margonaut ini dimanfaatkan untuk bersekutu dengan Satron melawan The Five Sages dan Fasilnor.
The Five Devils adalah 5 jiwa jahat yang diam di dalam tanah kegelapan. Dia akan muncul ke permukaan melalui tubuh makhluk lainnya, yang menyerahkan dirinya kepada kekuatan mereka.
Machdion Iblis Laut Barat, tertawa gembira melihat kejadian ini. "Kita akan punya sekutu baru, kawan-kawan" serunya kepada seluruh penghuni Laut Barat.
"Hanya saja, kita butuh tubuh untuk ditempati. Kira-kira, Satron dan 4 fellowships mau gak ya?" Ucap 4 Iblis (devils) lainnya.
Salah satu dari The Five Devils itu aslinya adalah seorang Peri. Peri Sulistianto nama aslinya. Namun sejak bergabung dengan The Five Devils, nama panggungnya adalah Mozzlion Pickon.
Satu lagi adalah Helmer Genderuwo dari Gurun Timur, wujudnya berbentuk manusia berkepala domba, lebih tepatnya domba garut asli.
Ada juga Hipasdon, Iblis Pantai Selatan. Konon katanya sempat ada affair dengan ratu pantai selatan.
Dan satu lagi adalah the chief of Five Devils, yg terkuat dari semua Five Devils. Berdiam di dalam jurang di tengah-tengah dunia. Yg Terkuat dari semua Five Devils,kesaktiannya bahkan bersaing ketat dengan Aria the Eldest saat dulu ia belum introvert.
Beliau adalah yang terhormat Tuan muda Fuadon Rosmon Hidayaton McFaddenon, atau biasa disebut Iblis Fu.
Iblis Fu ini adalah penguasa Underworld. Dialah yg tiba-tiba membuat lubang misterius dahulu kala di tengah2 perang balgebunen yg menyedot masuk banyak ksatria yg bertarung.
***

The Return of the Five Devils.
Iblis Fu berencana mengadakan pertemuan dengan semua devils. Oleh karena itu, dia mengetik perintah: "NET SEND * KUMPUL COY!". Dan terdengarlah rentetan bunyi "teet" itu, disertai gerutuan, "Maba bego, pake net send bintang lagi nih!".
*Duar!!!* terdengar ledakan dari arah setiap orang yg menggerutu. "Apa Aku tanya pendapatmu?" tanya Iblis Fu dengan mata nanar ke setiap orang yg tadi menggerutu.
Mereka berembug, untuk negosiasi kontrak dengan Satron dan para Fellowships. Setelah rencana matang, akhirnya Helmer merasuki seekor kelinci dan berangkat ke Fisiphenia, dimana Satron dan Fellowship berada.
Kelinci yang sudah dirasuki Helmer menghadapi banyak godaan dari ayam kampus yang berseliweran di Fisiphenia. Namun Helmer tetap teguh melanjutkan perjalanan mencari Satron dan Fellowship.
Tentu saja itu karena Helmer adalah penyuka sesama jenis, makanya ia lebih dikenal dengan sebutan Iblis Bencong dari Gurun Timur.
Setibanya di gedung dewan tetua, Helmer dikejutkan dengan segerombolan Tikus2 raksasa yang sedang kelaparan di parkiran gedung. "Ciiitt citt ciiiiittt (aseekk! MAKAN MAKAN!!)", kata tikus2 tsb serempak ketika melihat Helmer.
Tepat sebelum tikus-tikus raksasa itu menyergap kelinci, Helmer pindah dan merasuki biji salak yang tergeletak di sana. Dan menggelindinglah dia sampai ke depan kamar Satron.
Tiba-tiba biji salak itu terinjak oleh pelayan Satron yang baru keluar dari kamar. "Aduuuh kepreet" kata biji salak yang tidak lain tidak bukan adalah Helmer.
Pelayan Satron itu heran, Perasaan tadi udah bersih deh kamarnya Tuan Satron, tapi kok ada biji salak ya? Tidak mau ambil pusing, biji salaknya dibuang ke tong sampah. Sampah-sampah akan segera dibuang ke dekat gurun, sekitar batas luar kota.
Helmer ingin kabur ke tubuh pelayan ini. Tapi sayangnya, Iblis tidak akan mampu memasuki tubuh manusia tanpa mereka menginginkannya sendiri.
Tidak lama kemudian, petugas pengangkut sampah memasukkan isi tong sampah berisi biji salak tersebut ke truk sampah. Truk kemudian dibawa melintasi gurun. Ke arah tempat dimana Satron sedang merenung, setelah kegagalannya menangkap Margonaut...
Supir Truk sampah ini masih newbie, tapi sok-sok ngebut secepat kentut. Tiba-tiba Truk tersandung batu dan "gubrak!" tuuh kaan jatuh, Truknya jadi terbalik-balik di Gurun.
Dan biji salak yg diisi Helmer pun menggelinding ke arah Satron. Satron yang sedang di tengah gurun sendirian dan kesepian pun kaget melihat ada sebuah biji salak di sebelahnya. Karena perutnya keroncongan, akhirnya dia pun memakannya.
Saat mengunyah biji tersebut Satron bergumam, "Hmm, biji yg aneh, rasanya tidak seperti biji". "Teksturnya lembut, sedikit sepet-sepet asoy, tapi merekah di dalam mulut". "Pokoknya maknyus pemirsa", lanjut Satron. Tiba-tiba Satron merasa pusing..
.. hingga Satron pun tertidur. Di dalam tidurnya, Helmer menemui Satron dan berbincang-bincang menawarkan sebuah kontrak persetubuhan. Emm.. Maksud penulis adalah kontrak persekutuan dan perizinan untuk memasuki tubuh mereka sebagai media persekutuan.
"Aku bisa membantu engkau mengalahkan Margonaut dan The Five Sages, namun aku perlu mengendalikan tubuhmu dan kekuatan-kekuatanmu", tawar Helmer kepada Satron.
"Selain itu aku akan memberikanmu kekuatan untuk membalaskan dendam mu pada Fasilnor, yang telah berjasa membunuh Putri tercintamu. Bagaimana, Satron?", Tambah Helmer. Satron pun terdiam sejenak, namun bara api dendam mulai memercik dalam hatinya.
Beruntung Satron masih bisa mengendalikan emosinya yang hampir saja mengambil alih akal sehatnya. "Bagaimana aku bisa yakin kau tidak akan mengkhianatiku setelah kita berhasil mengalahkan Margonaut?", tanya Satron penuh selidik.
"Aku akan memberikan kekuatan dan senjata rahasia yang dapat membuatmu menghancurkan Fasilnor, dan kau bisa menjadi Raja di muka Wimor ini", bujuk Helmer. Satron pun kembali memikirkan tawaran menggiurkan itu.
"Dan yang lebih penting kau akan lebih berkuasa dari siapapun, termasuk Aria the Eldest" lanjut Helmer *Jgerrr!!* mendengar nama itu hati Satron kembali berguncang, pikirannya terlambung ke masa lalu, masa2 dimana ia adalah sebulir biji hina.
Apalagi Satron adalah pembawa sifat Raja dari Aria The Eldest, sehingga keinginannya untuk menjadi Raja Sejagat Wimor sudah sejak dulu diimpikannya. Kini hati dan niatnya tampak sudah teguh, dan siap menjawab ajakan dari Helmer.
"Baiklah aku siap bekerja sama dengan kalian" jawab Satron yg telah dibutakan kekuasaan. "Kapan kita bisa mulai?" tanyanya "Sabar Satron, aku hanyalah pembawa pesan, untuk proses selanjutnya kau harus menghadap Iblis Fu" jawab Helmer.
Lalu Satron mengajak Fellowship untuk menjadi media Five Devils beraksi. Tadinya, mereka gak mau. Tapi dengan iming-iming permen, akhirnya mereka luluh juga. Satron dan Fellowship berangkat menuju..
Sebenarnya para Fellowship tidak mengetahui apa tujuan kepergian mereka dengan Satron. Mereka hanya luluh karna permen yang dikasih dan karena mereka adalah abdi dari Satron. Akhirnya mereka pun mengikuti Satron menuju tempat Iblis Fu.
Karena singgasana iblis Fu ada di dalam tanah. Mereka pun naik subway ke sana, berhubung praktis dan harga tiketnya terjangkau. Randolf yg untuk pertama kalinya akan pergi jauh sebelumnya berpamitan ke tetangganya sekaligus mengadakan selametan.
Di tengah perjalanan saat ingin membeli tiket busway ke singgasana Iblis Fu, tiba-tiba Fellowship tersadar bahwa..
.. mereka sudah berjalan ke arah yang salah. Mereka mengetahui bahwa Iblis Fu adalah iblis yg sangat liar. Dialah penyebab bencana alam akhir-akhir ini di Wimor. Tetapi apa daya, mereka harus mengikuti perintah dari tuan mereka, Satron.
Lalu tibalah mereka semua di pintu gerbang Singgasana Iblis. Entah kenapa, denyut nadi terasa deras dan jantung semakin berdegap-degup.
Singgasana Iblis sungguh menyeramkan. Di samping pintu gerbang terdapat tulang belulang tergantung tak berdaya, seperti habis disiksa. Di atas pintu gerbang terdapat berbagai macam tengkorak kepala binatang. "Hiss sungguh menyeramkan", batin mereka.
Penghuni Singgasan Iblis pun tidak kalah seramnya. Terlihat di sebelah kiri sepasukan Goblin dan Orc berkulit merah sedang berbaris rapi. Di atas mereka berterbangan sejumlah Gargoyle yang cukup besar.
Yang lebih menyeramkan lagi, Satron dan para Fellowship disambut oleh bidadari bencong tanpa busana menuju jantung Singgasana.
Randolf hanya bisa menelan ludah melihat para bidadari penyambut mereka. "Wah gede-gede yah!" seru Randolf ketika matanya tertuju kepada...
.... deretan bangunan di kanan kirinya. Bentuknya tidak karuan, dengan sisi tajam seperti duri. Aura kegelapan sangat terasa di sini. Tapi anehnya Randolf merasa nyaman, mungkin karena berbau gelap-gelap itu.
Rasa nyaman itu tiba-tiba hilang ketika melihat sosok-sosok yang mendekat. Tapi tidak, tidak mungkin! Mereka telah mati dalam perang Balgebunen! Prastudon, dan...Randalf?
Mengetahui sosok yang dia dekati adalah Randolf anaknya, Randalf pun berusaha melepaskan diri dari dekapan Prastudon. "Malu ah, Bang!" lirih Randalf pelan kepada Prastudon. Sementara Randalf masih seperti tidak percaya dengan apa yang dilihatnya.
"Ayah...? Aku pikir ayah dibunuh dalam Perang Balgebunen? Dulu ibu yang cerita lho..." , tanya Randolf kepada randalf. Di saat yang sama, Prastudil (luka pada tangan kanannya sudah disembuhkan) menanyakan hal yang sama kepada Prastudon.
"Ayah belum mati nak, Ayah memang ditelan lubang hitam sewaktu Perang Balgebunen. Sejak saat itu, Ayah hidup di Underworld, tanpa memakai underwear, tapi kamu jangan underestimate, understand?" Randalf berusaha menjelaskan keadaannya yang nista itu.
"Anakku", ujar Randalf." "Ayah diperbudak iblis Fu karena kalah maen pingpong"."Tapi apa daya, terpaksa kulakukan demi sesuap biji".
"Sedangkan aku kalah dalam bermain WE.", jelas Prastudon kepada Prastudil, "Dan selain itu, kebetulan Iblis Fu adalah ketua EAL, jadi aku harus mengikuti segala perintahnya. Ngoding, ikut Futorial, bikin runes. Selama bertahun-tahun terakhir ini".
"Sungguh kejam iblis itu.. beraninya dia melakukan hal terkutuk itu kepada ayahku yang sudah cukup terkutuk.." sungut randolf.
"Tenang anakku, selalu ada hikmah di balik sengsara. Ayah jadi bisa kenal lebih dekat dengan Bang Prastudon, ternyata dia lelaki yang baik", Randalf coba menenangkan anaknya.
"..dia bisa jadi ibumu yang baik nak...", sambung Randalf.
Ah, tolol! (tm) pikir Prastudil, mengapa ayahku tidak mengadu bermain catur saja melawan iblis Fu? Pasti terbebas deh!.
"Heh, sudah, sudah! Reuni keluarganya nanti aja!" teriak Iblis Fu kesal.
"Itu yang kribo iket aja, ngobrol mulu, bikin lama", lanjut Iblis Fu. Rupanya di Underworld Iblis Fu bisa mengetahui semua kejadian dari jauh. "Ayo jalan Tuan Iblis Fu sudah menunggu", kata salah satu bidadari bencong.
Satron, serta Fellowship \& keluarga berjalan menuju ruangan pribadi milik Iblis Fu, yang dinamakannya My Incineration Center, atau singkatnya MIC. Pusat untuk membakar sampah yang terproduksi banyak sekali akhir-akhir ini...
"Gak puasa lu ya?" celetuk Randolf pada seseorang yang dilihatnya sedang minum dari sebuah cawan hitam. Sosok itu kemudian berbalik, dengan tatapan matanya yang tajam. Dialah Iblis Fu.
"APA AKU TANYA PENDAPATMU!??" Bentaknya menggelegar dan membuat singgasana bergoncang, menandakan kekuasaan dan kedahsyatan Iblis Fu ditakuti di singgasana.
"Ampun Oom... Ampun Oom...", Randolf berujar seraya ketakutan. Meskipun kata-katanya malah terdengar seperti bocah yang hendak dicabuli seorang paedofil.
Satron, tamu yang paling dihormati saat itu, berusaha menenangkan Iblis Fu dengan berucap: "Alangkah baiknya kita langsung berbincang membahas persekutuan kita yang hebat ini".
"Baiklah..", nada bicara Iblis Fu mulai menurun. "Ada apa gerangan kiranya Ki Sanak jauh-jauh datang kemari?" lanjut Iblis Fu.
"Aku dan keempat orang ini," Satron menunjuk Wong Cen Lau, Randolf, Prastudil, dan Direzon, "bersedia untuk meminjamkan tubuh kami sebagai medium bagi the Five Devils, dengan syarat kami nantinya akan menguasai Wimor, sekaligus membalas dendamku!".
"Dan kami ingin ayah kami dibebaskan", ujar Randolf dan Prastudon. "Heh berani sekali kalian mengajukan permintaan seperti ini padaku, si Raja Iblis", Teriak Iblis Fu. "Untuk menguji loyalitas dan niat kalian, kalian harus dites terlebih dahulu".
"Gag bisa kurang lagi tuh Gan?" ceplos Mozzlion Pickon menyela Iblis Fu.
"Diam kau, Peri!" bentak Iblis Fu. "Baiklah, kalau begitu, kalian berlima akan diuji oleh dua suhu kami, A Lin dan A Num. Bila kalian lulus semua, tawaran sebelumnya saya terima. Tapi bila tidak... kalian akan merasakan akibatnya", Iblis Fu mengancam.
Mendengar nama suhu A Lin dan suhu A Num yang legendaris itu, Prastudil menjadi sumringah. Semua mata menjadi terkesima melihat wajahnya menjadi secerah embun pagi hari.
Singkat cerita supaya petualangan kita tidak panjang-panjang amat tapi bosan, Prastudil seorang diri berhasil melewati tes A Lin dan A Num dengan gemilang. Masing-masing lulus dengan nilai A+ dan predikat cum laude.
Satron, Wong Cen Lau, Randolf, dan Direzion pun merasakan akibatnya, yaitu...
Mendapat nilai I dikarenaka mereka tidak boleh mengikuti ujian akhir tes tersebut akibat absensi mereka kurang dari 75%. Mereka pun harus mengikuti tes yang lain sedangkan Prastudil dipersilahkan untuk bersenang-senang dahulu sambil menunggu yg lain.
Mereka harus mengikuti tes A Dbis dengan instrukturnya adalah tokoh yang sangat terkenal di dunia bisnis dangdut, Bang Toyib. Tapi karena sudah 3 kali puasa dan 3 kali lebaran Bang Toyib tidak pulang, maka tesnya pun diganti dengan...
Sebuah tes yang lebih berat lagi bersama Suhu A Num. Namun, kali ini mereka meminta nasihat-nasihat dari Prastudil sebelum tes dimulai, dan akhirnya mereka semua lulus! Iblis Fu merasa puas. "Baiklah, tawaran kalian saya terima", ujarnya.
"Tapi Tuan, masa kita memberikan dunia Wimor untuk mereka kuasai nanti?" sergah Hipasdon. Iblis Fu hanya mengedipkan sebelah matanya kepada Hipasdon untuk memberi tanda. Tapi sayang bidadari bencong di belakang Hipasdon melihatnya, dan menjadi Ge-eR.
Lalu Iblis Fu berkata pada Satron: "Satron, saya akan bergabung dengan kamu. Dan kita akan menjadi Iblis Futron". Iblis Hipasdon join dengan Prastudil menjadi Iblis Hipasdil, Iblis Machdon join dengan Randalf menjadi Iblis Machdalf.
"Tunggu dulu!", teriak Helmer, "Satron harusnya bergabung denganku! Kami sudah sepakat!" Iblis Fu menyeringai, "Boss dapet pilihan pertama dong. Udah, gabung sama Direzion sana, jadi Iblis Helmion".
Iblis Mozzlion Pickon jadi kebingungan: "Saya sama siapa, dong?" tanyanya. Karena dilihatnya fellowship yang tersisa adalah Wong, maka Mozzlion Pickon join dengan Wong menjadi Iblis Wong Pickon.
"Hahah, lengkap sudah pasukan kita, skarang hanya butuh Mithiel dan ramalannya untuk mendapatkan senjata rahasia kita", Kata Iblis Futron. Para iblis pun berpisah Machdalf dan Helmion menjemput Mithiel, sisanya menuju Fasilnor untuk berperang.
***

Final Battle: The Five Devils vs. The Five Sages.
Sementara itu di daratan Fasilnor, The Five Sages sedang duduk ngopi.
"There's something wrong!" kata Chanium kepada Jhomon. Entah apa yang dimaksud Chanium saat itu.
Chanium merasakan 3 kekuatan besar sedang bergerak menuju Fasilnor.
"Apa yg kau rasakan Chanium?" tanyanya sambil ngopi, "Kelihatannya kau gusar sekali?" kali ini sambil menenggak habis kopinya.
"3 kekuatan jahat yang sangat besar!" ucap Chanium sembari terlompat dari tempat duduknya, membuat isi gelas kopinya tertumpah. Para Sages terkaget dibuatnya, kemudian berusaha merasakan datangnya kekuatan tersebut.
"Tapi tunggu dulu!" tiba-tiba Bobats menyela sambil kembali menuang kopi ke gelasnya. "Ada apa Bats?" ketus Jhomon yg tersedak akibat dikagetkan Bobats. "Ada 2 kekuatan lagi yg sedang mengarah ke sekolah Azzura Rosso!" jawab Bobats. "Hah?" ....
"Itu pasti Satron, tapi tak mungkin ia memiliki hawa iblis sedahsyat ini, apakah mungkin ia..", gumam Margonaut. "Lebih baik kita segera mensummon pasukan kita dan berangkat ke 2 tempat tersebut. Aku merasa ada hal yang tak beres di balik semua ini.".
"Margonaut, bagaimana kondisi Mithiel? Apa kau berhasil mencegah Satron mendengar ramalan itu" tanya Chanium. "Sial! aku lupa" jawab Margonaut. Sementara itu Anjariel sedang pusing bagaimana caranya mengembalikan Mithiel kembali ke sekolahnya.
Mithiel ngambek which is dia males banget kembali ke sekolah.
Tanpa basabasi, Anjariel langsung memasukkan Mithiel ke dalam karung lalu menaikkannya ke punggung Tikus2 raksasa kemudian menyuruh mereka berangkat ke sekolah. Sementara itu, Machdalf dan Helmion menunggu di sekolah sampai jamuran.
"Akhirnya datang juga!", ujar Machdalf. Anjariel terperanjat melihat kedua sosok mengerikan tersebut dan tiba-tiba ::Brebet!!:: Dalam sekejap, Anjariel telah terjatuh dari tikusnya dan Mithiel telah dibawa kabur oleh kedua iblis.
"Piwwiit" Anjariel pun bersiul memanggil kereta kudanya yg mampu berlari melampaui kecepatan kentut. Segera ia pun melaju mengejar kedua iblis itu. Sementara Mithiel yg di dalam karung tidak menyadari kalau ia sudah berpindahtangan.
Namun memang secepat2nya kuda Anjariel berlari, tetap masih kalah cepat dibandingkan dengan kecepatan para iblis. Tidak butuh waktu lama agar mithiel dan kedua iblis lenyap dari pandangan Anjariel.
Mithiel gusar dalam gendongan Helmion yang tampan, perpaduan wajah belanda dan kelembutan orang garut. Mithiel semakin kecut, tapi matanya berbinar-binar.
Helmion dan Machdalf pun pergi secepat kilat menuju Gua Babe, dimana senjata rahasia pemusnah massal yang konon katanya terucap dalam ramalan Fitriel disimpan. Mithiel adalah kunci untuk menggunakan senjata legendaris itu.
Mereka pun meraba-raba seluruh dinding pintu gua mencari dimana lubang kuncinya. Tapi sampai dinding gua itu halus karena diraba-raba melulu, mereka tidak juga menemukan lubang kuncinya. Helmion pun marah dan membentak Mithiel "Lubangnya dimana sih?".
"Mas nanya lubang yang mana nih?", balas Mithiel sambil tersipu malu. "Ya lubang kunci lahh, lubang yg lain mah kita bicarakan di lain waktu saja". Tapi Mithiel teringat pesan ibunya agar tidak pernah membuka kunci ke Gua Babe.
Sementara itu Five Sages memutuskan untuk berpencar, Jhomon dan Ibamiel ditugaskan untuk menyelamatkan Mithiel sedangkan yg lain bersiap menghadapi iblis Fu. Jhomon dan Ibamiel segera menuju ke sekolah Mithiel, namun mereka tak menemui apapun kecuali.
Anjariel yang terkulai lemas di tanah. "Sial, Mithiel dibawa pergi oleh setan-setan alas itu", kata Jhomon. "Oh tidak, jangan2 mereka ingin membuka senjata legendaris itu?", Seru Ibamiel. Mereka berdua pun bergegas menyusul ke Gua Babe.
"Eh, [Njar] boleh minjem kereta kuda elo gak?" tanya Jhomon, "Biar lebih cepet, lagian kita dah ccapek nih lari dari tadi" lanjutnya. "Oke deh, tapi balikin ya!" kata Anjariel sambil menyerahkan tali kekang kereta kudanya.
Sementara itu di Pintu depan Gua Babe, Helmion dan Machdalf berhasil memaksa Mithiel untuk membuka gerbangnya. "Cepet buka oy, dengan senjata yg di ada di dalam, dunia akan jadi milik kami para iblis. hahahhaha grok grok", tawa Machdalf.
Mithiel terpaksa membuka pintu Gua itu, karena dia sudah tidak tahan melihat tampang menjijikkan kedua setan itu. Dia pun mengetuk pintu, dan mengucapkan "Assalamu'alaykum", dan pintu Gua pun tiba-tiba terbuka. "Wakss, gitu doang caranya?".
Begitu pintu goa terbuka,goa itu bergemuruh, sebuah lorong gelap dan berbau anyir memanjang lurus ke dalam goa.
Mereka pun langsung menyusuri goa tersebut, sesekali mereka tergelincir, sebab lantai goa itu penuh dgn lumut. Namun kesigapan mereka agak kurang karena tiba-tiba Machdalf kehilangan keseimbangan dan menabrak Helmion yg kemudian menabrak pula Mithiel.
Tanpa sengaja Mithiel berteriak "E buset" dan seketika itu, ratusan obor menyala sepanjang goa dan di ujung terlihat sinar yang menyilaukan mata. "Astaga" kata Machdalf dan Hilmion bersamaan, dan seketika itu juga, obor padam. Obor yang aneh.
Nun kejauhan di sana mereka melihat seberkas cahaya yg menyilaukan. "Nah pasti itu dia!" seru Machdalf dan Helmion kompakan. Segera mereka berlari ke arah cahaya itu.
Lama sudah mereka berlari namun tak kunjung sampai. Akhirnya mereka berhenti, "Jangan-jangan pake password lagi?" kata Mahdalf pada Hilmion. "Password?" tanya Mithiel dan tiba-tiba cahaya itu mendekati mereka. Ternyata passwordnya adalah 'Password'.
Dan dalam sekejap mata, mereka bisa melihat bahwa ternyata kilatan cahaya berasal dari bilah sebuah pedang. Inilah Debian Sword yg kesohor itu.
Debian Sword adalah sebuah mahakarya dari salah seorang Five Sages of Fasilnor. Menurut legenda, Debian Sword di buat dari bahan-bahan langka, seperti ekor phoenix, tanduk unicorn, sayap pegasus yang di campur dengan logam Adamantium.
Debian Sword menurut legenda adalah pedang dimana setiap orang yang memilikinya akan mampu mempunyai kekuatan yg dapat menguasai dunia. Kekuatan yg sebanding dengan kekuatan Aria the Eldest saat dia belum menjadi introvert dulu.
Debian Sword adalah buatan Ibamiel, dan menurut legenda pula, hanya orang-orang beriman lah yang dapat mencabut pedang tersebut dari batu yang seolah-olah mencengkeram dengan kuat, ah sungguh dahsyad kekuatan pedang itu.
Helmion pun segera kalap melihat Debian Sword, dia langsung mencoba untuk mencabut pedang itu. Dia pun menarik kuat2 pedang itu sambil mengerang. "Errrrghhhh!!!" erangannya membahana di seluruh gua dan di luar gua. Membuat orang2 berpikir aneh2.
"Udah udah..abis abis..lo tu udah gagal..sini gantian gw" kata Machdalf sotoy yang dengan sigap menggenggam erat pedang itu. "uuuwwwoooh..auuuwoooo" teriaknya tapi pedang tak bergeming sedikitpun dari tempatnya.
"Tunggu dulu Kisanak!" tiba-tiba mereka dikagetkan oleh teriakan dari luar goa, kemudian mereka melihat sebuah kereta kuda dan dua penumpangnya masuk ke dalam gua, yg kemudian tergelincir yg membuat kereta kuda dan penumpangnya jatuh terguling2.
"Tunggu sebentar kisanak" kata salah seorang di antaranya sambil merapikan baju dan menolong temannya kemudian langsung berdiri tegak, sok sok jaim. "Jangan kau teruskan usahamu yang sia-sia itu sob" katanya meyakinkan Machdalf.
"Siapa kah kalian?" tanya Machdalf tanpa melepaskan pelukannya dari Debian Sword membuat orang2 agak mual melihatnya."Berani-beraninya kalian menyela kami? Hah? Hah? Hah?" lanjutnya lagi.
"Kami adalah..." kata dua orang itu serempak membentuk formasi aneh. "Ibamiel" kata salah seorang dari mereka, "dan aku Jhomon" kata lainnya. "Dengan kekuatan ilmu..akan menghukummu" lanjutnya serempak, tidak lupa dengan gaya ala Sailormoon.
"Banyak omong kalian!" sahut Helmion, Duar!!! dia pun mengayunkan kapak raksasanya, yg entah dari mana tiba2 bisa muncul ke arah Ibamiel dan Jhomon. Beruntung kedua orang itu mampu menghindar dengan bersalto ke belakang.
Jhomon pun membalas mereka dengan serangan paket datagram segede gaban yang membuat Machdalf terpental ke belakang dan langsung bersalto juga tak mau kalah namun akhirnya kejedot stalakmit gua. "Mampuy" kata Jhomon penuh dengan senyum kemenangan.
Kemudian Jhomon bersiap-siap menhajar Helmion, dia pun melakukan lompatan ke arah Helmion sambil berteriak "Yahoooook!". Duar!!! Helmion mampu menghindar sayang dia mendarat di tempat yg licin jadi dia pun terpelanting juga akhirnya.
Tak terima dirinya dipermalukan di depan Mithiel yang lagi asik nonton pertarungan sambil makan popcorn yang entah dapat darimana, Machdalf pun mengeluarkan jurus Strike Freedom "BLAAAR!!" yang mengenai kaki Jhomon karena terlambat menghindar.
"Eh gantian dong, gw blom nterang2 nih dari tadi" tiba2 Ibamiel menceletuk. Dan dengan jurus kunyuk-nya dia pun menyerang sekaligus Helmion dan Machdalf.
Jurus Kunyuk Ibamiel sangat hebat, sehebat Jurus Mabuknya Jet Lee. Helmion pun terkena cakaran di mukanya. Helmion marah, karena asetnya yang berharga di rusak, dan langsung mengayunkan kapaknya mengenai dahi Ibamiel sehingga dahinya menjadi hitam.
Ibamiel pun terdesak, hingga ke pojokan tempat Debian Sword berada. Dia pun segera mencabut pedang itu, merasa posisinya sudah semakin sulit. Dengan mudah dia mencabut pedang itu, "Kreekkkk!!!" begitulah bunyi saat Debian Sword dicabut.
Seketika semua terdiam,goa tersebut bergemuruh, sang pedang pusaka telah bangkit dari tidurnya. Ibamiel pun menebaskan pedangnya ke arah Helmion dan ditangkis dengan kapaknya, namun kapak itu patah karena Debian Sword terbuat dari adamantium.
"Gileee!!!, masih sakti aja nih pedang" kata Ibamiel sambil memandangi pedang buatannya ribuan tahun lalu itu. Saat Ibamiel tengah asyik mengagumi pedang buatannya itu yg sudah ribuan tahun tak ditemuinya, Machdalf mengambil kesempatan untuk mencuri.
"Eits gag kena donk ah" kata Ibamiel mengelak dan mengejek Machdalf.Namun kesempatan tak di sia siakan Helmion, saat Ibamiel tengah asik mengejek Machdalf, secepat kilat diambilnya pedang itu dari tangan Ibamiel dan menghilang di ikuti Machdalf.
Dan "BREBET!!!" tiba2 Machdalf mengeluarkan jurus yg mengeluarkan jaring yg segera mengikat Ibamiel, Jhomon, dan Mithiel. Tanpa ba-bi-bu lagi Helmion dan Machdalf pun kabur menggunakan kereta kuda Anjariel yg dipinjem Jhomon tadi.
Machdalf dan Helmion pun menghela kereta kudanya secepat kilat, menuju ke markas mereka untuk menyerahkan Debian Sword ke Futron. Saking cepatnya kereta mereka bergerak, tak sadar ternyata kudanya ketinggalan di belakang kereta. Kereta yang aneh.
Sementara itu Anjariel datang pergi menyusul ke Gua Babe. Terkejutlah dia mendapati Ibamiel dan Jhomon terikat lemas. Kemudian dia pun melepaskan ikatan kedua Sages tersebut. "Kita harus cepat ke Fasilnor. Fasilnor dalam bahaya!", seru Ibamiel.
"Naik apa kita kesana?kereta kuda gw ilang" tanya Anjariel gusar. "Tenang-tenang, kita naek ini" kata Ibamiel sambil menunjukkan sebuah kunci. "Apaan tu" kata yang lain serempak. "Kunci mobil kijang gw, tu gw parkir di bawah" kata Ibamiel santai.
***

Takorilien merupakan sebuah lembah perbatasan antara Fisiphenia dan Fasilnor bagian utara. Pasukan Futron yang terdiri dari mahluk-mahluk paling mengerikan sejagat Wimor terlihat dari kejauhan dan memenuhi horizon.
Hangatnya hawa jahat mereka berhembus sampai ke lembah Takorilien yang dingin, membuat para binatang penghuni lembah keluar dari sarangnya, karena udara yang biasanya sejuk menjadi panas.
Futron berdiri di garis paling depan memimpin pasukan monster itu. Dua Iblis lainnya yaitu Wong Pickon dan Hipasdil berada di sektor kanan dan kiri Futron memimpin pasukan masing-masing yang tak kalah mengerikan dengan pasukan Futron, iblis gitu loh.
Pemandangan di bukit itu sangat mengerikan bagi penjaga benteng perbatasan. Mereka pun segeara bersiap-siap mengerahkan pasukan. Kurir tercepat pun dikeluarkan untuk meminta bantuan kepada headquarters. Penduduk benteng pun mulai diliputi rasa cemas.
Suasana sangat mencekam kelam itu ditambah langit yang menjadi gelap dan awan mendung kejar-mengejar, halilintar sambar-menyambar, serigala melonglong, anjing menggonggong, kampred berseliweran.
Tiba-tiba asap mengepul dari arah selatan benteng perbatasan. Terlihat panji-panji lambang 5 Sages bermunculan. "Itu para five sagess, kita selamat", seru seorang penjaga benteng. Dari kejauhan terlihat Chanium, Bobats, dan Margonaut memimpin pasukan.
Kemehek-mehek!! Sebelum sempat penjaga benteng itu bergirang ria, Para Iblis langsung memasang kuda-kuda untuk menembakan bola api ke arah Margonaut, Duar! Duar! Duar!! .
Bola api membuat langit bagaikan diterangi seribu matahari. Langit pun berubah warnanya menjadi silver. Inilah SilverLight.
Serangan tersebut membuat kaget kubu Sages, hampir 1/3 bagian sayap kanan pasukan mereka lenyap oleh serangan mendadak tersebut. "Jangan panik, rapatkan saf rapatkan barisan. Seranggg!!", pimpin Chanium.
Namun terlambat Hipasdil memimpin pasukannya mengepung dari arah kanan luar pertahanan Fasilnor, yg membuat pasukan itu kini terkepung. Chanium pun kini terdesak.
Merasa terdesak, Chanium pun mengeluarkan jurus Eliminasi Gauss Jordan, "Blaarrr!!!" suara dentuman terdengar menghantam pasukan Hipasdil. Pasukan Hipasdil tercerai berai.
Melihat kondisi yg menguntungkan, Margonaut bersama klan Shaitonnya menyerbu langsung ke jantung pertahanan pasukan monster. Dimana Futron yang sedang mengupil tidak menyadari bahaya yg mengarah ke dia.
Brebet!! klan Shaiton berhasil mendekap mesra Futron yang tak bisa bergeming dan masih sempet-sempetnya mengupil.
Slepet!! dengan sekali tebasan Debian Sword, seluruh pasukan klan Shaiton yang mendekap Futron berubah menjadi abu. "Anjrid, curang abis!!" teriak Margonaut tak percaya dengan apa yang dia lihat.
Ternyata tanpa disadari oleh Sages, Machdalf dan Helmion telah kembali membawa Debian Sword. Melihat kejadian ini pasukan Sages menjadi gentar, seperti orang yg mau menyeberang jalan sudah di tengah jalan bingung mau maju atau mundur.
"Woi! yang balik badan ntar gw injek2 ampe tinggal 4 Byte!", teriak Jhomon yang makin memposisikan para pasukan Sages dalam posisi terjepit. Maju kena, mundur kena *kaya judul filmnya warkop*.
Dengan tambahan pasukan Machdalf dan Helmion, pasukan iblis pun bertambah liar dan brutal. Dalam sekali serangan, benteng perbatasan berhasil dikuasai. Kubu Sages tampak gentar, mereka terpaksa mundur 20 kilo ke arah benteng Gedung C.
Futron tertawa senang atas kemenangannya. The five sages mengatur strategi dan melihat buku kurikulum, buku yang menyimpan jurus-jurus rahasia mereka. Sementara dari kejauhan, Elf wanita mengamati setiap kejadian dengan seksama.
Buku kurikulum itu sudah bapuk, terlalu lama tidak diperbaharui lagi. Ketika dibaca kembali, rasanya ada yang ganjil dalam buku kurikulum tersebut. Komposisinya jurusnya amat membingungkan dan tidak dilengkapi dengan jurus-jurus terbaru.
Buku tersebut bahkan kalah up-to-date dibanding fotokopian di Pako.
Namun, ada satu jurus klasik yang belum pernah dibahas dan dipelajari. Jurus itu tidak diajarkan di Universitas A maupun Institut Teknologi B manapun. Jurus itu adalah..
Jurus Pedati!
Ke esokan harinya perang pun dilanjutkan. Pasukan Futron berteriak-berteriak dengan lantang, sementara pasukan Fasilnor membuat posisi siaga. Jauh dibelakang mereka The Five Sage sedang menyiapkan Jurus Pedati yang sudah terlupakan...
Gelombang serangan pertama dimulai dengan serangan anjing-anjing liar pasukan Hipasdil, meudian diikuti serangan udara dari korps Helmion yang terdiri dari gargoyles mengerikan. Dengan sekejap serangan pertama sudah membuat barisan depan kalang kabut.
Barisan kedua pasukan Fasilnor telah bersiap, kondisi bertahan. Sesuai instruksi, mereka tidak boleh menyerang. Persiapan selesai, Jurus Pedati membuat sebuah lobang diudara dan keluarlah Odin.
Namun sebelum Odin beraksi, Iblis Machdalf sudah mengirimkan serigala betina liarnya, Asena untuk beraksi duluan. Serigala betina itu mengobrak-abrik pertahanan barisan kedua.
Odin son of Acephun adalah summon terkuat yg dimiliki Fasilnor. Dengan sigap Odin pun menangkal gelombang serangan serigala dan gargoyles Iblis. Kini pasukan Fasilnor dapat bernafas untuk beberapa detik.
" Tunggu! " teriak Iblis Futron kepada para Iblis. Para Iblis pun menahan serangannya. "Itu adalah Odin, itu adalah jurus Pedati yang hilang dari kaum Iblis".
The Five Sage bangkit dari posisi summoning. Dengan gerakan mata Chanium separuh pasukan Fasilnor beserta ke empat Five Sage maju. Odin mengamuk menghancurkan barisan depan pasukan Iblis. Futron terkejut... perang semakin kisruh...
Futron pun tak mau kalah, dia dan keempat Iblis merapal jurus sakti mandraguna yang terlarang. Jurus Biji Iblis. Dengan bantuan Debian Sword, jurus ini sangatlah mengerikan. Srebet, dengan sekali rapalan, hilanglah seluruh korps Jhomon menjadi debu.
Iblis Futron (Iblis Fu dan Satron) ditambah Debian Sword, keadaan ini memang membuat pertarungan semakin tidak berimbang.
Tetapi pasukan Numerikal Chanium tidak dapat dipandang remeh. Dalam keadaan kacau balau ini, mereka dengan tepat waktu datang dan menyerbu sayap kanan pasukan Iblis. Chanium pun terus mendesak maju ke pertahanan Futron.
Sementara itu Futron menganti siasat. Dia memerintahkan para jendralnya melawan Odin. Namun pasukannya terus digempur pasukan Fasilnor. The Five Sage tidak tinggal diam. Futron dalam keadaan terdesak.
Kini pasukan gabungan Fasilnor sudah sampai ke barisan pasukan cadangan Futron. Sementara itu 4 Iblis bahu membahu menerjang Odin, Odin pun tertahan lajunya untuk sementara. Futron pun kesal, kemudian dia pun mengeluarkan jurus simpanannya.
Futron meringis kesal... tidak pernah terpikir jurus ini akan dia keluarkan. Kembali langit menjadi terang. Ya ini Silverlight, namun terlihat dari terangnya langit jurus ini jauh lebih kuat, jauh lebih hebat dari sebelumnya.
Holy Silverlight Sonata Arctica, jurus termaut dan terlarang yg pernah diciptakan oleh kaum iblis. Langit pun mendadak berwarna silver, dan tiba-tiba petir2 menggelegar ke arah pasukan Fasilnor. Diikuti munculnya sosok besar berjubah hitam.
Dari langit sosok ini melemparkan petirnya ke arah pasukan belakang Fasilnor. Tepat pada ledakan petir tersebut, langit kembali normal. Sosok besar itu kini terlihat jelas, dia dikenal sebagai Isrovil son of Zeusese. Pembawa petaka didunia.
Petir-petir tadi hampir menyisakan hanya 1/4 pasukan Fasilnor yang bertahan. Isrovil merupakan ahli listrik bekas kerja di PLN, sehingga ia bisa membuat serangan listrik yang sangat dahsyat. Kini pertempuran kembali seimbang.
Namun kehadiran Isrovil dapat menjadi kunci kekalahan pasukan Fasilnor. Chanium mengkaji situasi tersebut. Keberuntungan sepertinya berpihak. Para jendral iblis yang sudah letih melarikan diri dari pertarungan dengan Odin.
***

Pertarungan sangat sengit dan hampir mencakup sebagian besar pusat kerajaan Fasilnor. Hutan-hutan terbakar, bangunan hancur, warga sipil yang tak tahu menahu pun kehilangan nyawa mereka. Sungguh kejam aksi para Iblis di tanah Fasilnor.
Chainum memutuskan sudah saatnya menghabisi Futron. Odin mengganti targetnya, Isrovil. Mengendari kudanya yang mampu berjalan diudara, dia berpacu, bersiap menyerang Isrovil.
Isrovil dan Odin adalah musuh bebuyutan di dunia summon-summonan, darah Isrovil pun panas melihatnya, ia pun menerjang Odin di udara. BUOOMM!! Terjadilah ledakan keras menyusul benturan kedua mahluk summon tersebut.
Mahluk-mahluk summon tidak dapat bertempur di dunia kita. Mereka harus kembali ke dunia mereka. Tak ada yang tahu kisah mereka selanjutnya. Namun semua tahu bahwa The Five Sages langsung mengepung Futron dan melancarkan serangan-serangan mematikan.
4 Iblis lain mencoba mendobrak kepungan yang dilakukan oleh pasukan Fasilnor dari segala arah. Namun usaha mereka tak kunjung berbuah. Futron pun kini terkepung, tetapi semakin meliar dan membunuh semua yg ada di dekatnya.
Futron semakin kalap, dia menyerang membabi buta. Setiap gerakan akan mengirim satu orang ke akhir hidupnya. Tangannya tak lagi dapat dikontrol pikirannya. Gerakannya bagaikan tarian yang dibimbing Debian Sword. Bagaikan sebuah manual, tanpa cacat.
Korban sudah terlalu banyak berjatuhan di kedua belah pihak. Korps Jhomon dan korps Margonaut hampir sudah tak ada daya tempur, begitu sebaliknya dengan korps 4 Iblis. Futron pun kian mengganas menari dengan Debian Sword.
Tarian Futron nyaris tak kasat mata dan semakin cepat. Hanya The Five Sage yang dapat mengimbanginya. Para Jendral Iblis merasa ada yang salah dengan Futron dan ada yang salah dengan pertempuran ini.
Pertempuran yang sudah mulai tidak kelihatan ujungnya ini makin menggusarkan seluruh pasukan dari kedua belah pihak. Cukup banyak kerugian yg ditimbulkan. The fellowship pun makin gundah gulana hatinya.
Penggabungan The Fellowship dengan para iblis tidak dilakukan sepenuh hati. Tidak seperti Satron dan Iblis Fu. Kadang kala terjadi peperangan dalam diri Jendral-jendral Iblis ini, salah satunya adalah saat ini.
Kesadaran the Fellowship sepertinya muncul, pikiran-pikiran mereka tidak lagi dikuasai oleh para iblis, entah mengapa, semua terasa begitu janggal.
Namun pengaruh para Iblis semakin kuat, menyadari tubuh inangnya menolak, mereka pun semakin mencengkeram jiwa-jiwa para fellowship.
Fellowship makin merasa yakin kalau mereka di bodohi oleh para iblis. The Fellowship pun berusaha sekuat tenaga melawan cengekraman para Iblis, yang akhirnya mereka berhasil memperoleh kesadaran masing-masing. Para Iblis pun lepas dari tubuh mereka.
Kilatan cahaya yang silau meliputi proses terlepasnya para Iblis dari tubuh inangnya. Keempat fellowship pun jatuh pingsan selepas para iblis keluar dari tubuh mereka. Keempat iblis yg telah keluar itu.
Melayang-layang di arena pertempuran tanpa arah. Tanpa tubuh inangnya, para iblis hanyalah sekumpulan makhluk yang tidak berdaya. Mereka pun melayang-layang mendekati Futron yang sedang bertarung dengan Five Sages.
Fu yang merasakan bahwa Jendral Iblis telah kehilangan pengaruh mereka pada The Fellowship menarik mereka masuk ke dalam tubuh Satron. Futron pun semakin bertambah kuat.
Bersatunya kelima iblis ini dalam satu tubuh membuat tubuh inangnya mengeluarkan listrik, seperti Son Goku dengan super saiya 3 nya tapi bedanya Futron tidak menjadi gondrong. Ctar ctar, suara kilatan listrik di sekeliling tubuhnya sangat keras.
BLAR!!! seketika tubuh Futron bercahaya terang. Hawa Iblis tercium sangat pekat, membuat pasukan yg tidak kuat imannya tertunduk lemas. Aura Ultimate Devil memang sungguh dahsyat.
Kini Futron semakin liar lagi, satu sabetan pedang Debian Swordnya dapat menebas hingga 500 meter dan ratusan prajurit pun rubuh. Hentakan kakinya membuat tanah disekitarnya retak. Teriakkannya membuat orang dalam radius 5 kilo menjadi gila.
Ultimate Devil plus Satron dan Debian Sword, sungguh sebuah kombinasi yang sangat mengerikan. Futron berubah menjadi salah satu makhluk terkuat di alam semesta, selain Aria The Eldest, Ar-Mour dan Yulion.
Five Sages pun bahu membahu melawan Futron, mereka terus menerus menggempurnya dari semua arah. Tapi Futron begitu kuat hingga serangan Sages tidak berasa, hanya seperti sentilan saja.
Tidak ada prajurit ataupun jendral yg berani mendekati Futron. Namun tiba-tiba Anjariel muncul dari kerumunan pasukan, hanya dialah seorang yg berani, karena dia tidak dapat mati lagi. Namun saat sudah berhadapan dgn Futron, dia berhenti dan diam...
Dia pun mengingat kontrak nyawanya dengan Satron. "haha Anjariel, bergabunglah denganku. Kau adalah milikku", Futron berkata. Sementara itu Fellowship sudah sadarkan diri dan menyiapkan diri untuk berhadapan dengan Futron.
Fellowship bersiap-siap, masing-masing mengeluarkan senjata pamungkasnya, memasang kuda-kuda untuk bertempur dalam pertempuran hidup dan mati ini. Mereka terperanjat, melihat tanah di sekeliling mereka hancur, retak dan porak poranda.
Terlebih lagi mereka melihat Anjariel sahabat perjalanan mereka di kubu Futron. Mau tak mau mereka kini harus menghadapinya. Sungguh pedih memang perang ini, tidak ada lagi kawan maupun lawan. Sungguh keji.
Seseorang yang memang terus mengamati dari jauh sekarang telah berdiri. Dia tahu sesuatu yang penting akan segera terjadi. Sesuatu yang menentukan nasib Fasilnor, bahkan nasib Wilmor.
Dialah Aria the Eldest akhirnya dia memutuskan unutk meninggalkan dunia autis-nya dan berkontribusi untuk menyelamatkan nasib semesta.
"Aku adalah Aria the Eldest" kata Aria ketika muncul di tengah-tengah arena pertempuran. Turunnya Aria the Eldest dari singgasananya membuat semua orang terkejut. Baru pertama kali ini mereka melihat sosok sang Legenda, kecuali Satron bekas bijinya.
Namun ketika dia menyadari banyak sekali orang di padang Takorilien. Aria agak ragu, hasrat autisnya bergejolak, apalagi saat dia menjadi pusat perhatian seluruh petarung di situ.
Akhirnya Aria melarikan diri dan kembali ke khayangan. Futron semakin bergejolak setelah melihat Aria. Energi penghancur yang sangat kuat terasa. Dengan yakin dia tancapkan Debian Sword ke tanah. Tanah meretak dan gunung gundah ingin meledak.
Bledar..Duaaaar..Door..Kratak Kratak, gunung gunung di Wilmor bergejolak, memuntahkan isi perutnya, lahar dimana-mana, langit di selimuti kabut hitam kelam.
Semua tertegun... inikah kekuatan Futron atau ini adalah kekuatan Debian Sword. Hanya The Five Sage dan Anjariel yang mengerti bahwa ini kekuatan Debian Sword. Pedang keramat yang seharusnya sudah dilupakan.
Kini kehancuran Wimor tinggal menunggu waktu... Akankah itu terjadi... Di manakah kini pahlawan pujaan hati yg dapat menyelamatkan Wimor dari kehancuran...
Di tengah kekacauan yang terjadi, para prajurit setan Futron kembali bangkit dari kubur dan makin menimbulkan kekacauan di seluruh penjuru Wimor. Orang2 tak berdosa menjadi korban. Fellowship pun merasa bersalah karena telah membangunkan Para Iblis.
Sebagai makhluk yang masih mencintai kehidupan di Wimor pun berteriak dalam hati "Selamatkan Wimor..!" The Five Sage, The Fellowship dan Aria The Eldest, satu-satunya pilihan mereka adalah bersatu padu melawan Futron.
Keadaan telah berubah, pertempuran antara Iblis dengan Fasilnor telah menjadi peperangan untuk mengentikan Futron dan menyelamatkan Wilmor. Debian Sword pun ditarik, napas-napas terhenti sesaat. Semua memerah karena warna lahar.
Tiba-tiba langit terbakar api, ketika Debian Sword diangkat tinggi.
"Ini akan berakhir sekarang" dan guntur bergelegar disekitar Futron. Tanah-tanah terangkat. The Five Sage tak mampu bergerak. Hanya Ibamiel yang tahu ini memang yang terakhir.
Namun Ibamiel memilih diam, sambil memperhatikan daratan Fasilnor yang indah telah diselimuti oleh api.
DOENNNNG tiba-tiba terdengar bunyi sangkakala dari arah laut. Semua orang menengok. Tiba-tiba muncul satu armada kapal penuh dengan pasukan. Di kapal paling depan di atas haluan berdiri dengan gagah pahlawan pujaan hati Ar-Mour.
Ar-Mour sang pahlawan dalam legenda telah datang, bersama pasukan pilihan dari Fasilnor. Ar-Mour datang untuk menumpas Futron, yang telah membuat WIlmor jadi porak poranda. Sungguh mulia memang ksatria kita satu ini.
Pasukan Ar-Mour yang berpakaian putih mengkilat dengan cepat menerjang pasukan Futron. Terjangan itu membawa angin segar bagi pasukan gabungan Wimor dan semangat pun kembali membara di hati mereka untuk menghancurkan Futron dan menyelematkan Wimor.
Terjangan yg sangat dahsyat. Pukulan telak pun dirasakan kubu Futron... Keadaan semakin tidak menguntungkan mereka, ketika kapal2 perang pasukan Ar Mour menembakkan meriam2 mereka ke kubu2 pertahanan pasukan Futron.
Namun, Helmion mengetahui kelemahan Ar Mour. Ar Mour tidak tahan geli, apalagi kalo digelitikin di daerah pinggangnya!
Karena tidak tahan geli, Ar-Mour pun mengeluarkan jurus maut Jurus Ganteng Bersertifikat. Dalam sekejap jurus itu melibas kawan maupun lawan. Dunia pun semakin kelam, kehancuran dunia hanya tinggal menunggu waktu saja.
Kehancuran dunia diprediksi akan terjadi sebelum matahari terbit keesokan harinya. Jurus Ganteng Bersertifikat tidak dapat dikalahkan pasukan Futron. Namun, pada tengah malam, terjadi keajaiban: masa berlaku sertifikat ganteng Ar-Mour habis!
Merasa mendapat momen untuk melakukan serangan balik. Pasukan Futron langsung mengumpulkan sisa2 pasukan dan tenaga mereka. Futron pun mulai menyerang sambil melayang di udara dia pun melontarkan jurus bola api gattling gun ke arah pasukan Ar-Mour.
Sementara Ar-Mour kebingungan. Memperpanjang sertifikat gantengnya butuh tanda tangan ketua RT, dan membutuhkan waktu proses satu minggu tambahan. Birokrasi oh birokrasi...
Pasukan Ar-Mour dan para Sages kocar-kacir menghindari serangan gattling gun Futron. Suasana sangat mencekam karena hujan bola api terus menerus menghujam padang pertempuran. Bahkan saking kalapnya Futron sampai menyerang pasukannya sendiri.
"Bos Futron, kok nyerang kita sih?" salah satu pasukannya ngomel. "Oh.." Bos Futron mikir dulu cari alasan apa. "Sori.. sori.. salah pencet".
Semakin lama... kondisi semakin kacau... pasukan yg bertarung sudah tidak kenal mana lawan mana kawan. Cerita-cerita dan legenda yg dituturkan pun simpang siur. Tidak ada yg tahu pasti kelanjutan pertarungan dahsyat ini.
Ar-Mour bilang: "Gimana sih nih. penulisnya pada pikun semua yak!!?".
Hal ini disebabkan dulu setiap seseorangbercerita ttg kisah Yuli Saga di tengah perbincangan yang hangat, tiba2 orang2 akan langsung kehilangan gairah dan meninggalkan perbincangan. Seolah2 ada kutukan bila memperbincangkan Yuli Saga. Entah kenapa???
Lupa saya itu kenapa...
Seperti peperangannya yang carut marut, kisah peperangan ini pun carut marut. Hanya dataran penuh darah dan teriakan bagai kegilaan dari para prajurit yang terus terekam dalam kenangan.
tiba-tiba Randolf pun berteriak dengan lirih: "Bring my blood to hell, mosquitos!!". Halah, ternyata dia belum diolesin Domestos Nomos.
Karena bosen ni cerita udah setaun kok gak tamat2, akhirnya Aria the eldest memutuskan untuk mereset seluruh carut marut yang terjadi. Aria mengeluarkan skill ultimate-nya sehingg seluruh bala tentara futron (termasuk futron) lenyap tak berbekas.
"Wah... gak bisa begitu dong ar! Elo musti kontribusi juga dong. Jangan asal reset-reset aja". Tiba-tiba Futron berteriak. Ternyata dia gak keikut ter-reset.
Oleh: anwarchandra
Jadi sekarang pokoknya tinggal Aria The Eldest \& Futron. Pertarungan ini sampai tetes darah yang terakhir.

Diambil dari: \url{http://cerpenista.com/cerpen/baca/yuli\_saga\_fate\_of\_fasilnor}. 

\printindex

\end{document}
